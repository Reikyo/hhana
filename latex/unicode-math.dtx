% \iffalse
% ^^A %%%%%%%%%%%%%%%%%%%%%%%%%%%%%%%
% ^^A   SELF-EXTRACTION BEGINS HERE
% ^^A %%%%%%%%%%%%%%%%%%%%%%%%%%%%%%%
%<*internal>
\begingroup
\input docstrip.tex
\keepsilent
\let\MetaPrefix\DoubleperCent
\declarepreamble\texpreamble
Copyright 2006-2012   Will Robertson <will.robertson@latex-project.org>
Copyright 2010-2011 Philipp Stephani <st_philipp@yahoo.de>
Copyright 2012          Khaled Hosny <khaledhosny@eglug.org>

This package is free software and may be redistributed and/or modified under
the conditions of the LaTeX Project Public License, version 1.3c or higher
(your choice): <http://www.latex-project.org/lppl/>.

This work is "author-maintained" by Will Robertson.
\endpreamble
\def\MetaPrefix{--}
\declarepreamble\luapreamble
Copyright 2006-2011 Will Robertson <will.robertson@latex-project.org>
Copyright 2010-2011 Philipp Stephani <st_philipp@yahoo.de>
Copyright 2012 Khaled Hosny <khaledhosny@eglug.org>

This package is free software and may be redistributed and/or modified under
the conditions of the LaTeX Project Public License, version 1.3c or higher
(your choice): <http://www.latex-project.org/lppl/>.

This work is "author-maintained" by Will Robertson.
\endpreamble
\nopostamble
\askforoverwritefalse
\let\MetaPrefix\DoubleperCent
\usepreamble\texpreamble
\generate{\file{unicode-math.sty}{
  \from{unicode-math.dtx}{preamble}
  \from{unicode-math.dtx}{msg}
  \from{unicode-math.dtx}{load}
}}
\generate{\file{unicode-math-xetex.sty}{
  \from{unicode-math.dtx}{package,XE}
}}
\generate{\file{unicode-math-luatex.sty}{
  \from{unicode-math.dtx}{package,LU}
}}
\def\MetaPrefix{--}
\usepreamble\luapreamble
\generate{\file{unicode-math.lua}{\from{unicode-math.dtx}{lua}}}
\let\MetaPrefix\DoubleperCent
\nopreamble
\def\tempa{plain}
\ifx\tempa\fmtname\endgroup\expandafter\bye\fi
\generate{\file{dtx-style.sty}{\from{\jobname.dtx}{dtx-style}}}
\endgroup
\ProvidesFile{unicode-math.dtx}
%</internal>
%<preamble&!XE&!LU>\ProvidesPackage{unicode-math}
%<preamble&XE>\ProvidesPackage{unicode-math-xetex}
%<preamble&LU>\ProvidesPackage{unicode-math-luatex}
%<*preamble>
  [2012/05/30 v0.7 Unicode maths in XeLaTeX and LuaLaTeX]
%</preamble>
%<*internal>
\documentclass[a4paper]{ltxdoc}
\usepackage{dtx-style}
\begin{document}
  \DocInput{\jobname.dtx}
\end{document}
%</internal>
% \fi
%
% ^^A %%%%%%%%%%%%%%%%%%%%%%%%%%%%%
% ^^A   DOCUMENTATION BEGINS HERE
% ^^A %%%%%%%%%%%%%%%%%%%%%%%%%%%%%
%
% \title{Experimental Unicode mathematical typesetting: The \pkg{unicode-math} package}
% \author{Will Robertson, Philipp Stephani and Khaled Hosny\\
%         \texttt{will.robertson@latex-project.org}}
% \date{\umfiledate \qquad \umfileversion}
%
% \maketitle
%
% \begin{abstract}
% \noindent
% This is the first incarnation of the \pkg{unicode-math} package, which is
% intended to be a complete implementation of Unicode
% maths for \LaTeX\ using the \XeTeX\ and Lua\TeX\ typesetting engines.
% With this package, changing maths fonts is as easy as changing
% text fonts --- and there are more and more maths fonts appearing now.
% Maths input can also be simplified with Unicode since literal glyphs may be
% entered instead of control sequences in your document source.
%
% The package provides support for both \XeTeX\ and Lua\TeX. The different
% engines provide differing levels of support for Unicode maths.
% Please let us know of any troubles.
%
% Alongside this documentation file, you should be able to find a minimal
% example demonstrating the use of the package,
% `\texttt{unimath-example.ltx}'. It also comes with a separate document,
% `\texttt{unimath-symbols.pdf}',
% containing a complete listing of mathematical symbols defined by
% \pkg{unicode-math}, including comparisons between different fonts.
%
% Finally, while the STIX fonts may be used with this package, accessing
% their alphabets in their `private user area' is not yet supported.
% (Of these additional alphabets there is a separate caligraphic design
% distinct to the script design already included.)
% Better support for the STIX fonts is planned for an upcoming revision of the
% package after any problems have been ironed out with the initial version.
%
% \end{abstract}
%
% \doparttoc\faketableofcontents
%
% \newpage
% \part{User documentation}
% \parttoc
% \section{Introduction}
%
% This document describes the \pkg{unicode-math} package, which is an
% \emph{experimental} implementation of a macro to Unicode glyph encoding for
% mathematical characters.
%
% Users who desire to specify maths alphabets only (Greek and Latin letters,
% and Arabic numerals)
% may wish to use Andrew Moschou's \pkg{mathspec} package instead.
% (\XeTeX-only at time of writing.)
%
% \section{Acknowledgements}
%
% Many thanks to:
% Microsoft for developing the mathematics extension to OpenType as part of
% Microsoft Office~2007;
% Jonathan Kew for implementing Unicode math support in \XeTeX;
% Taco Hoekwater for implementing Unicode math support in \LuaTeX;
% Barbara Beeton for her prodigious effort compiling the definitive list of Unicode math
% glyphs and their \LaTeX\ names (inventing them where necessary), and also
% for her thoughtful replies to my sometimes incessant questions;
% Philipp Stephani for extending the package to support \LuaTeX.
% Ross Moore and Chris Rowley have provided moral and technical support
% from the very early days with great insight into the issues we face trying
% to extend and use \TeX\ in the future.
% Apostolos Syropoulos, Joel Salomon, Khaled Hosny, and Mariusz Wodzicki
% have been fantastic beta testers.
%
% \section{Getting started}
%
% Load \pkg{unicode-math} as a regular \LaTeX\ package. It should be loaded
% after any other maths or font-related package in case it needs to overwrite
% their definitions. Here's an example:
% \begin{quote}
% \begin{verbatim}
% \usepackage{amsmath} % if desired
% \usepackage{unicode-math}
% \setmathfont{Asana-Math.otf}
% \end{verbatim}
% \end{quote}
%
% Three OpenType maths fonts are included by default in \TeX\ Live 2011:
% Latin Modern Math, Asana Math, and XITS Math.
% These can be loaded directly with their filename
% with both \XeLaTeX\ and \LuaLaTeX; resp.,
% \begin{quote}
% \begin{verbatim}
% \setmathfont{lmmath-regular.otf}
% \setmathfont{Asana-Math.otf}
% \setmathfont{xits-math.otf}
% \end{verbatim}
% \end{quote}
% Other OpenType maths fonts may be loaded in the usual way; please see the
% \pkg{fontspec} documentation for more information.
%
% Once the package is loaded, traditional TFM-based fonts are not supported any more;
% you can only switch to a different OpenType math font using the \cs{setmathfont} command.
% If you do not load an OpenType maths font before |\begin{document}|, Latin Modern Math (see above) will be loaded automatically.
%
%
% \subsection{Package options}
% Package options may be set when the package as loaded or at any later
% stage with the \cs{unimathsetup} command. Therefore, the following two
% examples are equivalent:
% \begin{quote}
% \begin{verbatim}
% \usepackage[math-style=TeX]{unicode-math}
% % OR
% \usepackage{unicode-math}
% \unimathsetup{math-style=TeX}
% \end{verbatim}
% \end{quote}
% Note, however, that some package options affects how maths is initialised
% and changing an option such as |math-style| will not take effect until a
% new maths font is set up.
%
% Package options may \emph{also} be used when declaring new maths fonts,
% passed via options to the \cs{setmathfont} command.
% Therefore, the following two examples are equivalent:
% \begin{quote}
% \begin{verbatim}
% \unimathsetup{math-style=TeX}
% \setmathfont{Cambria Math}
% % OR
% \setmathfont[math-style=TeX]{Cambria Math}
% \end{verbatim}
% \end{quote}
%
% A short list of package options is shown in \tabref{pkgopt}.
% See following sections for more information.
%
% \begin{table}\centering
%   \topcaption{Package options.}
%   \tablabel{pkgopt}
%   \begin{tabular}{lll}
%     \toprule
%     Option & Description & See\dots \\
%     \midrule
%     |math-style| & Style of letters & \secref{math-style} \\
%     |bold-style| & Style of bold letters & \secref{bold-style} \\
%     |sans-style| & Style of sans serif letters & \secref{sans-style} \\
%     |nabla|      & Style of the nabla symbol & \secref{nabla} \\
%     |partial|    & Style of the partial symbol & \secref{partial} \\
%     |vargreek-shape|  & Style of phi and epsilon & \secref{vargreek-shape} \\
%     |colon| & Behaviour of \cs{colon} & \secref{colon} \\
%     |slash-delimiter| & Glyph to use for `stretchy' slash & \secref{slash-delimiter} \\
%     \bottomrule
%   \end{tabular}
% \end{table}
%
% \subsection{Known issues}
%
% In some cases, \XeTeX's math support is either missing or I have not
% discovered how to access features for various types of maths construct.
% An example of this are horizontal extensible symbols, such as arrows that can
% grow longer if necessary. Behaviour with such symbols is not necessarily
% going to be consistent; please report problem areas to me.
%
% Symbols for maths characters have been inherited from the STIX project and
% may change slightly in the long term. We have tried to preserve backwards
% compatibility with \LaTeX\ conventions as best as possible; again, please
% report areas of concern.
%
% \section{Unicode maths font setup}
%
% In the ideal case, a single Unicode font will contain all maths glyphs we
% need. The file |unicode-math-table.tex| (based on Barbara Beeton's \STIX\ table)
% provides the mapping between Unicode
% maths glyphs and macro names (all 3298 — or however many — of them!). A
% single command
% \codeline{\cmd\setmathfont\oarg{font features}\marg{font name}}
% implements this for every every symbol and alphabetic variant.
% That means |x| to $x$, |\xi| to $\xi$, |\leq| to $\leq$, etc., |\mathscr{H}|
% to $\mathscr{H}$ and so on, all for Unicode glyphs within a single font.
%
% This package deals well with Unicode characters for maths
% input. This includes using literal Greek letters in formulae,
% resolving to upright or italic depending on preference.
%
% Font features specific to \pkg{unicode-math} are shown in \tabref{mathfontfeatures}.
% Package options (see \tabref{pkgopt}) may also be used.
% Other \pkg{fontspec} features are also valid.
%
% \begin{table}\centering
%   \topcaption{Maths font options.}
%   \tablabel{mathfontfeatures}
%   \begin{tabular}{lll}
%     \toprule
%     Option & Description & See\dots \\
%     \midrule
%     |range| & Style of letters & \secref{range} \\
%     |script-font| & Font to use for sub- and super-scripts & \secref{sscript} \\
%     |script-features| & Font features for sub- and super-scripts & \secref{sscript} \\
%     |sscript-font| & Font to use for nested sub- and super-scripts & \secref{sscript} \\
%     |sscript-features| & Font features for nested sub- and super-scripts & \secref{sscript} \\
%     \bottomrule
%   \end{tabular}
% \end{table}
%
% \subsection{Using multiple fonts}
% \seclabel{range}
%
% There will probably be few cases where a single Unicode maths font suffices
% (simply due to glyph coverage). The \STIX\ font comes to mind as a
% possible exception. It will therefore be necessary to delegate specific
% Unicode ranges of glyphs to separate fonts:
%   \codeline{\cmd\setmathfont|[range=|\meta{unicode range}|,|\meta{font features}|]|\marg{font name}}
% where \meta{unicode range} is a comma-separated list of Unicode slots and
% ranges such as |{"27D0-"27EB,"27FF,"295B-"297F}|. You may also use the macro
% for accessing the glyph, such as \cs{int}, or whole collection of symbols with
% the same math type, such as \cs{mathopen}, or complete math styles such as \cs{mathbb}.
% (Only numerical slots, however, can be used in ranged declarations.)
%
% \subsubsection{Control over maths alphabets}
%
% Exact control over maths alphabets can be somewhat involved.
% Here is the current plan.
% \begin{itemize}
% \item |[range=\mathbb]| to use the font for `bb' letters only.
% \item |[range=\mathbfsfit/{greek,Greek}]| for Greek lowercase and uppercase only (also with |latin|, |Latin|, |num| as possible options for Latin lower-/upper-case and numbers, resp.).
% \item |[range=\mathsfit->\mathbfsfit]| to map to different output alphabet(s) (which is rather useless right now but will become less useless in the future).
% \end{itemize}
%
% And now the trick.
% If a particular math alphabet is not defined in the font, fall back onto the lower-base plane (i.e., upright) glyphs.
% Therefore, to use an \ascii-encoded fractur font, for example, write
% \par{\centering|\setmathfont[range=\mathfrak]{SomeFracturFont}|\par}\noindent
% and because the math plane fractur glyphs will be missing, \pkg{unicode-math} will know to use the \ascii\ ones instead.
% If necessary this behaviour can be forced with |[range=\mathfrac->\mathup]|.
%
%
% \subsection{Script and scriptscript fonts/features}
% \seclabel{sscript}
%
% Cambria Math uses OpenType font features to activate smaller optical sizes
% for scriptsize and scriptscriptsize symbols (the $B$ and $C$, respectively,
% in $A_{B_C}$). Other fonts will possibly use entirely separate fonts.
%
% The features |script-font| and |sscript-font| allow alternate fonts to be
% selected for the script and scriptscript sizes, and |script-features| and
% |sscript-features| to apply different OpenType features to them.
%
% By default |script-features| is defined as |Style=MathScript| and |sscript-features| is |Style=MathScriptScript|.
% These correspond to the two levels of OpenType's |ssty| feature tag.
% If the |(s)script-features| options are specified manually, you must
% additionally specify the |Style| options as above.
%
%
% \subsection{Maths `versions'}
%
% \LaTeX\ uses a concept known as `maths versions' to switch math fonts
% mid-document.
% This is useful because it is more efficient than loading a complete maths
% font from scratch every time---especially with thousands of glyphs in the case of Unicode maths!
% The canonical example for maths versions is to select a `bold' maths font
% which might be suitable for section headings, say.
% (Not everyone agrees with this typesetting choice, though; be careful.)
%
% To select a new maths font in a particular version, use the syntax
%   \codeline{\cmd\setmathfont|[version=|\meta{version name}|,|\meta{font features}|]|\marg{font name}}
% and to switch between maths versions mid-document use the standard \LaTeX\ command
% \cmd\mathversion\marg{version name}.
%
% \section{Maths input}
%
% \XeTeX's Unicode support allows maths input through two methods. Like
% classical \TeX, macros such as \cmd\alpha, \cmd\sum, \cmd\pm, \cmd\leq, and
% so on, provide verbose access to the entire repertoire of characters defined
% by Unicode. The literal characters themselves may be used instead, for more
% readable input files.
%
% \subsection{Math `style'}
% \seclabel{math-style}
%
% Classically, \TeX\ uses italic lowercase Greek letters and \emph{upright}
% uppercase Greek letters for variables in mathematics. This is contrary to
% the \textsc{iso} standards of using italic forms for both upper- and lowercase.
% Furthermore, the French have been
% known to use upright uppercase \emph{Latin} letters as well as upright
% upper- and lowercase Greek. Finally, it is not unknown to use upright letters
% for all characters, as seen in the Euler fonts.
%
% The \pkg{unicode-math} package accommodates these possibilities with an
% interface heavily inspired by Walter Schmidt's \pkg{lucimatx} package: a
% package option \opt{math-style} that takes one of four arguments:
% \opt{TeX}, \opt{ISO}, \opt{french}, or \opt{upright} (case sensitive).
%
% The philosophy behind the interface to the mathematical alphabet symbols
% lies in \LaTeX's attempt of separating content and formatting. Because input
% source text may come from a variety of places, the upright and
% `mathematical' italic Latin and Greek alphabets are \emph{unified} from the
% point of view of having a specified meaning in the source text. That is, to
% get a mathematical ‘$x$’, either the ascii (`keyboard') letter |x| may
% be typed, or the actual Unicode character may be used. Similarly for Greek
% letters. The upright or italic forms are then chosen based on the
% |math-style| package option.
%
% If glyphs are desired that do not map as per the package option (for
% example, an upright `g' is desired but typing |$g$| yields `$g$'),
% \emph{markup} is required to specify this; to follow from the example:
% |\mathup{g}|. Maths alphabets commands such as \cmd\mathup\ are detailed
% later.
%
% \paragraph{Alternative interface}
% However, some users may not like this convention of normalising their input.
% For them, an upright |x| is an upright `x' and that's that.
% (This will be the case when obtaining source text from copy/pasting PDF or
% Microsoft Word documents, for example.)
% For these users, the |literal| option to |math-style| will effect this behaviour.
%
% The \opt{math-style} options' effects are shown in brief in \tabref{math-style}.
%
% \begin{table}
%   \centering
%   \topcaption{Effects of the \opt{math-style} package option.}
%   \tablabel{math-style}
%   \begin{tabular}{@{}>{\ttfamily}lcc@{}}
%     \toprule
%       & \multicolumn{2}{c}{Example} \\
%        \cmidrule(l){2-3}
%       \rmfamily Package option & Latin & Greek \\
%     \midrule
%       math-style=ISO & $(a,z,B,X)$ & $\mathit{(\alpha,\beta,\Gamma,\Xi)}$ \\
%       math-style=TeX & $(a,z,B,X)$ & $(\mathit\alpha,\mathit\beta,\mathup\Gamma,\mathup\Xi)$ \\
%       math-style=french & $(a,z,\mathup B,\mathup X)$ & $(\mathup\alpha,\mathup\beta,\mathup\Gamma,\mathup\Xi)$ \\
%       math-style=upright & $(\mathup a,\mathup z,\mathup B,\mathup X)$ & $(\mathup\alpha,\mathup\beta,\mathup\Gamma,\mathup\Xi)$ \\
%     \bottomrule
%   \end{tabular}
% \end{table}
%
%
% \subsection{Bold style}
% \seclabel{bold-style}
%
% Similar as in the previous section, ISO standards differ somewhat to \TeX's
% conventions (and classical typesetting) for `boldness' in mathematics. In
% the past, it has been customary to use bold \emph{upright} letters to denote
% things like vectors and matrices. For example, \( \mathbfup{M} =
% (\mitM_x,\mitM_y,\mitM_z) \). Presumably, this was due to the relatively
% scarcity of bold italic fonts in the pre-digital typesetting era. It has
% been suggested that \emph{italic} bold symbols are used nowadays instead.
%
% Bold Greek letters have simply been bold variant glyphs of their regular
% weight, as in \( \mbfitxi = (\mitxi_\mitr,\mitxi_\mitphi,\mitxi_\mittheta)
% \). Confusingly, the syntax in \LaTeX\ has been different for these two
% examples: \cmd\mathbf\ in the former (`$\mathbfup{M}$'), and \cmd\bm\ (or
% \cmd\boldsymbol, deprecated) in the latter (`$\mbfitxi$').
%
% In \pkg{unicode-math}, the \cmd\mathbf\ command works directly with both
% Greek and Latin maths alphabet characters and depending on package option
% either switches to upright for Latin letters (|bold-style=TeX|) as well or
% keeps them italic (|bold-style=ISO|).
%
% To match the package options for non-bold characters, with option
% |bold-style=upright| all bold characters are upright, and
% |bold-style=literal| does not change the upright/italic shape of the letter.
%
% Upright and italic bold mathematical letters input as direct Unicode
% characters are normalised with the same rules. For example, with
% |bold-style=TeX|, a literal bold italic latin character will be typeset
% upright.
%
% Note that \opt{bold-style} is independent of \opt{math-style}, although if
% the former is not specified then sensible defaults are chosen based on the
% latter.
%
% The \opt{bold-style} options' effects are shown in brief in
% \tabref{bold-style}.
%
% \begin{table}
%   \centering
%   \topcaption{Effects of the \opt{bold-style} package option.}
%   \tablabel{bold-style}
%   \begin{tabular}{@{}>{\ttfamily}lcc@{}}
%     \toprule
%       & \multicolumn{2}{c}{Example} \\
%        \cmidrule(l){2-3}
%       \rmfamily Package option & Latin & Greek \\
%     \midrule
%       bold-style=ISO & $(\mathbfit a, \mathbfit z, \mathbfit B, \mathbfit X)$ & $(\mathbfit\alpha, \mathbfit\beta, \mathbfit\Gamma, \mathbfit\Xi)$ \\
%       bold-style=TeX & $(\mathbfup a,\mathbfup z,\mathbfup B,\mathbfup X)$ & $(\mathbfit\alpha, \mathbfit\beta,\mathbfup \Gamma,\mathbfup \Xi)$ \\
%       bold-style=upright & $(\mathbfup a,\mathbfup z,\mathbfup B,\mathbfup X)$ & $(\mathbfup \alpha,\mathbfup \beta,\mathbfup \Gamma,\mathbfup \Xi)$ \\
%     \bottomrule
%   \end{tabular}
% \end{table}
%
%
% \subsection{Sans serif style}
% \seclabel{sans-style}
%
% Unicode contains upright and italic, medium and bold mathematical alphabet characters.
% These may be explicitly selected with the \cs{mathsfup}, \cs{mathsfit}, \cs{mathbfsfup}, and \cs{mathbfsfit}
% commands discussed in \secref{all-math-alphabets}.
%
% How should the generic \cs{mathsf} behave? Unlike bold, sans serif is used much more sparingly
% in mathematics. I've seen recommendations to typeset tensors in sans serif italic
% or sans serif italic bold (e.g., examples in the \pkg{isomath} and \pkg{mattens} packages).
% But \LaTeX's \cs{mathsf} is \textsl{upright} sans serif.
%
% Therefore I reluctantly add the package options |[sans-style=upright]| and |[sans-style=italic]| to control the behaviour of \cs{mathsf}.
% The |upright| style sets up the command to use upright sans serif, including Greek;
% the |italic| style switches to using italic in both Latin and Greek alphabets.
% In other words, this option simply changes the meaning of \cs{mathsf} to either \cs{mathsfup} or \cs{mathsfit}, respectively.
% Please let me know if more granular control is necessary here.
%
% There is also a |[sans-style=literal]| setting, set automatically with |[math-style=literal]|, which retains the uprightness of the input characters used when selecting the sans serif output.
%
% \subsubsection{What about bold sans serif?}
%
% While you might want your bold upright and your sans serif italic, I don't believe you'd also want
% your bold sans serif upright (or all vice versa, if that's even conceivable). Therefore, bold sans
% serif follows from the setting for sans serif; it is completely independent of the setting for bold.
%
% In other words, \cs{mathbfsf} is either \cs{mathbfsfup} or \cs{mathbfsfit} based on |[sans-style=upright]| or |[sans-style=italic]|, respectively. And |[sans-style=literal]| causes \cs{mathbfsf} to retain the same italic or upright shape as the input, and turns it bold sans serif.
%
% Note well! There is no medium-weight sans serif Greek alphabet in Unicode; therefore, |\mathsf{\alpha}| does not make sense (simply produces `$\mathsf{\alpha}$') while |\mathbfsf{\alpha}| gives `$\mathsf{\alpha}$'.
%
% \subsection{All (the rest) of the mathematical alphabets}
% \seclabel{all-math-alphabets}
%
% Unicode contains separate codepoints for most if not all variations of alphabet
% shape one may wish to use in mathematical notation. The complete list is shown
% in \tabref{mathalphabets}. Some of these have been covered in the previous sections.
%
% The math font switching commands do not nest; therefore if you want
% sans serif bold, you must write |\mathsfbf{...}| rather than |\mathbf{\mathsf{...}}|.
% This may change in the future.
%
% \begin{table}
% \caption{Mathematical alphabets defined in Unicode. Black dots indicate an alphabet exists in the font specified; blue dots indicate shapes that should always be taken from the upright font even in the italic style. See main text for description of \cs{mathbbit}.}
% \tablabel{mathalphabets}
% \centering
% \def\Y{\textbullet}
% \def\M{\textcolor{blue}{\textbullet}}
% \begin{tabular}{@{} lll l ccc @{}}
% \toprule
% \multicolumn{3}{c}{Font} & & \multicolumn{3}{c}{Alphabet} \\
% \cmidrule(r){1-3}
% \cmidrule(l){5-7}
% Style & Shape & Series & Switch & Latin & Greek & Numerals \\
% \midrule
% Serif      & Upright & Normal & \cs{mathup}     & \Y & \Y & \Y  \\
%            &         & Bold   & \cs{mathbfup}   & \Y & \Y & \Y  \\
%            & Italic  & Normal & \cs{mathit}     & \Y & \Y & \M  \\
%            &         & Bold   & \cs{mathbfit}   & \Y & \Y & \M  \\
% Sans serif & Upright & Normal & \cs{mathsfup}   & \Y &    & \Y  \\
%            & Italic  & Normal & \cs{mathsfit}   & \Y &    & \M  \\
%            & Upright & Bold   & \cs{mathbfsfup} & \Y & \Y & \Y  \\
%            & Italic  & Bold   & \cs{mathbfsfit} & \Y & \Y & \M  \\
% Typewriter & Upright & Normal & \cs{mathtt}     & \Y &    & \Y  \\
% Double-struck & Upright & Normal & \cs{mathbb}     & \Y &    & \Y  \\
%               & Italic  & Normal & \cs{mathbbit}   & \Y &    &  \\
% Script     & Upright & Normal & \cs{mathscr}    & \Y &    &     \\
%            &         & Bold   & \cs{matbfscr}   & \Y &    &     \\
% Fraktur    & Upright & Normal & \cs{mathfrak}   & \Y &    &     \\
%            &         & Bold   & \cs{mathbffrac} & \Y &    &     \\
% \bottomrule
% \end{tabular}
% \end{table}
%
% \subsubsection{Double-struck}
%
% The double-struck alphabet (also known as `blackboard bold') consists of
% upright Latin letters $\{\mathbb{a}$--$\mathbb{z}$,$\mathbb{A}$$\mathbb{Z}\}$,
% numerals $\mathbb{0}$--$\mathbb{9}$, summation symbol $\mathbb\sum$, and four
% Greek letters only: $\{\mathbb{\gamma\pi\Gamma\Pi}\}$.
%
% While |\mathbb{\sum}| does produce a double-struck summation symbol,
% its limits aren't properly aligned. Therefore,
% either the literal character or the control sequence \cs{Bbbsum} are
% recommended instead.
%
% There are also five Latin \emph{italic} double-struck letters: $\mathbbit{Ddeij}$.
% These can be accessed (if not with their literal characters or control sequences)
% with the \cs{mathbbit} alphabet switch, but note that only those five letters
% will give the expected output.
%
% \subsubsection{Caligraphic vs.\ Script variants}
%
% The Unicode maths encoding contains an alphabet style for `Script' letters,
% and while by default \cs{mathcal} and \cs{mathscr}
% are synonyms, there are some situations when a
% separate `Caligraphic' style is needed as well.
%
% If a font contains alternate glyphs for a separat caligraphic style,
% they can be selected explicitly as shown below.
% This feature is currently only supported by the XITS~Math font, where
% the caligraphic letters are accessed with the same glyph slots as the
% script letters but with the first stylistic set feature (|ss01|) applied.
% \begin{verbatim}
%   \setmathfont[range={\mathcal,\mathbfcal},StylisticSet=1]{XITS Math}
% \end{verbatim}
% An example is shown below.
% \begin{quote}
% \setmathfont[range=\mathscr]{XITS Math}
% \setmathfont[range=\mathcal,StylisticSet=1]{XITS Math}
% The Script style (\cs{mathscr}) in XITS Math is: $\mathscr{ABCXYZ}$\par
% The Caligraphic style (\cs{mathcal}) in XITS Math is: $\mathcal{ABCXYZ}$
% \end{quote}
%
%
% \subsection{Miscellanea}
%
% \subsubsection{Nabla}
% \seclabel{nabla}
%
%  The symbol $\nabla$ comes in the six forms shown in \tabref{nabla}.
%  We want an individual option to specify whether we want upright or italic
%  nabla by default (when either upright or italic nabla is used in the
%  source). \TeX\ classically uses an upright nabla, and \textsc{iso}
%  standards agree with this convention.
%  The package options |nabla=upright| and
%  |nabla=italic| switch between the two choices, and |nabla=literal| respects
%  the shape of the input character. This is then inherited
%  through \cmd\mathbf; \cmd\mathit\ and \cmd\mathup\ can be used to force one
%  way or the other.
%
% |nabla=italic| is the default. |nabla=literal| is
% activated automatically after |math-style=literal|.
%
% \begin{table}
%   \centering
%   \topcaption{The various forms of nabla.}
%   \tablabel{nabla}
%   \let \tmpshow\empty
%   \begin{tabular}{@{}llc@{}}
%       \toprule
%     \multicolumn{2}{@{}l}{Description} & Glyph
%      \\ \cmidrule(r){1-2}\cmidrule(l){3-3}
%     Upright & Serif & $\mathup\nabla$ \\
%     & Bold serif & $\mathbfup\nabla$ \\
%     & Bold sans & $\mathbfsfup\nabla$ \\
%       \cmidrule(lr){1-2}\cmidrule(lr){3-3}
%     Italic & Serif & $\mathit\nabla$ \\
%     & Bold serif & $\mathbfit\nabla$ \\
%     & Bold sans  & $\mathbfsfit\nabla$ \\
%       \bottomrule
%   \end{tabular}
% \end{table}
%
% \subsubsection{Partial}
% \seclabel{partial}
%
% The same applies to the symbols \unichar{2202} partial differential and
% \unichar{1D715} math italic partial differential.
%
% At time of writing, both the Cambria Math and STIX fonts display these
% two glyphs in the same italic style, but this is hopefully a bug that will
% be corrected in the future~--- the `plain' partial differential should
% really have an upright shape.
%
% Use the |partial=upright| or |partial=italic| package options to specify
% which one you would like, or |partial=literal| to have the same character
% used in the output as was used for the input.
% The default is (always, unless someone requests and
% argues otherwise) |partial=italic|.\footnote{A good argument would revolve
% around some international standards body recommending upright over italic.
% I just don't have the time right now to look it up.} |partial=literal|
% is activated following |math-style=literal|.
%
% See \tabref{partial} for the variations on the partial differential symbol.
%
% \begin{table}
%   \centering
%   \topcaption{The various forms of the partial differential. Note that in
% the fonts used to display these glyphs, the first upright partial is
% incorrectly shown in an italic style.}
%   \tablabel{partial}
%   \begin{tabular}{@{}llc@{}}
%       \toprule
%     \multicolumn{2}{@{}l}{Description} & Glyph
%      \\ \cmidrule(r){1-2}\cmidrule(l){3-3}
%     Regular   & Upright & $\mathup\partial$ \\
%               & Italic  & $\mathit\partial$ \\
%     Bold      & Upright & $\mathbfup\partial$ \\
%               & Italic  & $\mathbfit\partial$ \\
%     Sans bold & Upright & $\mathbfsfup\partial$ \\
%               & Italic  & $\mathbfsfit\partial$ \\
%       \bottomrule
%   \end{tabular}
% \end{table}
%
% \subsubsection{Epsilon and phi: $\epsilon$ vs.\ $\varepsilon$ and $\phi$ vs.\ $\varphi$}
% \seclabel{vargreek-shape}
%
% \TeX\ defines \cs{epsilon} to look like $\epsilon$ and \cs{varepsilon} to
% look like $\varepsilon$. By constrast, the Unicode glyph directly after delta and before zeta
% is `epsilon' and looks like $\varepsilon$; there is a subsequent variant of
% epsilon that looks like $\epsilon$. This creates a problem. People who
% use Unicode input won't want their glyphs transforming; \TeX\ users will be
% confused that what they think as `normal epsilon' is actual the `variant
% epsilon'. And the same problem exists for `phi'.
%
% We have an option to control this behaviour.
% With |vargreek-shape=TeX|,
% \cs{phi} and \cs{epsilon} produce $\phi$ and $\epsilon$ and
% \cs{varphi} and \cs{varepsilon} produce $\varphi$ and $\varepsilon$.
% With |vargreek-shape=unicode|, these symbols are swapped.
% Note, however, that Unicode characters are not affected by this option.
% That is, no remapping occurs of the characters/glyphs, only the control sequences.
%
% The package default is to use |vargreek-shape=TeX|.
%
% \subsubsection{Primes}
%
% Primes ($x'$) may be input in several ways. You may use any combination
% the \ascii\ straight quote (\texttt{\char`\'}) or the Unicode prime \unichar{2032}
% ($'$); when multiple primes occur next to each other, they chain
% together to form double, triple, or quadruple primes if the font contains
% pre-drawn glyphs. The individual prime glyphs are accessed, as usual,
% with the \cs{prime} command, and the double-, triple-, and quadruple-prime
% glyphs are available with \cs{dprime}, \cs{trprime}, and \cs{qprime},
% respectively.
%
% If the font does not contain the pre-drawn glyphs or more than four primes
% are used, the single prime glyph is used multiple times with a negative
% kern to get the spacing right. There is no user interface to adjust this
% negative kern yet (because I haven't decided what it should look like);
% if you need to, write something like this:
% \begin{verbatim}
% \ExplSyntaxOn
% \muskip_gset:Nn \g_um_primekern_muskip { -\thinmuskip/2 }
% \ExplySyntaxOff
% \end{verbatim}
%
% Backwards or reverse primes behave in exactly the same way; use the \ascii\
% back tick (\texttt{\char`\`}) or the Unicode reverse prime \unichar{2035}
% ({\umfont\char"2035}).
% The command to access the backprime is \cs{backprime}, and
% multiple backwards primes can accessed with \cs{backdprime},
% \cs{backtrprime}, and \cs{backqprime}.
%
% In all cases above, no error checking is performed if you attempt to
% access a multi-prime glyph in a font that doesn't contain one. For this
% reason, it may be safer to write \verb|x''''| instead of \verb|x\qprime|
% in general.
%
% If you ever need to enter the straight quote |'| or the backtick |`| in
% maths mode, these glyphs can be accessed with \cs{mathstraightquote} and
% \cs{mathbacktick}.
%
% \subsubsection{Unicode subscripts and superscripts}
%
% You may, if you wish, use Unicode subscripts and superscripts in your
% source document. For basic expressions, the use of these characters
% can make the input more readable.
% Adjacent sub- or super-scripts will be concatenated into a single
% expression.
%
% The range of subscripts and superscripts supported by this package
% are shown in \figref{superscripts,subscripts}. Please request more if
% you think it is appropriate.
%
% \begin{figure}\centering
% \fbox{\fontspec{Charis SIL}\Large
% A
% ^^^^2070 ^^^^00b9 ^^^^00b2 ^^^^00b3 ^^^^2074 ^^^^2075 ^^^^2076 ^^^^2077
% ^^^^2078 ^^^^2079 ^^^^207a ^^^^207b ^^^^207c ^^^^207d ^^^^207e ^^^^2071
% ^^^^207f
% Z}
% \caption{
%   The Unicode superscripts supported as input characters.
%   These are the literal glyphs from Charis SIL,
%   not the output seen when used for maths input.
%   The `A' and `Z' are to provide context for the size and
%   location of the superscript glyphs.
% }
% \figlabel{superscripts}
% \end{figure}
%
% \begin{figure}\centering
% \fbox{\fontspec{Charis SIL}\Large
% A
% ^^^^2080 ^^^^2081 ^^^^2082 ^^^^2083 ^^^^2084 ^^^^2085 ^^^^2086 ^^^^2087
% ^^^^2088 ^^^^2089 ^^^^208a ^^^^208b ^^^^208c ^^^^208d ^^^^208e ^^^^2090
% ^^^^2091 ^^^^1d62 ^^^^2092 ^^^^1d63 ^^^^1d64 ^^^^1d65 ^^^^2093 ^^^^1d66
% ^^^^1d67 ^^^^1d68 ^^^^1d69 ^^^^1d6a
% Z}
% \caption{
%   The Unicode subscripts supported as input characters.
%   See note from \figref{superscripts}.
% }
% \figlabel{subscripts}
% \end{figure}
%
% \subsubsection{Colon}
% \seclabel{colon}
%
% The colon is one of the few confusing characters of Unicode maths.
% In \TeX, \texttt{:} is defined as a colon with relation spacing: `$a:b$'.
% While \cs{colon} is defined as a colon with punctuation spacing: `$a\colon b$'.
%
% In Unicode, \unichar{003A} {colon} is defined as a punctuation symbol,
% while \unichar{2236} {ratio} is the colon-like symbol used in mathematics to denote
% ratios and other things.
%
% This breaks the usual straightforward mapping from control sequence to Unicode input character
% to (the same) Unicode glyph.
%
% To preserve input compatibility, we remap the \ascii\ input character `\texttt{:}' to \unichar{2236}.
% Typing a literal \unichar{2236} char will result in the same output.
% If \pkg{amsmath} is loaded, then the definition of \cs{colon} is inherited from there
% (it looks like a punctuation colon with additional space around it).
% Otherwise, \cs{colon} is made to output a colon with \cs{mathpunct} spacing.
%
% The package option |colon=literal| forces \ascii\ input `|:|' to be printed as \cs{mathcolon} instead.
%
%
% \subsubsection{Slashes and backslashes}
% \seclabel{slash-delimiter}
%
% There are several slash-like symbols defined in Unicode. The complete list is shown in \tabref{slashes}.
%
% \begin{table}\centering
% \caption{Slashes and backslashes.}
% \tablabel{slashes}
% \begin{tabular}{@{}cl@{}cl@{}}
% \toprule
% Slot & Name & Glyph & Command  \\
% \midrule
% \unichar{002F} & \textsc{solidus}                 & \umfont \char"002F & \cs{slash} \\
% \unichar{2044} & \textsc{fraction slash}          & \umfont \char"2044 & \cs{fracslash} \\
% \unichar{2215} & \textsc{division slash}          & \umfont \char"2215 & \cs{divslash} \\
% \unichar{29F8} & \textsc{big solidus}             & \umfont \char"29F8 & \cs{xsol} \\
% \midrule
% \unichar{005C} & \textsc{reverse solidus}         & \umfont \char"005C & \cs{backslash} \\
% \unichar{2216} & \textsc{set minus}               & \umfont \char"2216 & \cs{smallsetminus} \\
% \unichar{29F5} & \textsc{reverse solidus operator}& \umfont \char"29F5 & \cs{setminus} \\
% \unichar{29F9} & \textsc{big reverse solidus}     & \umfont \char"29F9 & \cs{xbsol} \\
% \bottomrule
% \end{tabular}
% \end{table}
%
% In regular \LaTeX\ we can write \cs{left}\cs{slash}\dots\cs{right}\cs{backslash}
% and so on and obtain extensible delimiter-like symbols. Not all of the Unicode slashes
% are suitable for this (and do not have the font support to do it).
%
% \paragraph{Slash}
%
% Of \unichar{2044} {fraction slash}, TR25 says that it is:
% \begin{quote}
% \dots used to build up simple fractions in running text\dots
% however parsers of mathematical texts should be prepared to handle fraction slash
% when it is received from other sources.
% \end{quote}
%
% \unichar{2215} {division slash} should be used when division is represented
% without a built-up fraction; $\pi\approx22/7$, for example.
%
% \unichar{29F8} {big solidus} is a `big operator' (like $\sum$).
%
% \paragraph{Backslash}
%
% The \unichar{005C} {reverse solidus} character \cs{backslash} is used for denoting
% double cosets: $A\backslash B$. (So I'm led to believe.)
% It may be used as a `stretchy' delimiter if supported by the font.
%
% MathML uses \unichar{2216} {set minus} like this: $A\smallsetminus B$.\footnote{\S4.4.5.11 \url{http://www.w3.org/TR/MathML3/}}
% The \LaTeX\ command name \cs{smallsetminus} is used for backwards compatibility.
%
% Presumably, \unichar{29F5} {reverse solidus operator} is intended to
% be used in a similar way, but it could also (perhaps?) be used to
% represent `inverse division': $\pi\approx7\mathbin{\backslash}22$.^^A
% \footnote{This is valid syntax in the Octave and Matlab programming languages,
% in which it means matrix inverse pre-multiplication. I.e., $A\mathbin{\backslash} B\equiv A^{-1}B$.}
% The \LaTeX\ name for this character is \cs{setminus}.
%
% Finally, \unichar{29F9} {big reverse solidus} is a `big operator' (like $\sum$).
%
% \paragraph{How to use all of these things}
%
% Unfortunately, font support for the above characters/glyphs is rather inconsistent.
% In Cambria Math, the only slash that grows (say when writing
% \[
% \left.\left[\begin{array}{cc} a & b \\ c & d\end{array}\right]\middle\slash
%       \left[\begin{array}{cc} 1 & 1 \\ 1 & 0\end{array}\right] \right.\quad )
% \]
% is the \textsc{fraction slash}, which we just established above is
% sort of only supposed to be used in text.
%
% Of the above characters, the following are allowed to be used after
% \cs{left}, \cs{middle}, and \cs{right}:
% \begin{itemize}
% \item \cs{solidus};
% \item \cs{fracslash};
% \item \cs{slash}; and,
% \item \cs{backslash} (the only reverse slash).
% \end{itemize}
%
% However, we assume that there is only \emph{one} stretchy slash
% in the font; this is assumed by default to be \unichar{002F} {solidus}.
% Writing \cs{left/} or \cs{left}\cs{slash} or \cs{left}{fracslash}
% will all result in the same stretchy delimiter being used.
%
% The delimiter used can be changed with the |slash-delimiter| package option.
% Allowed values are |ascii|, |frac|, and |div|, corresponding to the respective
% Unicode slots.
%
% For example: as mentioned above, Cambria Math's stretchy slash is
% \unichar{2044} {fraction slash}. When using Cambria Math, then
% \pkg{unicode-math} should be loaded with the |slash-delimiter=frac| option.
% (This should be a font option rather than a package option, but
% it will change soon.)
%
%
% \subsubsection{Growing and non-growing accents}
% \seclabel{growing-accents}
%
% There are a few accents for which \TeX\ has both non-growing and growing
% versions.  Among these are \cs{hat} and \cs{tilde}; the corresponding growing
% versions are called \cs{widehat} and \cs{widetilde}, respectively.
%
% Older versions of \XeTeX\ and \LuaTeX\ did not support this distinction,
% however, and \emph{all} accents there were growing automatically. (I.e.,
% \cs{hat} and \cs{widehat} are equivalent.) As of \LuaTeX\ v0.65 and \XeTeX\
% v0.9998, these wide/non-wide commands will again behave in their expected
% manner.
%
%
% \subsubsection{Pre-drawn fraction characters}
%
% Pre-drawn fractions \unichar{00BC}--\unichar{00BE}, \unichar{2150}--\unichar{215E}
% are not suitable for use in mathematics output. However, they can be useful
% as input characters to abbreviate common fractions.
% \begin{center}
% \fontspec{DejaVuSerif.ttf} ^^A available in TeX Live 2012 if not earlier
% ¼ ½ ¾  ↉ ⅐ ⅑ ⅒ ⅓ ⅔ ⅕ ⅖ ⅗ ⅘ ⅙ ⅚ ⅛ ⅜ ⅝ ⅞
% \end{center}
% For example, instead of writing `|\tfrac12 x|', you may consider it more readable to have
% `|½x|' in the source instead.
%
% If the \cs{tfrac} command exists (i.e., if \pkg{amsmath} is loaded or
% you have specially defined \cs{tfrac} for this purpose), it will be used
% to typeset the fractions. If not, regular \cs{frac} will be used. The command
% to use (\cs{tfrac} or \cs{frac}) can be forced either way with the package
% option |active-frac=small| or |active-frac=normalsize|, respectively.
%
% \subsubsection{Circles}
%
% Unicode defines a large number of different types of circles for a variety
% of mathematical purposes. There are thirteen alone just considering the
% all white and all black ones, shown in \tabref{circles}.
%
% \LaTeX\ defines considerably fewer: \cs{circ} and cs{bigcirc} for white;
% \cs{bullet} for black. This package maps those commands to \cs{vysmwhtcircle},
% \cs{mdlgwhtcircle}, and \cs{smblkcircle}, respectively.
%
% \begin{table}
% \def\showchar#1#2#3{ \textsc{u}+{\small\ttfamily #1} & \texttt{\string#3} & \umfont \char"#1 \\}
% \begin{tabular}{@{}llc@{}}
% \toprule
% Slot & Command & Glyph \\
% \midrule
% \showchar{00B7}{centerdot}{\cdotp}
% \showchar{22C5}{small middle dot}{\cdot}
% \showchar{2219}{bullet operator}{\vysmblkcircle}
% \showchar{2022}{round bullet, filled}{\smblkcircle}
% \showchar{2981}{z notation spot}{\mdsmblkcircle}
% \showchar{26AB}{medium black circle}{\mdblkcircle}
% \showchar{25CF}{circle, filled}{\mdlgblkcircle}
% \showchar{2B24}{black large circle}{\lgblkcircle}
% \bottomrule
% \end{tabular}
% \def\showchar#1#2#3{ \umfont \char"#1 & \texttt{\string#3} & \textsc{u}+{\small\ttfamily #1} \\}
% \begin{tabular}{@{}cll@{}}
% \toprule
% Glyph & Command & Slot \\
% \midrule
% \\
% \\
% \showchar{2218}{composite function (small circle)}{\vysmwhtcircle}
% \showchar{25E6}{white bullet}{\smwhtcircle}
% \showchar{26AC}{medium small white circle}{\mdsmwhtcircle}
% \showchar{26AA}{medium white circle}{\mdwhtcircle}
% \showchar{25CB}{large circle}{\mdlgwhtcircle}
% \showchar{25EF}{large circle}{\lgwhtcircle}
% \bottomrule
% \end{tabular}
% \caption{Filled and hollow Unicode circles.}
% \tablabel{circles}
% \end{table}
%
% \subsubsection{Triangles}
%
% While there aren't as many different sizes of triangle as there are circle,
% there's some important distinctions to make between a few similar characters. See \tabref{uptriangles} for the full summary.
%
% These triangles all have different intended meanings. Note for backwards
% compatibility with \TeX, \unichar{25B3} has \emph{two} different mappings
% in \pkg{unicode-math}. \cs{bigtriangleup} is intended as a binary operator
% whereas \cs{triangle} is intended to be used as a letter-like symbol.
%
% But you're better off if you're using the latter form to indicate an
% increment to use the glyph intended for this purpose, \unichar{2206}: $\increment x$.
%
% Finally, given that $\triangle$ and $\increment$ are provided for you
% already, it is better off to only use upright Greek Delta $\Delta$ if you're
% actually using it as a symbolic entity such as a variable on its own.
%
% \begin{table}
% \begin{tabular}{@{}llcl@{}}
% \toprule
% Slot & Command & Glyph & Class \\
% \midrule
% \unichar{25B5} & \cs{vartriangle}      & \umfont \char"25B5 & binary \\
% \unichar{25B3} & \cs{bigtriangleup}    & \umfont \char"25B3 & binary \\
% \unichar{25B3} & \cs{triangle}         & \umfont \char"25B3 & ordinary \\
% \unichar{2206} & \cs{increment}        & \umfont \char"2206 & ordinary \\
% \unichar{0394} & \cs{mathup}\cs{Delta} & \umfont \char"0394 & ordinary \\
% \bottomrule
% \end{tabular}
% \caption{Different upwards pointing triangles.}
% \tablabel{uptriangles}
% \end{table}
%
% \iffalse
% \subsubsection{Normalising some input characters}
%
% I believe
% all variant forms should be used as legal input that is normalised to
% a consistent output glyph, because we want to be fault-tolerant in the input.
% Here are the duplicates:
% \begin{quote}\obeylines
% \unichar {251} {latin small letter alpha}
% \unichar {25B} {latin small letter epsilon}
% \unichar {263} {latin small letter gamma}
% \unichar {269} {latin small letter iota}
% \unichar {278} {latin small letter phi}
% \unichar {28A} {latin small letter upsilon}
% \unichar {190} {latin capital letter epsilon}
% \unichar {194} {latin capital letter gamma}
% \unichar {196} {latin capital letter iota}
% \unichar {1B1} {latin capital letter upsilon}
% \end{quote}
%
% (Not yet implemented.)
% \fi
%
% \section{Advanced}
%
% \subsection{Warning messages}
%
% This package can produce a number of informational messages to try and inform the user when something might be going wrong due to package conflicts or something else.
% As an experimental feature, these can be turn off on an individual basis with the package option |warnings-off| which takes a comma-separated list of warnings to suppress.
% A warning will give you its name when printed on the console output; e.g.,
% \begin{Verbatim}[gobble=2]
%   * unicode-math warning: "mathtools-colon"
%   *
%   * ... <warning message> ...
% \end{Verbatim}
% This warning could be suppressed by loading the package as follows:
% \begin{Verbatim}[gobble=2]
%   \usepackage[warnings-off={mathtools-colon}]{unicode-math}
% \end{Verbatim}
%
% \subsection{Programmer's interface}
%
% (Tentative and under construction.)
% If you are writing some code that needs to know the current
% maths style (\cs{mathbf}, \cs{mathit}, etc.), you can query the
% variable \cs{l_um_mathstyle_tl}. It will contain the maths style
% without the leading `math' string; for example,
% |\mathbf { \show \l_um_mathstyle_tl }|
% will produce `bf'.
%
% \StopEventually{}
%
% \clearpage
% \part{Package implementation}
% \parttoc
%
% \section{Header code}
%
% We (later on) bifurcate the package based on the engine being used.
%    \begin{macrocode}
%<*load>
\luatex_if_engine:T { \RequirePackage{unicode-math-luatex} \endinput }
\xetex_if_engine:T  { \RequirePackage{unicode-math-xetex}  \endinput }
%</load>
%    \end{macrocode}
% The shared part of the code starts here before the split above.
%    \begin{macrocode}
%<*preamble&!XE&!LU>
%    \end{macrocode}
%
%    \begin{macrocode}
\usepackage{ifxetex,ifluatex}
\ifxetex
  \ifdim\number\XeTeXversion\XeTeXrevision in<0.9998in%
    \PackageError{unicode-math}{%
      Cannot run with this version of XeTeX!\MessageBreak
      You need XeTeX 0.9998 or newer.%
    }\@ehd
  \fi
\else\ifluatex
  \ifnum\luatexversion<64%
    \PackageError{unicode-math}{%
      Cannot run with this version of LuaTeX!\MessageBreak
      You need LuaTeX 0.64 or newer.%
    }\@ehd
  \fi
\else
  \PackageError{unicode-math}{%
    Cannot be run with pdfLaTeX!\MessageBreak
    Use XeLaTeX or LuaLaTeX instead.%
  }\@ehd
\fi\fi
%    \end{macrocode}
%
% \paragraph{Packages}
%    \begin{macrocode}
\RequirePackage{expl3}[2011/07/01]
\RequirePackage{xparse}[2009/08/31]
\RequirePackage{l3keys2e}
\RequirePackage{fontspec}[2010/10/25]
\RequirePackage{catchfile}
\RequirePackage{fix-cm} % avoid some warnings
\RequirePackage{filehook}[2011/01/03]
%    \end{macrocode}
% Start using \LaTeX3 --- finally!
%    \begin{macrocode}
\ExplSyntaxOn
%    \end{macrocode}
%
% \paragraph{Extra \pkg{expl3} variants}
%    \begin{macrocode}
\cs_generate_variant:Nn \tl_put_right:Nn {cx}
\cs_generate_variant:Nn \seq_if_in:NnTF {NV}
\cs_generate_variant:Nn \prop_gput:Nnn {Nxn}
\cs_generate_variant:Nn \prop_get:NnN {cxN}
\cs_generate_variant:Nn \prop_if_in:NnTF {cx}
%    \end{macrocode}
% Extra expansion command:
%    \begin{macrocode}
\cs_set:Npn \exp_args:NNcc #1#2#3#4 {
  \exp_after:wN #1 \exp_after:wN #2
    \cs:w #3 \exp_after:wN \cs_end:
    \cs:w #4 \cs_end:
}
%    \end{macrocode}
%
%
% \paragraph{Conditionals}
%
%    \begin{macrocode}
\bool_new:N \l_um_ot_math_bool
\bool_new:N \l_um_init_bool
\bool_new:N \l_um_implicit_alph_bool
\bool_new:N \g_um_mainfont_already_set_bool
%    \end{macrocode}
% For \opt{math-style}:
%    \begin{macrocode}
\bool_new:N \g_um_literal_bool
\bool_new:N \g_um_upLatin_bool
\bool_new:N \g_um_uplatin_bool
\bool_new:N \g_um_upGreek_bool
\bool_new:N \g_um_upgreek_bool
%    \end{macrocode}
% For \opt{bold-style}:
%    \begin{macrocode}
\bool_new:N \g_um_bfliteral_bool
\bool_new:N \g_um_bfupLatin_bool
\bool_new:N \g_um_bfuplatin_bool
\bool_new:N \g_um_bfupGreek_bool
\bool_new:N \g_um_bfupgreek_bool
%    \end{macrocode}
% For \opt{sans-style}:
%    \begin{macrocode}
\bool_new:N \g_um_upsans_bool
\bool_new:N \g_um_sfliteral_bool
%    \end{macrocode}
% For assorted package options:
%    \begin{macrocode}
\bool_new:N \g_um_upNabla_bool
\bool_new:N \g_um_uppartial_bool
\bool_new:N \g_um_literal_Nabla_bool
\bool_new:N \g_um_literal_partial_bool
\bool_new:N \g_um_texgreek_bool
\bool_set_true:N \g_um_texgreek_bool
\bool_new:N \l_um_smallfrac_bool
\bool_new:N \g_um_literal_colon_bool
%    \end{macrocode}
%
% \paragraph{Variables}
%    \begin{macrocode}
\int_new:N \g_um_fam_int
%    \end{macrocode}
%
%    \begin{macrocode}
\tl_const:Nn \c_um_math_alphabet_name_latin_tl {Latin,~lowercase}
\tl_const:Nn \c_um_math_alphabet_name_Latin_tl {Latin,~uppercase}
\tl_const:Nn \c_um_math_alphabet_name_greek_tl {Greek,~lowercase}
\tl_const:Nn \c_um_math_alphabet_name_Greek_tl {Greek,~uppercase}
\tl_const:Nn \c_um_math_alphabet_name_num_tl   {Numerals}
\tl_const:Nn \c_um_math_alphabet_name_misc_tl  {Misc.}
%    \end{macrocode}
%
% \subsection{Extras}
%
% \begin{macro}{\um_glyph_if_exist:nTF}
%: TODO: Generalise for arbitrary fonts! \cs{l_um_font} is not always the one used for a specific glyph!!
%    \begin{macrocode}
\prg_new_conditional:Nnn \um_glyph_if_exist:n {p,TF,T,F} {
  \iffontchar \l_um_font #1 \scan_stop:
    \prg_return_true:
  \else:
    \prg_return_false:
  \fi:
}
\cs_generate_variant:Nn \um_glyph_if_exist_p:n {c}
\cs_generate_variant:Nn \um_glyph_if_exist:nTF {c}
\cs_generate_variant:Nn \um_glyph_if_exist:nT  {c}
\cs_generate_variant:Nn \um_glyph_if_exist:nF  {c}
%    \end{macrocode}
% \end{macro}
%
% \subsection{Function variants}
%
%    \begin{macrocode}
\cs_generate_variant:Nn \fontspec_set_family:Nnn {Nx}
\cs_generate_variant:Nn \fontspec_set_fontface:NNnn {NNx}
%    \end{macrocode}
%
% \subsection{Package options}
%
% \begin{macro}{\unimathsetup}
% This macro can be used in lieu of or later to override
% options declared when the package is loaded.
%    \begin{macrocode}
\DeclareDocumentCommand \unimathsetup {m} {
  \keys_set:nn  {unicode-math} {#1}
}
%    \end{macrocode}
% \end{macro}
%
%
% \paragraph{math-style}
%    \begin{macrocode}
\keys_define:nn {unicode-math} {
  normal-style .choice_code:n =
  {
    \bool_set_false:N \g_um_literal_bool
    \ifcase \l_keys_choice_int
      \bool_set_false:N \g_um_upGreek_bool
      \bool_set_false:N \g_um_upgreek_bool
      \bool_set_false:N \g_um_upLatin_bool
      \bool_set_false:N \g_um_uplatin_bool
    \or
      \bool_set_true:N \g_um_upGreek_bool
      \bool_set_false:N \g_um_upgreek_bool
      \bool_set_false:N \g_um_upLatin_bool
      \bool_set_false:N \g_um_uplatin_bool
    \or
      \bool_set_true:N \g_um_upGreek_bool
      \bool_set_true:N \g_um_upgreek_bool
      \bool_set_true:N \g_um_upLatin_bool
      \bool_set_false:N \g_um_uplatin_bool
    \or
      \bool_set_true:N \g_um_upGreek_bool
      \bool_set_true:N \g_um_upgreek_bool
      \bool_set_true:N \g_um_upLatin_bool
      \bool_set_true:N \g_um_uplatin_bool
    \or
      \bool_set_true:N \g_um_literal_bool
    \fi
  } ,
  normal-style .generate_choices:n = {ISO,TeX,french,upright,literal} ,
}
%    \end{macrocode}
%
%    \begin{macrocode}
\keys_define:nn {unicode-math} {
  math-style .choice_code:n =
  {
    \ifcase \l_keys_choice_int
      \unimathsetup {
        normal-style=ISO,
        bold-style=ISO,
        sans-style=italic,
        nabla=upright,
        partial=italic,
      }
    \or
      \unimathsetup {
        normal-style=TeX,
        bold-style=TeX,
        sans-style=upright,
        nabla=upright,
        partial=italic,
      }
    \or
      \unimathsetup {
        normal-style=french,
        bold-style=upright,
        sans-style=upright,
        nabla=upright,
        partial=upright,
      }
    \or
      \unimathsetup {
        normal-style=upright,
        bold-style=upright,
        sans-style=upright,
        nabla=upright,
        partial=upright,
      }
    \or
      \unimathsetup {
        normal-style=literal,
        bold-style=literal,
        sans-style=literal,
        colon=literal,
        nabla=literal,
        partial=literal,
      }
    \fi
  } ,
  math-style .generate_choices:n = {ISO,TeX,french,upright,literal} ,
}
%    \end{macrocode}
%
% \paragraph{bold-style}
%    \begin{macrocode}
\keys_define:nn {unicode-math} {
  bold-style .choice_code:n = {
    \bool_set_false:N \g_um_bfliteral_bool
    \ifcase \l_keys_choice_int
      \bool_set_false:N \g_um_bfupGreek_bool
      \bool_set_false:N \g_um_bfupgreek_bool
      \bool_set_false:N \g_um_bfupLatin_bool
      \bool_set_false:N \g_um_bfuplatin_bool
    \or
      \bool_set_true:N \g_um_bfupGreek_bool
      \bool_set_false:N \g_um_bfupgreek_bool
      \bool_set_true:N \g_um_bfupLatin_bool
      \bool_set_true:N \g_um_bfuplatin_bool
    \or
      \bool_set_true:N \g_um_bfupGreek_bool
      \bool_set_true:N \g_um_bfupgreek_bool
      \bool_set_true:N \g_um_bfupLatin_bool
      \bool_set_true:N \g_um_bfuplatin_bool
    \or
      \bool_set_true:N \g_um_bfliteral_bool
    \fi
  } ,
  bold-style .generate_choices:n = {ISO,TeX,upright,literal} ,
}
%    \end{macrocode}
%
% \paragraph{sans-style}
%    \begin{macrocode}
\keys_define:nn {unicode-math} {
  sans-style .choice_code:n = {
    \ifcase \l_keys_choice_int
      \bool_set_false:N \g_um_upsans_bool
    \or
      \bool_set_true:N \g_um_upsans_bool
    \or
      \bool_set_true:N \g_um_sfliteral_bool
    \fi
  } ,
  sans-style .generate_choices:n = {italic,upright,literal} ,
}
%    \end{macrocode}
%
%
% \paragraph{Nabla and partial}
%    \begin{macrocode}
\keys_define:nn {unicode-math} {
  nabla .choice_code:n = {
    \bool_set_false:N \g_um_literal_Nabla_bool
    \ifcase \l_keys_choice_int
      \bool_set_true:N \g_um_upNabla_bool
    \or
      \bool_set_false:N \g_um_upNabla_bool
    \or
      \bool_set_true:N \g_um_literal_Nabla_bool
    \fi
  } ,
  nabla .generate_choices:n = {upright,italic,literal} ,
}
%    \end{macrocode}
%
%    \begin{macrocode}
\keys_define:nn {unicode-math} {
  partial .choice_code:n = {
    \bool_set_false:N \g_um_literal_partial_bool
    \ifcase \l_keys_choice_int
      \bool_set_true:N \g_um_uppartial_bool
    \or
      \bool_set_false:N \g_um_uppartial_bool
    \or
      \bool_set_true:N \g_um_literal_partial_bool
    \fi
  } ,
  partial .generate_choices:n = {upright,italic,literal} ,
}
%    \end{macrocode}
%
% \paragraph{Epsilon and phi shapes}
%    \begin{macrocode}
\keys_define:nn {unicode-math} {
  vargreek-shape .choice: ,
  vargreek-shape / unicode .code:n = {
    \bool_set_false:N \g_um_texgreek_bool
  } ,
  vargreek-shape / TeX .code:n = {
    \bool_set_true:N \g_um_texgreek_bool
  }
}
%    \end{macrocode}
%
% \paragraph{Colon style}
%    \begin{macrocode}
\keys_define:nn {unicode-math} {
  colon .choice: ,
  colon / literal .code:n = {
    \bool_set_true:N \g_um_literal_colon_bool
  } ,
  colon / TeX .code:n = {
    \bool_set_false:N \g_um_literal_colon_bool
  }
}
%    \end{macrocode}
%
% \paragraph{Slash delimiter style}
%    \begin{macrocode}
\keys_define:nn {unicode-math} {
  slash-delimiter .choice: ,
  slash-delimiter / ascii .code:n = {
    \tl_set:Nn \g_um_slash_delimiter_usv {"002F}
  } ,
  slash-delimiter / frac .code:n = {
    \tl_set:Nn \g_um_slash_delimiter_usv {"2044}
  } ,
  slash-delimiter / div .code:n = {
    \tl_set:Nn \g_um_slash_delimiter_usv {"2215}
  }
}
%    \end{macrocode}
%
%
% \paragraph{Active fraction style}
%    \begin{macrocode}
\keys_define:nn {unicode-math} {
  active-frac .choice: ,
  active-frac / small .code:n = {
    \cs_if_exist:NTF \tfrac {
      \bool_set_true:N \l_um_smallfrac_bool
    }{
      \um_warning:n {no-tfrac}
      \bool_set_false:N \l_um_smallfrac_bool
    }
    \use:c{um_setup_active_frac:}
  } ,
  active-frac / normalsize .code:n = {
    \bool_set_false:N \l_um_smallfrac_bool
    \use:c{um_setup_active_frac:}
  }
}
%    \end{macrocode}
%
% \paragraph{Debug/tracing}
%
%
%    \begin{macrocode}
\keys_define:nn {unicode-math}
  {
    warnings-off .code:n =
      {
        \clist_map_inline:nn {#1}
          { \msg_redirect_name:nnn { unicode-math } { ##1 } { none } }
      }
  }
%    \end{macrocode}
%
%    \begin{macrocode}
\keys_define:nn {unicode-math} {
  trace .choice: ,
  trace / debug .code:n = {
    \msg_redirect_module:nnn { unicode-math } { log } { warning }
  } ,
  trace / on .code:n = {
    % default
  } ,
  trace / off .code:n = {
    \msg_redirect_module:nnn { unicode-math } { log } { none }
  } ,
}
%    \end{macrocode}
%
%    \begin{macrocode}
\unimathsetup {math-style=TeX}
\unimathsetup {slash-delimiter=ascii}
\unimathsetup {trace=off}
\cs_if_exist:NT \tfrac {
  \unimathsetup {active-frac=small}
}
\ProcessKeysOptions {unicode-math}
%    \end{macrocode}
%
%
% \section{\Hologo{LuaLaTeX} module}
%
% We create a \pkg{luatexbase} module that contains Lua functions for use with \hologo{LuaLaTeX}.
%    \begin{macrocode}
%</preamble&!XE&!LU>
%<*lua>
local err, warn, info, log = luatexbase.provides_module({
  name        = "unicode-math",
  date        = "2012/04/23",
  version     = 0.1,
  description = "Unicode math typesetting for LuaLaTeX",
  author      = "Khaled Hosny, Will Robertson, Philipp Stephani",
  licence     = "LPPL v1.3+"
})
%    \end{macrocode}
% \LuaTeX\ does not provide interface to accessing
% \texttt{(Script)ScriptPercentScaleDown} math constants, so we
% emulate \XeTeX\ behaviour by setting \cs{fontdimen10} and
% \cs{fontdimen11}.
%    \begin{macrocode}
local function set_sscale_dimens(fontdata)
  local mc = fontdata.MathConstants
  if mc then
    fontdata.parameters[10] = mc.ScriptPercentScaleDown or 70
    fontdata.parameters[11] = mc.ScriptScriptPercentScaleDown or 50
  end
end
luatexbase.add_to_callback("luaotfload.patch_font", set_sscale_dimens, "unicode_math.set_sscale_dimens")
%    \end{macrocode}
%
% Cambria Math has too small \texttt{DisplayOperatorMinHeight} constant, so we
% patch it to amore accebtable value.
%
%    \begin{macrocode}
local function patch_cambria_domh(fontdata)
  local mc = fontdata.MathConstants
  local mh = 2800 / fontdata.units * fontdata.size
  if fontdata.psname == "CambriaMath" and mc then
    if mc.DisplayOperatorMinHeight < mh then
      mc.DisplayOperatorMinHeight = mh
    end
  end
end
luatexbase.add_to_callback("luaotfload.patch_font", patch_cambria_domh, "cambria.domh")
%</lua>
%    \end{macrocode}
%
% (Error messages and warning definitions go here from the |msg| chunk
%  defined in \secref[vref]{codemsg}.)
%
% \section{Bifurcation}
% And here the split begins. Most of the code is still shared, but
% code for \LuaTeX\ uses the `LU' prefix and code for \XeTeX\ uses `XE'.
%
%    \begin{macrocode}
%<*package&(XE|LU)>
\ExplSyntaxOn
%    \end{macrocode}
%
% \subsection{Engine differences}
%
%    \begin{macrocode}
\cs_new:Nn \um_cs_compat:n
%<XE>  { \cs_set_eq:cc {U#1} {XeTeX#1}   }
%<LU>  { \cs_set_eq:cc {U#1} {luatexU#1} }
\um_cs_compat:n {mathcode}
\um_cs_compat:n {delcode}
\um_cs_compat:n {mathcodenum}
\um_cs_compat:n {mathcharnum}
\um_cs_compat:n {mathchardef}
\um_cs_compat:n {radical}
\um_cs_compat:n {mathaccent}
\um_cs_compat:n {delimiter}
%    \end{macrocode}
%
%    \begin{macrocode}
%<*LU>
\RequirePackage { lualatex-math } [ 2011/08/07 ]
\RequirePackage { luatexbase }
\RequirePackage { luaotfload } [ 2010/11/26 ]
\RequireLuaModule { unicode-math } [ 2012/04/23 ]
%</LU>
%    \end{macrocode}
%
% \subsection{Alphabet Unicode positions}
%
% Before we begin, let's define the positions of the various Unicode
% alphabets so that our code is a little more readable.\footnote{`\textsc{u.s.v.}' stands
% for `Unicode scalar value'.}
%
% Rather than `readable', in the end, this makes the code more extensible.
%    \begin{macrocode}
\cs_new:Nn \usv_set:nnn {
  \tl_set:cn { \um_to_usv:nn {#1}{#2} } {#3}
}
\cs_new:Nn \um_to_usv:nn { g_um_#1_#2_usv }
%    \end{macrocode}
% \paragraph{Alphabets}
%    \begin{macrocode}
\usv_set:nnn {up}{num}{48}
\usv_set:nnn {up}{Latin}{65}
\usv_set:nnn {up}{latin}{97}
\usv_set:nnn {up}{Greek}{"391}
\usv_set:nnn {up}{greek}{"3B1}
\usv_set:nnn {it}{Latin}{"1D434}
\usv_set:nnn {it}{latin}{"1D44E}
\usv_set:nnn {it}{Greek}{"1D6E2}
\usv_set:nnn {it}{greek}{"1D6FC}
\usv_set:nnn {bb}{num}{"1D7D8}
\usv_set:nnn {bb}{Latin}{"1D538}
\usv_set:nnn {bb}{latin}{"1D552}
\usv_set:nnn {scr}{Latin}{"1D49C}
\usv_set:nnn {cal}{Latin}{"1D49C}
\usv_set:nnn {scr}{latin}{"1D4B6}
\usv_set:nnn {frak}{Latin}{"1D504}
\usv_set:nnn {frak}{latin}{"1D51E}
\usv_set:nnn {sf}{num}{"1D7E2}
\usv_set:nnn {sfup}{num}{"1D7E2}
\usv_set:nnn {sfit}{num}{"1D7E2}
\usv_set:nnn {sfup}{Latin}{"1D5A0}
\usv_set:nnn {sf}{Latin}{"1D5A0}
\usv_set:nnn {sfup}{latin}{"1D5BA}
\usv_set:nnn {sf}{latin}{"1D5BA}
\usv_set:nnn {sfit}{Latin}{"1D608}
\usv_set:nnn {sfit}{latin}{"1D622}
\usv_set:nnn {tt}{num}{"1D7F6}
\usv_set:nnn {tt}{Latin}{"1D670}
\usv_set:nnn {tt}{latin}{"1D68A}
%    \end{macrocode}
% Bold:
%    \begin{macrocode}
\usv_set:nnn {bf}{num}{"1D7CE}
\usv_set:nnn {bfup}{num}{"1D7CE}
\usv_set:nnn {bfit}{num}{"1D7CE}
\usv_set:nnn {bfup}{Latin}{"1D400}
\usv_set:nnn {bfup}{latin}{"1D41A}
\usv_set:nnn {bfup}{Greek}{"1D6A8}
\usv_set:nnn {bfup}{greek}{"1D6C2}
\usv_set:nnn {bfit}{Latin}{"1D468}
\usv_set:nnn {bfit}{latin}{"1D482}
\usv_set:nnn {bfit}{Greek}{"1D71C}
\usv_set:nnn {bfit}{greek}{"1D736}
\usv_set:nnn {bffrak}{Latin}{"1D56C}
\usv_set:nnn {bffrak}{latin}{"1D586}
\usv_set:nnn {bfscr}{Latin}{"1D4D0}
\usv_set:nnn {bfcal}{Latin}{"1D4D0}
\usv_set:nnn {bfscr}{latin}{"1D4EA}
\usv_set:nnn {bfsf}{num}{"1D7EC}
\usv_set:nnn {bfsfup}{num}{"1D7EC}
\usv_set:nnn {bfsfit}{num}{"1D7EC}
\usv_set:nnn {bfsfup}{Latin}{"1D5D4}
\usv_set:nnn {bfsfup}{latin}{"1D5EE}
\usv_set:nnn {bfsfup}{Greek}{"1D756}
\usv_set:nnn {bfsfup}{greek}{"1D770}
\usv_set:nnn {bfsfit}{Latin}{"1D63C}
\usv_set:nnn {bfsfit}{latin}{"1D656}
\usv_set:nnn {bfsfit}{Greek}{"1D790}
\usv_set:nnn {bfsfit}{greek}{"1D7AA}
%    \end{macrocode}
%
%    \begin{macrocode}
\usv_set:nnn {bfsf}{Latin}{ \bool_if:NTF \g_um_upLatin_bool \g_um_bfsfup_Latin_usv \g_um_bfsfit_Latin_usv }
\usv_set:nnn {bfsf}{latin}{ \bool_if:NTF \g_um_uplatin_bool \g_um_bfsfup_latin_usv \g_um_bfsfit_latin_usv }
\usv_set:nnn {bfsf}{Greek}{ \bool_if:NTF \g_um_upGreek_bool \g_um_bfsfup_Greek_usv \g_um_bfsfit_Greek_usv }
\usv_set:nnn {bfsf}{greek}{ \bool_if:NTF \g_um_upgreek_bool \g_um_bfsfup_greek_usv \g_um_bfsfit_greek_usv }
\usv_set:nnn {bf}{Latin}{ \bool_if:NTF \g_um_bfupLatin_bool \g_um_bfup_Latin_usv \g_um_bfit_Latin_usv }
\usv_set:nnn {bf}{latin}{ \bool_if:NTF \g_um_bfuplatin_bool \g_um_bfup_latin_usv \g_um_bfit_latin_usv }
\usv_set:nnn {bf}{Greek}{ \bool_if:NTF \g_um_bfupGreek_bool \g_um_bfup_Greek_usv \g_um_bfit_Greek_usv }
\usv_set:nnn {bf}{greek}{ \bool_if:NTF \g_um_bfupgreek_bool \g_um_bfup_greek_usv \g_um_bfit_greek_usv }
%    \end{macrocode}
% Greek variants:
%    \begin{macrocode}
\usv_set:nnn {up}{varTheta}{"3F4}
\usv_set:nnn {up}{Digamma}{"3DC}
\usv_set:nnn {up}{varepsilon}{"3F5}
\usv_set:nnn {up}{vartheta}{"3D1}
\usv_set:nnn {up}{varkappa}{"3F0}
\usv_set:nnn {up}{varphi}{"3D5}
\usv_set:nnn {up}{varrho}{"3F1}
\usv_set:nnn {up}{varpi}{"3D6}
\usv_set:nnn {up}{digamma}{"3DD}
%    \end{macrocode}
% Bold:
%    \begin{macrocode}
\usv_set:nnn {bfup}{varTheta}{"1D6B9}
\usv_set:nnn {bfup}{Digamma}{"1D7CA}
\usv_set:nnn {bfup}{varepsilon}{"1D6DC}
\usv_set:nnn {bfup}{vartheta}{"1D6DD}
\usv_set:nnn {bfup}{varkappa}{"1D6DE}
\usv_set:nnn {bfup}{varphi}{"1D6DF}
\usv_set:nnn {bfup}{varrho}{"1D6E0}
\usv_set:nnn {bfup}{varpi}{"1D6E1}
\usv_set:nnn {bfup}{digamma}{"1D7CB}
%    \end{macrocode}
% Italic Greek variants:
%    \begin{macrocode}
\usv_set:nnn {it}{varTheta}{"1D6F3}
\usv_set:nnn {it}{varepsilon}{"1D716}
\usv_set:nnn {it}{vartheta}{"1D717}
\usv_set:nnn {it}{varkappa}{"1D718}
\usv_set:nnn {it}{varphi}{"1D719}
\usv_set:nnn {it}{varrho}{"1D71A}
\usv_set:nnn {it}{varpi}{"1D71B}
%    \end{macrocode}
% Bold italic:
%    \begin{macrocode}
\usv_set:nnn {bfit}{varTheta}{"1D72D}
\usv_set:nnn {bfit}{varepsilon}{"1D750}
\usv_set:nnn {bfit}{vartheta}{"1D751}
\usv_set:nnn {bfit}{varkappa}{"1D752}
\usv_set:nnn {bfit}{varphi}{"1D753}
\usv_set:nnn {bfit}{varrho}{"1D754}
\usv_set:nnn {bfit}{varpi}{"1D755}
%    \end{macrocode}
% Bold sans:
%    \begin{macrocode}
\usv_set:nnn {bfsfup}{varTheta}{"1D767}
\usv_set:nnn {bfsfup}{varepsilon}{"1D78A}
\usv_set:nnn {bfsfup}{vartheta}{"1D78B}
\usv_set:nnn {bfsfup}{varkappa}{"1D78C}
\usv_set:nnn {bfsfup}{varphi}{"1D78D}
\usv_set:nnn {bfsfup}{varrho}{"1D78E}
\usv_set:nnn {bfsfup}{varpi}{"1D78F}
%    \end{macrocode}
% Bold sans italic:
%    \begin{macrocode}
\usv_set:nnn {bfsfit}{varTheta}  {"1D7A1}
\usv_set:nnn {bfsfit}{varepsilon}{"1D7C4}
\usv_set:nnn {bfsfit}{vartheta}  {"1D7C5}
\usv_set:nnn {bfsfit}{varkappa}  {"1D7C6}
\usv_set:nnn {bfsfit}{varphi}    {"1D7C7}
\usv_set:nnn {bfsfit}{varrho}    {"1D7C8}
\usv_set:nnn {bfsfit}{varpi}     {"1D7C9}
%    \end{macrocode}
% Nabla:
%    \begin{macrocode}
\usv_set:nnn {up}    {Nabla}{"02207}
\usv_set:nnn {it}    {Nabla}{"1D6FB}
\usv_set:nnn {bfup}  {Nabla}{"1D6C1}
\usv_set:nnn {bfit}  {Nabla}{"1D735}
\usv_set:nnn {bfsfup}{Nabla}{"1D76F}
\usv_set:nnn {bfsfit}{Nabla}{"1D7A9}
%    \end{macrocode}
% Partial:
%    \begin{macrocode}
\usv_set:nnn {up}    {partial}{"02202}
\usv_set:nnn {it}    {partial}{"1D715}
\usv_set:nnn {bfup}  {partial}{"1D6DB}
\usv_set:nnn {bfit}  {partial}{"1D74F}
\usv_set:nnn {bfsfup}{partial}{"1D789}
\usv_set:nnn {bfsfit}{partial}{"1D7C3}
%    \end{macrocode}
% \paragraph{Exceptions}
% These are need for mapping with the exceptions in other alphabets:
% (coming up)
%    \begin{macrocode}
\usv_set:nnn {up}{B}{`\B}
\usv_set:nnn {up}{C}{`\C}
\usv_set:nnn {up}{D}{`\D}
\usv_set:nnn {up}{E}{`\E}
\usv_set:nnn {up}{F}{`\F}
\usv_set:nnn {up}{H}{`\H}
\usv_set:nnn {up}{I}{`\I}
\usv_set:nnn {up}{L}{`\L}
\usv_set:nnn {up}{M}{`\M}
\usv_set:nnn {up}{N}{`\N}
\usv_set:nnn {up}{P}{`\P}
\usv_set:nnn {up}{Q}{`\Q}
\usv_set:nnn {up}{R}{`\R}
\usv_set:nnn {up}{Z}{`\Z}
%    \end{macrocode}
%
%    \begin{macrocode}
\usv_set:nnn {it}{B}{"1D435}
\usv_set:nnn {it}{C}{"1D436}
\usv_set:nnn {it}{D}{"1D437}
\usv_set:nnn {it}{E}{"1D438}
\usv_set:nnn {it}{F}{"1D439}
\usv_set:nnn {it}{H}{"1D43B}
\usv_set:nnn {it}{I}{"1D43C}
\usv_set:nnn {it}{L}{"1D43F}
\usv_set:nnn {it}{M}{"1D440}
\usv_set:nnn {it}{N}{"1D441}
\usv_set:nnn {it}{P}{"1D443}
\usv_set:nnn {it}{Q}{"1D444}
\usv_set:nnn {it}{R}{"1D445}
\usv_set:nnn {it}{Z}{"1D44D}
%    \end{macrocode}
%
%    \begin{macrocode}
\usv_set:nnn {up}{d}{`\d}
\usv_set:nnn {up}{e}{`\e}
\usv_set:nnn {up}{g}{`\g}
\usv_set:nnn {up}{h}{`\h}
\usv_set:nnn {up}{i}{`\i}
\usv_set:nnn {up}{j}{`\j}
\usv_set:nnn {up}{o}{`\o}
%    \end{macrocode}
%
%    \begin{macrocode}
\usv_set:nnn {it}{d}{"1D451}
\usv_set:nnn {it}{e}{"1D452}
\usv_set:nnn {it}{g}{"1D454}
\usv_set:nnn {it}{h}{"0210E}
\usv_set:nnn {it}{i}{"1D456}
\usv_set:nnn {it}{j}{"1D457}
\usv_set:nnn {it}{o}{"1D45C}
%    \end{macrocode}
% Latin `h':
%    \begin{macrocode}
\usv_set:nnn {bb}    {h}{"1D559}
\usv_set:nnn {tt}    {h}{"1D691}
\usv_set:nnn {scr}   {h}{"1D4BD}
\usv_set:nnn {frak}  {h}{"1D525}
\usv_set:nnn {bfup}  {h}{"1D421}
\usv_set:nnn {bfit}  {h}{"1D489}
\usv_set:nnn {sfup}  {h}{"1D5C1}
\usv_set:nnn {sfit}  {h}{"1D629}
\usv_set:nnn {bffrak}{h}{"1D58D}
\usv_set:nnn {bfscr} {h}{"1D4F1}
\usv_set:nnn {bfsfup}{h}{"1D5F5}
\usv_set:nnn {bfsfit}{h}{"1D65D}
%    \end{macrocode}
% Dotless `i' and `j:
%    \begin{macrocode}
\usv_set:nnn {up}{dotlessi}{"00131}
\usv_set:nnn {up}{dotlessj}{"00237}
\usv_set:nnn {it}{dotlessi}{"1D6A4}
\usv_set:nnn {it}{dotlessj}{"1D6A5}
%    \end{macrocode}
% Blackboard:
%    \begin{macrocode}
\usv_set:nnn {bb}{C}{"2102}
\usv_set:nnn {bb}{H}{"210D}
\usv_set:nnn {bb}{N}{"2115}
\usv_set:nnn {bb}{P}{"2119}
\usv_set:nnn {bb}{Q}{"211A}
\usv_set:nnn {bb}{R}{"211D}
\usv_set:nnn {bb}{Z}{"2124}
\usv_set:nnn {up}{Pi}       {"003A0}
\usv_set:nnn {up}{pi}       {"003C0}
\usv_set:nnn {up}{Gamma}    {"00393}
\usv_set:nnn {up}{gamma}    {"003B3}
\usv_set:nnn {up}{summation}{"02211}
\usv_set:nnn {it}{Pi}       {"1D6F1}
\usv_set:nnn {it}{pi}       {"1D70B}
\usv_set:nnn {it}{Gamma}    {"1D6E4}
\usv_set:nnn {it}{gamma}    {"1D6FE}
\usv_set:nnn {bb}{Pi}       {"0213F}
\usv_set:nnn {bb}{pi}       {"0213C}
\usv_set:nnn {bb}{Gamma}    {"0213E}
\usv_set:nnn {bb}{gamma}    {"0213D}
\usv_set:nnn {bb}{summation}{"02140}
%    \end{macrocode}
% Italic blackboard:
%    \begin{macrocode}
\usv_set:nnn {bbit}{D}{"2145}
\usv_set:nnn {bbit}{d}{"2146}
\usv_set:nnn {bbit}{e}{"2147}
\usv_set:nnn {bbit}{i}{"2148}
\usv_set:nnn {bbit}{j}{"2149}
%    \end{macrocode}
% Script exceptions:
%    \begin{macrocode}
\usv_set:nnn {scr}{B}{"212C}
\usv_set:nnn {scr}{E}{"2130}
\usv_set:nnn {scr}{F}{"2131}
\usv_set:nnn {scr}{H}{"210B}
\usv_set:nnn {scr}{I}{"2110}
\usv_set:nnn {scr}{L}{"2112}
\usv_set:nnn {scr}{M}{"2133}
\usv_set:nnn {scr}{R}{"211B}
\usv_set:nnn {scr}{e}{"212F}
\usv_set:nnn {scr}{g}{"210A}
\usv_set:nnn {scr}{o}{"2134}
%    \end{macrocode}
%
%    \begin{macrocode}
\usv_set:nnn {cal}{B}{"212C}
\usv_set:nnn {cal}{E}{"2130}
\usv_set:nnn {cal}{F}{"2131}
\usv_set:nnn {cal}{H}{"210B}
\usv_set:nnn {cal}{I}{"2110}
\usv_set:nnn {cal}{L}{"2112}
\usv_set:nnn {cal}{M}{"2133}
\usv_set:nnn {cal}{R}{"211B}
%    \end{macrocode}
% Fractur exceptions:
%    \begin{macrocode}
\usv_set:nnn {frak}{C}{"212D}
\usv_set:nnn {frak}{H}{"210C}
\usv_set:nnn {frak}{I}{"2111}
\usv_set:nnn {frak}{R}{"211C}
\usv_set:nnn {frak}{Z}{"2128}
%    \end{macrocode}
%
% \subsection{STIX fonts}
%
% Version 1.0.0 of the STIX fonts contains a number of
% alphabets in the private use area of Unicode; i.e.,
% it contains many math glyphs that have not (yet or if ever)
% been accepted into the Unicode standard.
%
% But we still want to be able to use them if possible.
%
%    \begin{macrocode}
%</package&(XE|LU)>
%<*stix>
%    \end{macrocode}
%
% \paragraph{Upright}
%    \begin{macrocode}
\usv_set:nnn {stixsfup}{partial}{"E17C}
\usv_set:nnn {stixsfup}{Greek}{"E17D}
\usv_set:nnn {stixsfup}{greek}{"E196}
\usv_set:nnn {stixsfup}{varTheta}{"E18E}
\usv_set:nnn {stixsfup}{varepsilon}{"E1AF}
\usv_set:nnn {stixsfup}{vartheta}{"E1B0}
\usv_set:nnn {stixsfup}{varkappa}{0000} % ???
\usv_set:nnn {stixsfup}{varphi}{"E1B1}
\usv_set:nnn {stixsfup}{varrho}{"E1B2}
\usv_set:nnn {stixsfup}{varpi}{"E1B3}
\usv_set:nnn {stixupslash}{Greek}{"E2FC}
%    \end{macrocode}
%
% \paragraph{Italic}
%    \begin{macrocode}
\usv_set:nnn {stixbbit}{A}{"E154}
\usv_set:nnn {stixbbit}{B}{"E155}
\usv_set:nnn {stixbbit}{E}{"E156}
\usv_set:nnn {stixbbit}{F}{"E157}
\usv_set:nnn {stixbbit}{G}{"E158}
\usv_set:nnn {stixbbit}{I}{"E159}
\usv_set:nnn {stixbbit}{J}{"E15A}
\usv_set:nnn {stixbbit}{K}{"E15B}
\usv_set:nnn {stixbbit}{L}{"E15C}
\usv_set:nnn {stixbbit}{M}{"E15D}
\usv_set:nnn {stixbbit}{O}{"E15E}
\usv_set:nnn {stixbbit}{S}{"E15F}
\usv_set:nnn {stixbbit}{T}{"E160}
\usv_set:nnn {stixbbit}{U}{"E161}
\usv_set:nnn {stixbbit}{V}{"E162}
\usv_set:nnn {stixbbit}{W}{"E163}
\usv_set:nnn {stixbbit}{X}{"E164}
\usv_set:nnn {stixbbit}{Y}{"E165}
%    \end{macrocode}
%
%    \begin{macrocode}
\usv_set:nnn {stixbbit}{a}{"E166}
\usv_set:nnn {stixbbit}{b}{"E167}
\usv_set:nnn {stixbbit}{c}{"E168}
\usv_set:nnn {stixbbit}{f}{"E169}
\usv_set:nnn {stixbbit}{g}{"E16A}
\usv_set:nnn {stixbbit}{h}{"E16B}
\usv_set:nnn {stixbbit}{k}{"E16C}
\usv_set:nnn {stixbbit}{l}{"E16D}
\usv_set:nnn {stixbbit}{m}{"E16E}
\usv_set:nnn {stixbbit}{n}{"E16F}
\usv_set:nnn {stixbbit}{o}{"E170}
\usv_set:nnn {stixbbit}{p}{"E171}
\usv_set:nnn {stixbbit}{q}{"E172}
\usv_set:nnn {stixbbit}{r}{"E173}
\usv_set:nnn {stixbbit}{s}{"E174}
\usv_set:nnn {stixbbit}{t}{"E175}
\usv_set:nnn {stixbbit}{u}{"E176}
\usv_set:nnn {stixbbit}{v}{"E177}
\usv_set:nnn {stixbbit}{w}{"E178}
\usv_set:nnn {stixbbit}{x}{"E179}
\usv_set:nnn {stixbbit}{y}{"E17A}
\usv_set:nnn {stixbbit}{z}{"E17B}
%    \end{macrocode}
%
%    \begin{macrocode}
\usv_set:nnn {stixsfit}{Numerals}{"E1B4}
\usv_set:nnn {stixsfit}{partial}{"E1BE}
\usv_set:nnn {stixsfit}{Greek}{"E1BF}
\usv_set:nnn {stixsfit}{greek}{"E1D8}
\usv_set:nnn {stixsfit}{varTheta}{"E1D0}
\usv_set:nnn {stixsfit}{varepsilon}{"E1F1}
\usv_set:nnn {stixsfit}{vartheta}{"E1F2}
\usv_set:nnn {stixsfit}{varkappa}{0000} % ???
\usv_set:nnn {stixsfit}{varphi}{"E1F3}
\usv_set:nnn {stixsfit}{varrho}{"E1F4}
\usv_set:nnn {stixsfit}{varpi}{"E1F5}
%    \end{macrocode}
%
%    \begin{macrocode}
\usv_set:nnn {stixcal}{Latin}{"E22D}
\usv_set:nnn {stixcal}{num}{"E262}
\usv_set:nnn {scr}{num}{48}
\usv_set:nnn {it}{num}{48}
%    \end{macrocode}
%
%    \begin{macrocode}
\usv_set:nnn {stixsfitslash}{Latin}{"E294}
\usv_set:nnn {stixsfitslash}{latin}{"E2C8}
\usv_set:nnn {stixsfitslash}{greek}{"E32C}
\usv_set:nnn {stixsfitslash}{varepsilon}{"E37A}
\usv_set:nnn {stixsfitslash}{vartheta}{"E35E}
\usv_set:nnn {stixsfitslash}{varkappa}{"E374}
\usv_set:nnn {stixsfitslash}{varphi}{"E360}
\usv_set:nnn {stixsfitslash}{varrho}{"E376}
\usv_set:nnn {stixsfitslash}{varpi}{"E362}
\usv_set:nnn {stixsfitslash}{digamma}{"E36A}
%    \end{macrocode}
%
% \paragraph{Bold}
%
%    \begin{macrocode}
\usv_set:nnn {stixbfupslash}{Greek}{"E2FD}
\usv_set:nnn {stixbfupslash}{Digamma}{"E369}
%    \end{macrocode}
%
%    \begin{macrocode}
\usv_set:nnn {stixbfbb}{A}{"E38A}
\usv_set:nnn {stixbfbb}{B}{"E38B}
\usv_set:nnn {stixbfbb}{E}{"E38D}
\usv_set:nnn {stixbfbb}{F}{"E38E}
\usv_set:nnn {stixbfbb}{G}{"E38F}
\usv_set:nnn {stixbfbb}{I}{"E390}
\usv_set:nnn {stixbfbb}{J}{"E391}
\usv_set:nnn {stixbfbb}{K}{"E392}
\usv_set:nnn {stixbfbb}{L}{"E393}
\usv_set:nnn {stixbfbb}{M}{"E394}
\usv_set:nnn {stixbfbb}{O}{"E395}
\usv_set:nnn {stixbfbb}{S}{"E396}
\usv_set:nnn {stixbfbb}{T}{"E397}
\usv_set:nnn {stixbfbb}{U}{"E398}
\usv_set:nnn {stixbfbb}{V}{"E399}
\usv_set:nnn {stixbfbb}{W}{"E39A}
\usv_set:nnn {stixbfbb}{X}{"E39B}
\usv_set:nnn {stixbfbb}{Y}{"E39C}
%    \end{macrocode}
%
%    \begin{macrocode}
\usv_set:nnn {stixbfbb}{a}{"E39D}
\usv_set:nnn {stixbfbb}{b}{"E39E}
\usv_set:nnn {stixbfbb}{c}{"E39F}
\usv_set:nnn {stixbfbb}{f}{"E3A2}
\usv_set:nnn {stixbfbb}{g}{"E3A3}
\usv_set:nnn {stixbfbb}{h}{"E3A4}
\usv_set:nnn {stixbfbb}{k}{"E3A7}
\usv_set:nnn {stixbfbb}{l}{"E3A8}
\usv_set:nnn {stixbfbb}{m}{"E3A9}
\usv_set:nnn {stixbfbb}{n}{"E3AA}
\usv_set:nnn {stixbfbb}{o}{"E3AB}
\usv_set:nnn {stixbfbb}{p}{"E3AC}
\usv_set:nnn {stixbfbb}{q}{"E3AD}
\usv_set:nnn {stixbfbb}{r}{"E3AE}
\usv_set:nnn {stixbfbb}{s}{"E3AF}
\usv_set:nnn {stixbfbb}{t}{"E3B0}
\usv_set:nnn {stixbfbb}{u}{"E3B1}
\usv_set:nnn {stixbfbb}{v}{"E3B2}
\usv_set:nnn {stixbfbb}{w}{"E3B3}
\usv_set:nnn {stixbfbb}{x}{"E3B4}
\usv_set:nnn {stixbfbb}{y}{"E3B5}
\usv_set:nnn {stixbfbb}{z}{"E3B6}
%    \end{macrocode}
%
%    \begin{macrocode}
\usv_set:nnn {stixbfsfup}{Numerals}{"E3B7}
%    \end{macrocode}
%
% \paragraph{Bold Italic}
%    \begin{macrocode}
\usv_set:nnn {stixbfsfit}{Numerals}{"E1F6}
%    \end{macrocode}
%
%    \begin{macrocode}
\usv_set:nnn {stixbfbbit}{A}{"E200}
\usv_set:nnn {stixbfbbit}{B}{"E201}
\usv_set:nnn {stixbfbbit}{E}{"E203}
\usv_set:nnn {stixbfbbit}{F}{"E204}
\usv_set:nnn {stixbfbbit}{G}{"E205}
\usv_set:nnn {stixbfbbit}{I}{"E206}
\usv_set:nnn {stixbfbbit}{J}{"E207}
\usv_set:nnn {stixbfbbit}{K}{"E208}
\usv_set:nnn {stixbfbbit}{L}{"E209}
\usv_set:nnn {stixbfbbit}{M}{"E20A}
\usv_set:nnn {stixbfbbit}{O}{"E20B}
\usv_set:nnn {stixbfbbit}{S}{"E20C}
\usv_set:nnn {stixbfbbit}{T}{"E20D}
\usv_set:nnn {stixbfbbit}{U}{"E20E}
\usv_set:nnn {stixbfbbit}{V}{"E20F}
\usv_set:nnn {stixbfbbit}{W}{"E210}
\usv_set:nnn {stixbfbbit}{X}{"E211}
\usv_set:nnn {stixbfbbit}{Y}{"E212}
%    \end{macrocode}
%
%    \begin{macrocode}
\usv_set:nnn {stixbfbbit}{a}{"E213}
\usv_set:nnn {stixbfbbit}{b}{"E214}
\usv_set:nnn {stixbfbbit}{c}{"E215}
\usv_set:nnn {stixbfbbit}{e}{"E217}
\usv_set:nnn {stixbfbbit}{f}{"E218}
\usv_set:nnn {stixbfbbit}{g}{"E219}
\usv_set:nnn {stixbfbbit}{h}{"E21A}
\usv_set:nnn {stixbfbbit}{k}{"E21D}
\usv_set:nnn {stixbfbbit}{l}{"E21E}
\usv_set:nnn {stixbfbbit}{m}{"E21F}
\usv_set:nnn {stixbfbbit}{n}{"E220}
\usv_set:nnn {stixbfbbit}{o}{"E221}
\usv_set:nnn {stixbfbbit}{p}{"E222}
\usv_set:nnn {stixbfbbit}{q}{"E223}
\usv_set:nnn {stixbfbbit}{r}{"E224}
\usv_set:nnn {stixbfbbit}{s}{"E225}
\usv_set:nnn {stixbfbbit}{t}{"E226}
\usv_set:nnn {stixbfbbit}{u}{"E227}
\usv_set:nnn {stixbfbbit}{v}{"E228}
\usv_set:nnn {stixbfbbit}{w}{"E229}
\usv_set:nnn {stixbfbbit}{x}{"E22A}
\usv_set:nnn {stixbfbbit}{y}{"E22B}
\usv_set:nnn {stixbfbbit}{z}{"E22C}
%    \end{macrocode}
%
%    \begin{macrocode}
\usv_set:nnn {stixbfcal}{Latin}{"E247}
%    \end{macrocode}
%
%    \begin{macrocode}
\usv_set:nnn {stixbfitslash}{Latin}{"E295}
\usv_set:nnn {stixbfitslash}{latin}{"E2C9}
\usv_set:nnn {stixbfitslash}{greek}{"E32D}
\usv_set:nnn {stixsfitslash}{varepsilon}{"E37B}
\usv_set:nnn {stixsfitslash}{vartheta}{"E35F}
\usv_set:nnn {stixsfitslash}{varkappa}{"E375}
\usv_set:nnn {stixsfitslash}{varphi}{"E361}
\usv_set:nnn {stixsfitslash}{varrho}{"E377}
\usv_set:nnn {stixsfitslash}{varpi}{"E363}
\usv_set:nnn {stixsfitslash}{digamma}{"E36B}
%    \end{macrocode}
%
%    \begin{macrocode}
%</stix>
%<*package&(XE|LU)>
%    \end{macrocode}
%
% \subsection{Overcoming \texorpdfstring{\cmd\@onlypreamble}{\textbackslash @onlypreamble}}
%
% The requirement of only setting up the maths fonts in the preamble is now removed. The following list might be overly ambitious.
%    \begin{macrocode}
\tl_map_inline:nn {
  \new@mathgroup\cdp@list\cdp@elt\DeclareMathSizes
  \@DeclareMathSizes\newmathalphabet\newmathalphabet@@\newmathalphabet@@@
  \DeclareMathVersion\define@mathalphabet\define@mathgroup\addtoversion
  \version@list\version@elt\alpha@list\alpha@elt
  \restore@mathversion\init@restore@version\dorestore@version\process@table
  \new@mathversion\DeclareSymbolFont\group@list\group@elt
  \new@symbolfont\SetSymbolFont\SetSymbolFont@\get@cdp
  \DeclareMathAlphabet\new@mathalphabet\SetMathAlphabet\SetMathAlphabet@
  \DeclareMathAccent\set@mathaccent\DeclareMathSymbol\set@mathchar
  \set@mathsymbol\DeclareMathDelimiter\@xxDeclareMathDelimiter
  \@DeclareMathDelimiter\@xDeclareMathDelimiter\set@mathdelimiter
  \set@@mathdelimiter\DeclareMathRadical\mathchar@type
  \DeclareSymbolFontAlphabet\DeclareSymbolFontAlphabet@
}{
  \tl_remove_once:Nn \@preamblecmds {\do#1}
}
%    \end{macrocode}
%
% \section{Fundamentals}
%
% \subsection{Enlarging the number of maths families}
%
% To start with, we've got a power of two as many \cmd\fam s as before. So (from |ltfssbas.dtx|) we want to redefine
%    \begin{macrocode}
%<*XE>
\def\new@mathgroup{\alloc@8\mathgroup\chardef\@cclvi}
\let\newfam\new@mathgroup
%</XE>
%    \end{macrocode}
% This is sufficient for \LaTeX's \cmd\DeclareSymbolFont-type commands to be able
% to define 256 named maths fonts.
% For \hologo{LuaLaTeX}, this is handled by the \pkg{lualatex-math} package.
%
% \subsection{Setting math chars, math codes, etc.}
%
% \begin{macro}{\um_set_mathsymbol:nNNn}
% \darg{A \LaTeX\ symbol font, e.g., \texttt{operators}}
% \darg{Symbol macro, \eg, \cmd\alpha}
% \darg{Type, \eg, \cmd\mathalpha}
% \darg{Slot, \eg, \texttt{"221E}}
% There are a bunch of tests to perform to process the various characters.
% The following assignments should all be fairly straightforward.
%    \begin{macrocode}
\cs_set:Nn \um_set_mathsymbol:nNNn {
  \prg_case_tl:Nnn #3 {
    \mathop { \um_set_big_operator:nnn {#1} {#2} {#4} }
    \mathopen
      {
        \tl_if_in:NnTF \l_um_radicals_tl {#2}
          {
            \cs_gset_protected_nopar:cpx {\cs_to_str:N #2 sign}
              { \um_radical:nn {#1} {#4} }
            \tl_set:cn {l_um_radical_\cs_to_str:N #2_tl} {\use:c{sym #1}~ #4}
          }
          {
            \um_set_delcode:nnn {#1} {#4} {#4}
            \um_set_mathcode:nnn {#4} \mathopen {#1}
            \cs_gset_protected_nopar:Npx #2
              { \um_delimiter:Nnn \mathopen {#1} {#4} }
          }
      }
    \mathclose
      {
        \um_set_delcode:nnn {#1} {#4} {#4}
        \um_set_mathcode:nnn {#4} \mathclose {#1}
        \cs_gset_protected_nopar:Npx #2
          { \um_delimiter:Nnn \mathclose {#1} {#4} }
      }
    \mathfence
      {
        \um_set_mathcode:nnn {#4} {#3} {#1}
        \um_set_delcode:nnn {#1} {#4} {#4}
        \cs_gset_protected_nopar:cpx {l \cs_to_str:N #2}
          { \um_delimiter:Nnn \mathopen  {#1} {#4} }
        \cs_gset_protected_nopar:cpx {r \cs_to_str:N #2}
          { \um_delimiter:Nnn \mathclose {#1} {#4} }
      }
    \mathaccent
      { \cs_gset_protected_nopar:Npx #2 { \um_accent:nnn {fixed} {#1} {#4} } }
    \mathbotaccent
      { \cs_gset_protected_nopar:Npx #2 { \um_accent:nnn {bottom~ fixed} {#1} {#4} } }
    \mathover
      {
        \cs_set_protected_nopar:Npx #2 ##1
          { \mathop { \um_accent:nnn {} {#1} {#4} {##1} } \limits }
      }
    \mathunder
      {
        \cs_set_protected_nopar:Npx #2 ##1
          { \mathop { \um_accent:nnn {bottom} {#1} {#4} {##1} } \limits }
      }
  }{
    \um_set_mathcode:nnn {#4} {#3} {#1}
  }
}
%    \end{macrocode}
% \end{macro}
%
%    \begin{macrocode}
\edef\mathfence{\string\mathfence}
\edef\mathover{\string\mathover}
\edef\mathunder{\string\mathunder}
\edef\mathbotaccent{\string\mathbotaccent}
%    \end{macrocode}
%
%
% \begin{macro}{\um_set_big_operator:nnn}
% \darg{Symbol font name}
% \darg{Macro to assign}
% \darg{Glyph slot}
% In the examples following, say we're defining for the symbol \cmd\sum\ ($\sum$).
% In order for literal Unicode characters to be used in the source and still
% have the correct limits behaviour, big operators are made math-active.
% This involves three steps:
% \begin{itemize}
% \item
% The active math char is defined to expand to the macro \cs{sum_sym}.
% (Later, the control sequence \cs{sum} will be assigned the math char.)
% \item
% Declare the plain old mathchardef for the control sequence \cmd\sumop.
% (This follows the convention of \LaTeX/\pkg{amsmath}.)
% \item
% Define \cs{sum_sym} as \cmd\sumop, followed by \cmd\nolimits\ if necessary.
% \end{itemize}
% Whether the \cmd\nolimits\ suffix is inserted is controlled by the
% token list \cs{l_um_nolimits_tl}, which contains a list of such characters.
% This list is checked dynamically to allow it to be updated mid-document.
%
% Examples of expansion, by default, for two big operators:
% \begin{quote}
% (~\cs{sum} $\to$~) $\sum$ $\to$ \cs{sum_sym} $\to$ \cs{sumop}\cs{nolimits}\par
% (~\cs{int} $\to$~) $\int$ $\to$ \cs{int_sym} $\to$ \cs{intop}
% \end{quote}
%    \begin{macrocode}
\cs_new:Nn \um_set_big_operator:nnn {
  \group_begin:
    \char_set_catcode_active:n {#3}
    \char_gmake_mathactive:n {#3}
    \um_active_char_set:wc #3 \q_nil { \cs_to_str:N #2 _sym }
  \group_end:
  \um_set_mathchar:cNnn {\cs_to_str:N #2 op} \mathop {#1} {#3}
  \cs_gset:cpx { \cs_to_str:N #2 _sym } {
    \exp_not:c { \cs_to_str:N #2 op   }
    \exp_not:n { \tl_if_in:NnT \l_um_nolimits_tl {#2} \nolimits }
  }
}
%    \end{macrocode}
% \end{macro}
%
%
% \begin{macro}{\um_set_mathcode:nnnn}
% \begin{macro}{\um_set_mathcode:nnn}
% \begin{macro}{\um_set_mathchar:NNnn}
% \begin{macro}{\um_set_mathchar:cNnn}
% \begin{macro}{\um_set_delcode:nnn}
% \begin{macro}{\um_radical:nn}
% \begin{macro}{\um_delimiter:Nnn}
% \begin{macro}{\um_accent:nnn}
% \begin{macro}{\um_accent_keyword:}
% These are all wrappers for the primitive commands that take numerical
% input only.
%    \begin{macrocode}
\cs_set:Npn \um_set_mathcode:nnnn #1#2#3#4 {
  \Umathcode \int_eval:n {#1} =
    \mathchar@type#2 \csname sym#3\endcsname \int_eval:n {#4} \scan_stop:
}
\cs_set:Npn \um_set_mathcode:nnn #1#2#3 {
  \Umathcode \int_eval:n {#1} =
    \mathchar@type#2 \csname sym#3\endcsname \int_eval:n {#1} \scan_stop:
}
\cs_set:Npn \um_set_mathchar:NNnn #1#2#3#4 {
  \Umathchardef #1 =
    \mathchar@type#2 \csname sym#3\endcsname \int_eval:n {#4} \scan_stop:
}
\cs_new:Nn \um_set_delcode:nnn {
  \Udelcode#2 = \csname sym#1\endcsname #3
}
\cs_new:Nn \um_radical:nn {
  \Uradical \csname sym#1\endcsname #2 \scan_stop:
}
\cs_new:Nn \um_delimiter:Nnn {
  \Udelimiter \mathchar@type#1 \csname sym#2\endcsname #3 \scan_stop:
}
\cs_new:Nn \um_accent:nnn {
  \Umathaccent #1~ \mathchar@type\mathaccent \use:c { sym #2 } #3 \scan_stop:
}
%    \end{macrocode}
%
%    \begin{macrocode}
\cs_generate_variant:Nn \um_set_mathchar:NNnn {c}
%    \end{macrocode}
% \end{macro}
% \end{macro}
% \end{macro}
% \end{macro}
% \end{macro}
% \end{macro}
% \end{macro}
% \end{macro}
% \end{macro}
%
%
% \begin{macro}{\char_gmake_mathactive:N}
% \begin{macro}{\char_gmake_mathactive:n}
%    \begin{macrocode}
\cs_new:Nn \char_gmake_mathactive:N {
  \global\mathcode `#1 = "8000 \scan_stop:
}
\cs_new:Nn \char_gmake_mathactive:n {
  \global\mathcode #1 = "8000 \scan_stop:
}
%    \end{macrocode}
% \end{macro}
% \end{macro}
%
%
% \subsection{The main \cs{setmathfont} macro}
%
% Using a |range| including large character sets such as \cmd\mathrel,
% \cmd\mathalpha, \etc, is \emph{very slow}!
% I hope to improve the performance somehow.
%
% \begin{macro}{\setmathfont}
% \doarg{font features}
% \darg{font name}
%    \begin{macrocode}
\cs_new:Nn \um_init: {
  \bool_set_true:N  \l_um_ot_math_bool
%    \end{macrocode}
% \begin{itemize}
% \item Erase any conception \LaTeX\ has of previously defined math symbol fonts;
% this allows \cmd\DeclareSymbolFont\ at any point in the document.
%    \begin{macrocode}
  \cs_set_eq:NN \glb@currsize \scan_stop:
%    \end{macrocode}
% \item To start with, assume we're defining the font for every math symbol character.
%    \begin{macrocode}
  \bool_set_true:N \l_um_init_bool
  \seq_clear:N \l_um_char_range_seq
  \clist_clear:N \l_um_char_num_range_clist
  \seq_clear:N \l_um_mathalph_seq
  \seq_clear:N \l_um_missing_alph_seq
%    \end{macrocode}
% \item By default use the `normal' math version
%    \begin{macrocode}
   \tl_set:Nn \l_um_mversion_tl {normal}
%    \end{macrocode}
% \item Other range initialisations
%    \begin{macrocode}
    \tl_set:Nn \um_symfont_tl {operators}
    \cs_set_eq:NN \_um_sym:nnn \um_process_symbol_noparse:nnn
    \cs_set_eq:NN \um_set_mathalphabet_char:Nnn \um_mathmap_noparse:Nnn
    \cs_set_eq:NN \um_remap_symbol:nnn \um_remap_symbol_noparse:nnn
    \cs_set_eq:NN \um_maybe_init_alphabet:n \um_init_alphabet:n
    \cs_set_eq:NN \um_map_char_single:nn \um_map_char_noparse:nn
    \cs_set_eq:NN \um_assign_delcode:nn \um_assign_delcode_noparse:nn
    \cs_set_eq:NN \um_make_mathactive:nNN \um_make_mathactive_noparse:nNN
%    \end{macrocode}
% \item Define default font features for the script and scriptscript font.
%    \begin{macrocode}
  \tl_set:Nn \l_um_script_features_tl  {Style=MathScript}
  \tl_set:Nn \l_um_sscript_features_tl {Style=MathScriptScript}
  \tl_set_eq:NN \l_um_script_font_tl   \l_um_fontname_tl
  \tl_set_eq:NN \l_um_sscript_font_tl  \l_um_fontname_tl
%    \end{macrocode}
% \end{itemize}
%    \begin{macrocode}
}
\DeclareDocumentCommand \setmathfont { O{} m } {
  \tl_set:Nn \l_um_fontname_tl {#2}
  \um_init:
%    \end{macrocode}
% Grab the current size information:
% (is this robust enough? Maybe it should be preceded by \cmd\normalsize).
% The macro \cmd\S@\meta{size}
% contains the definitions of the sizes used for maths letters, subscripts and subsubscripts in
% \cmd\tf@size, \cmd\sf@size, and \cmd\ssf@size, respectively.
%    \begin{macrocode}
  \cs_if_exist:cF { S@ \f@size } { \calculate@math@sizes }
  \csname S@\f@size\endcsname
%    \end{macrocode}
% Parse options and tell people what's going on:
%    \begin{macrocode}
  \keys_set_known:nnN {unicode-math} {#1} \l_um_unknown_keys_clist
  \bool_if:NT \l_um_init_bool { \um_log:n {default-math-font} }
%    \end{macrocode}
% Use \pkg{fontspec} to select a font to use.
%    \begin{macrocode}
  \um_fontspec_select_font:
%    \end{macrocode}
% Now define |\um_symfont_tl| as the \LaTeX\ math font to access everything:
%    \begin{macrocode}
  \cs_if_exist:cF { sym \um_symfont_tl }
    {
      \DeclareSymbolFont{\um_symfont_tl}
        {\encodingdefault}{\l_um_family_tl}{\mddefault}{\updefault}
    }
  \SetSymbolFont{\um_symfont_tl}{\l_um_mversion_tl}
    {\encodingdefault}{\l_um_family_tl}{\mddefault}{\updefault}
%    \end{macrocode}
% Set the bold math version.
%    \begin{macrocode}
  \tl_set:Nn \l_um_tmpa_tl {normal}
  \tl_if_eq:NNT \l_um_mversion_tl \l_um_tmpa_tl
    {
    \SetSymbolFont{\um_symfont_tl}{bold}
      {\encodingdefault}{\l_um_family_tl}{\bfdefault}{\updefault}
    }
%    \end{macrocode}
% Declare the math sizes (i.e., scaling of superscripts) for the specific
% values for this font,
% and set defaults for math fams two and three for legacy compatibility:
%    \begin{macrocode}
  \bool_if:nT {\l_um_ot_math_bool && !\g_um_mainfont_already_set_bool} {
    \bool_set_true:N \g_um_mainfont_already_set_bool
    \um_declare_math_sizes:
    \um_setup_legacy_fam_two:
    \um_setup_legacy_fam_three:
  }
%    \end{macrocode}
% And now we input every single maths char.
%    \begin{macrocode}
  \um_input_math_symbol_table:
%    \end{macrocode}
% Finally,
% \begin{itemize}
% \item Remap symbols that don't take their natural mathcode
% \item Activate any symbols that need to be math-active
% \item Enable wide/narrow accents
% \item Assign delimiter codes for symbols that need to grow
% \item Setup the maths alphabets (\cs{mathbf} etc.)
% \end{itemize}
%    \begin{macrocode}
  \um_remap_symbols:
  \um_setup_mathactives:
  \um_setup_accents:
  \um_setup_delcodes:
  \um_setup_alphabets:
  \um_setup_negations:
%    \end{macrocode}
% Prevent spaces, and that's it:
%    \begin{macrocode}
  \ignorespaces
}
%    \end{macrocode}
% \end{macro}
%
%
% \begin{macro}{\um_declare_math_sizes:}
% Set the math sizes according to the recommend font parameters:
%    \begin{macrocode}
\cs_new:Nn \um_declare_math_sizes:
  {
    \dim_compare:nF { \fontdimen 10 \l_um_font == 0pt }
      {
        \DeclareMathSizes { \f@size } { \f@size }
          { \um_fontdimen_to_scale:nn {10} {\l_um_font} }
          { \um_fontdimen_to_scale:nn {11} {\l_um_font} }
      }
  }
%    \end{macrocode}
% \end{macro}
%
%
%
% \begin{macro}{\um_setup_legacy_fam_two:}
%    \begin{macrocode}
\cs_new:Nn \um_setup_legacy_fam_two:
  {
    \fontspec_set_family:Nxn \l_um_family_tl
      {
      \l_um_font_keyval_tl,
      Scale=1.00001,
      FontAdjustment={
        \fontdimen8\font= \um_get_fontparam:nn {43} {FractionNumeratorDisplayStyleShiftUp}\relax
        \fontdimen9\font= \um_get_fontparam:nn {42} {FractionNumeratorShiftUp}\relax
        \fontdimen10\font=\um_get_fontparam:nn {32} {StackTopShiftUp}\relax
        \fontdimen11\font=\um_get_fontparam:nn {45} {FractionDenominatorDisplayStyleShiftDown}\relax
        \fontdimen12\font=\um_get_fontparam:nn {44} {FractionDenominatorShiftDown}\relax
        \fontdimen13\font=\um_get_fontparam:nn {21} {SuperscriptShiftUp}\relax
        \fontdimen14\font=\um_get_fontparam:nn {21} {SuperscriptShiftUp}\relax
        \fontdimen15\font=\um_get_fontparam:nn {22} {SuperscriptShiftUpCramped}\relax
        \fontdimen16\font=\um_get_fontparam:nn {18} {SubscriptShiftDown}\relax
        \fontdimen17\font=\um_get_fontparam:nn {18} {SubscriptShiftDownWithSuperscript}\relax
        \fontdimen18\font=\um_get_fontparam:nn {24} {SuperscriptBaselineDropMax}\relax
        \fontdimen19\font=\um_get_fontparam:nn {20} {SubscriptBaselineDropMin}\relax
        \fontdimen20\font=0pt\relax % delim1 = FractionDelimiterDisplaySize
        \fontdimen21\font=0pt\relax % delim2 = FractionDelimiterSize
        \fontdimen22\font=\um_get_fontparam:nn {15} {AxisHeight}\relax
      }
      } {\l_um_fontname_tl}
    \SetSymbolFont{symbols}{\l_um_mversion_tl}
      {\encodingdefault}{\l_um_family_tl}{\mddefault}{\updefault}

    \tl_set:Nn \l_um_tmpa_tl {normal}
    \tl_if_eq:NNT \l_um_mversion_tl \l_um_tmpa_tl
      {
      \SetSymbolFont{symbols}{bold}
        {\encodingdefault}{\l_um_family_tl}{\bfdefault}{\updefault}
      }
  }
%    \end{macrocode}
% \end{macro}
%
% \begin{macro}{\um_setup_legacy_fam_three:}
%    \begin{macrocode}
\cs_new:Nn \um_setup_legacy_fam_three:
  {
    \fontspec_set_family:Nxn \l_um_family_tl
      {
      \l_um_font_keyval_tl,
      Scale=0.99999,
      FontAdjustment={
        \fontdimen8\font= \um_get_fontparam:nn {48} {FractionRuleThickness}\relax
        \fontdimen9\font= \um_get_fontparam:nn {28} {UpperLimitGapMin}\relax
        \fontdimen10\font=\um_get_fontparam:nn {30} {LowerLimitGapMin}\relax
        \fontdimen11\font=\um_get_fontparam:nn {29} {UpperLimitBaselineRiseMin}\relax
        \fontdimen12\font=\um_get_fontparam:nn {31} {LowerLimitBaselineDropMin}\relax
        \fontdimen13\font=0pt\relax
      }
    } {\l_um_fontname_tl}
    \SetSymbolFont{largesymbols}{\l_um_mversion_tl}
      {\encodingdefault}{\l_um_family_tl}{\mddefault}{\updefault}

    \tl_set:Nn \l_um_tmpa_tl {normal}
    \tl_if_eq:NNT \l_um_mversion_tl \l_um_tmpa_tl
      {
      \SetSymbolFont{largesymbols}{bold}
        {\encodingdefault}{\l_um_family_tl}{\bfdefault}{\updefault}
      }
  }
%    \end{macrocode}
% \end{macro}
%
%
%    \begin{macrocode}
\cs_new:Nn \um_get_fontparam:nn
%<XE>  { \the\fontdimen#1\l_um_font\relax }
%<LU>  { \directlua{fontspec.mathfontdimen("l_um_font","#2")} }
%    \end{macrocode}
%
% Backward compatibility alias.
%    \begin{macrocode}
\cs_set_eq:NN \resetmathfont \setmathfont
%    \end{macrocode}
%
%
% \begin{macro}{\um_fontspec_select_font:}
% Select the font with \cs{fontspec} and define \cs{l_um_font} from it.
%    \begin{macrocode}
\cs_new:Nn \um_fontspec_select_font: {
  \tl_set:Nx \l_um_font_keyval_tl {
%<LU>     Renderer = Basic,
    BoldItalicFont = {}, ItalicFont = {},
    Script = Math,
    SizeFeatures = {
      {Size = \tf@size-} ,
      {Size = \sf@size-\tf@size ,
       Font = \l_um_script_font_tl ,
       \l_um_script_features_tl
      } ,
      {Size = -\sf@size ,
       Font = \l_um_sscript_font_tl ,
       \l_um_sscript_features_tl
      }
    },
    \l_um_unknown_keys_clist
  }
  \fontspec_set_fontface:NNxn \l_um_font \l_um_family_tl
    {\l_um_font_keyval_tl} {\l_um_fontname_tl}
%    \end{macrocode}
% Check whether we're using a real maths font:
%    \begin{macrocode}
  \group_begin:
    \fontfamily{\l_um_family_tl}\selectfont
    \fontspec_if_script:nF {math} {\bool_gset_false:N \l_um_ot_math_bool}
  \group_end:
}
%    \end{macrocode}
% \end{macro}
%
%
% \subsubsection{Functions for setting up symbols with mathcodes}
% \seclabel{mathsymbol}
%
% \begin{macro}{\um_process_symbol_noparse:nnn}
% \begin{macro}{\um_process_symbol_parse:nnn}
% If the \feat{range} font feature has been used, then only
% a subset of the Unicode glyphs are to be defined.
% See \secref{rangeproc} for the code that enables this.
%    \begin{macrocode}
\cs_set:Npn \um_process_symbol_noparse:nnn #1#2#3 {
  \um_set_mathsymbol:nNNn {\um_symfont_tl} #2#3{#1}
}
%    \end{macrocode}
%    \begin{macrocode}
\cs_set:Npn \um_process_symbol_parse:nnn #1#2#3 {
  \um_if_char_spec:nNNT{#1}{#2}{#3}{
    \um_process_symbol_noparse:nnn {#1}{#2}{#3}
  }
}
%    \end{macrocode}
% \end{macro}
% \end{macro}
%
%
% \begin{macro}{\um_remap_symbols:}
% \begin{macro}{\um_remap_symbol_noparse:nnn}
% \begin{macro}{\um_remap_symbol_parse:nnn}
% This function is used to define the mathcodes for those chars which should
% be mapped to a different glyph than themselves.
%    \begin{macrocode}
\cs_new:Npn \um_remap_symbols: {
  \um_remap_symbol:nnn{`\-}{\mathbin}{"02212}% hyphen to minus
  \um_remap_symbol:nnn{`\*}{\mathbin}{"02217}% text asterisk to "centred asterisk"
  \bool_if:NF \g_um_literal_colon_bool {
    \um_remap_symbol:nnn{`\:}{\mathrel}{"02236}% colon to ratio (i.e., punct to rel)
  }
}
%    \end{macrocode}
% \end{macro}
% Where |\um_remap_symbol:nnn| is defined to be one of these two, depending
% on the range setup:
%    \begin{macrocode}
\cs_new:Nn \um_remap_symbol_parse:nnn {
  \um_if_char_spec:nNNT {#3} {\@nil} {#2} {
    \um_remap_symbol_noparse:nnn {#1} {#2} {#3}
  }
}
\cs_new:Nn \um_remap_symbol_noparse:nnn {
  \clist_map_inline:nn {#1} {
    \um_set_mathcode:nnnn {##1} {#2} {\um_symfont_tl} {#3}
  }
}
%    \end{macrocode}
% \end{macro}
% \end{macro}
%
%
% \subsubsection{Active math characters}
%
% There are more math active chars later in the subscript/superscript section.
% But they don't need to be able to be typeset directly.
%
% \begin{macro}{\um_setup_mathactives:}
%    \begin{macrocode}
\cs_new:Npn \um_setup_mathactives: {
  \um_make_mathactive:nNN {"2032} \um_prime_single_mchar \mathord
  \um_make_mathactive:nNN {"2033} \um_prime_double_mchar \mathord
  \um_make_mathactive:nNN {"2034} \um_prime_triple_mchar \mathord
  \um_make_mathactive:nNN {"2057} \um_prime_quad_mchar   \mathord
  \um_make_mathactive:nNN {"2035} \um_backprime_single_mchar \mathord
  \um_make_mathactive:nNN {"2036} \um_backprime_double_mchar \mathord
  \um_make_mathactive:nNN {"2037} \um_backprime_triple_mchar \mathord
  \um_make_mathactive:nNN {`\'} \mathstraightquote \mathord
  \um_make_mathactive:nNN {`\`} \mathbacktick      \mathord
}
%    \end{macrocode}
% \end{macro}
%
% \begin{macro}{\um_make_mathactive:nNN}
% Makes |#1| a mathactive char, and gives cs |#2| the meaning of mathchar |#1|
% with class |#3|.
% You are responsible for giving active |#1| a particular meaning!
%    \begin{macrocode}
\cs_new:Nn \um_make_mathactive_parse:nNN
  {
    \um_if_char_spec:nNNT {#1} #2 #3
      { \um_make_mathactive_noparse:nNN {#1} #2 #3 }
  }
\cs_new:Nn \um_make_mathactive_noparse:nNN
  {
    \um_set_mathchar:NNnn #2 #3 {\um_symfont_tl} {#1}
    \char_gmake_mathactive:n {#1}
  }
%    \end{macrocode}
% \end{macro}
%
% \subsubsection{Delimiter codes}
%
%
% \begin{macro}{\um_assign_delcode:nn}
%    \begin{macrocode}
\cs_new:Nn \um_assign_delcode_noparse:nn {
  \um_set_delcode:nnn \um_symfont_tl {#1} {#2}
}
\cs_new:Nn \um_assign_delcode_parse:nn {
  \um_if_char_spec:nNNT {#2}{\@nil}{\@nil} {
    \um_assign_delcode_noparse:nn {#1} {#2}
  }
}
%    \end{macrocode}
% \end{macro}
%
%
% \begin{macro}{\um_assign_delcode:n}
% Shorthand.
%    \begin{macrocode}
\cs_new:Nn \um_assign_delcode:n { \um_assign_delcode:nn {#1} {#1} }
%    \end{macrocode}
% \end{macro}
%
%
%
% Some symbols that aren't mathopen/mathclose still need to have delimiter codes assigned.
% The list of vertical arrows may be incomplete.
% On the other hand, many fonts won't support them all being stretchy.
% And some of them are probably not meant to stretch, either. But adding them here doesn't hurt.
% \begin{macro}{\um_setup_delcodes:}
%    \begin{macrocode}
\cs_new:Npn \um_setup_delcodes: {
  \um_assign_delcode:nn {`\.} {\c_zero} % ensure \left. and \right. work
  \um_assign_delcode:nn {`\/}   {\g_um_slash_delimiter_usv}
  \um_assign_delcode:nn {"2044} {\g_um_slash_delimiter_usv} % fracslash
  \um_assign_delcode:nn {"2215} {\g_um_slash_delimiter_usv} % divslash
  \um_assign_delcode:n {"005C} % backslash
  \um_assign_delcode:nn {`\<} {"27E8} % angle brackets with ascii notation
  \um_assign_delcode:nn {`\>} {"27E9} % angle brackets with ascii notation
  \um_assign_delcode:n {"2191} % up arrow
  \um_assign_delcode:n {"2193} % down arrow
  \um_assign_delcode:n {"2195} % updown arrow
  \um_assign_delcode:n {"219F} % up arrow twohead
  \um_assign_delcode:n {"21A1} % down arrow twohead
  \um_assign_delcode:n {"21A5} % up arrow from bar
  \um_assign_delcode:n {"21A7} % down arrow from bar
  \um_assign_delcode:n {"21A8} % updown arrow from bar
  \um_assign_delcode:n {"21BE} % up harpoon right
  \um_assign_delcode:n {"21BF} % up harpoon left
  \um_assign_delcode:n {"21C2} % down harpoon right
  \um_assign_delcode:n {"21C3} % down harpoon left
  \um_assign_delcode:n {"21C5} % arrows up down
  \um_assign_delcode:n {"21F5} % arrows down up
  \um_assign_delcode:n {"21C8} % arrows up up
  \um_assign_delcode:n {"21CA} % arrows down down
  \um_assign_delcode:n {"21D1} % double up arrow
  \um_assign_delcode:n {"21D3} % double down arrow
  \um_assign_delcode:n {"21D5} % double updown arrow
  \um_assign_delcode:n {"21DE} % up arrow double stroke
  \um_assign_delcode:n {"21DF} % down arrow double stroke
  \um_assign_delcode:n {"21E1} % up arrow dashed
  \um_assign_delcode:n {"21E3} % down arrow dashed
  \um_assign_delcode:n {"21E7} % up white arrow
  \um_assign_delcode:n {"21E9} % down white arrow
  \um_assign_delcode:n {"21EA} % up white arrow from bar
  \um_assign_delcode:n {"21F3} % updown white arrow
}
%    \end{macrocode}
% \end{macro}
%
%
%
%
% \subsection{(Big) operators}
%
% Turns out that \XeTeX\ is clever enough to deal with big operators for us
% automatically with \cmd\Umathchardef. Amazing!
%
% However, the limits aren't set automatically; that is, we want to define,
% a la Plain \TeX\ \etc, |\def\int{\intop\nolimits}|, so there needs to be a
% transformation from \cmd\int\ to \cmd\intop\ during the expansion of
% \cmd\_um_sym:nnn\ in the appropriate contexts.
%
% \begin{macro}{\l_um_nolimits_tl}
% This macro is a sequence containing those maths operators that require a
% \cmd\nolimits\ suffix.
% This list is used when processing |unicode-math-table.tex| to define such
% commands automatically (see the macro \cs{um_set_mathsymbol:nNNn}).
% I've chosen essentially just the operators that look like integrals;
% hopefully a better mathematician can help me out here.
% I've a feeling that it's more useful \emph{not} to include the multiple
% integrals such as $\iiiint$, but that might be a matter of preference.
%    \begin{macrocode}
\tl_new:N \l_um_nolimits_tl
\tl_set:Nn \l_um_nolimits_tl {
  \int\iint\iiint\iiiint\oint\oiint\oiiint
  \intclockwise\varointclockwise\ointctrclockwise\sumint
  \intbar\intBar\fint\cirfnint\awint\rppolint
  \scpolint\npolint\pointint\sqint\intlarhk\intx
  \intcap\intcup\upint\lowint
}
%    \end{macrocode}
% \end{macro}
%
% \begin{macro}{\addnolimits}
% This macro appends material to the macro containing the list of operators
% that don't take limits.
%    \begin{macrocode}
\DeclareDocumentCommand \addnolimits {m} {
  \tl_put_right:Nn \l_um_nolimits_tl {#1}
}
%    \end{macrocode}
% \end{macro}
%
% \begin{macro}{\removenolimits}
% Can this macro be given a better name?
% It removes an item from the nolimits list.
%    \begin{macrocode}
\DeclareDocumentCommand \removenolimits {m} {
  \tl_remove_all:Nn \l_um_nolimits_tl {#1}
}
%    \end{macrocode}
% \end{macro}
%
% \subsection{Radicals}
%
% The radical for square root is organised in \cs{um_set_mathsymbol:nNNn}.
% I think it's the only radical ever.
% (Actually, there is also \cs{cuberoot} and \cs{fourthroot}, but they don't
%  seem to behave as proper radicals.)
%
% Also, what about right-to-left square roots?
%
% \begin{macro}{\l_um_radicals_tl}
% We organise radicals in the same way as nolimits-operators.
%    \begin{macrocode}
\tl_new:N \l_um_radicals_tl
\tl_set:Nn \l_um_radicals_tl {\sqrt \longdivision}
%    \end{macrocode}
% \end{macro}
%
% \subsection{Maths accents}
%
% Maths accents should just work \emph{if they are available in the font}.
%
% \subsection{Common interface for font parameters}
%
% \XeTeX\ and \LuaTeX\ have different interfaces for math font parameters.
% We use \LuaTeX’s interface because it’s much better, but rename the primitives to be more \LaTeX3-like.
% There are getter and setter commands for each font parameter.
% The names of the parameters is derived from the \LuaTeX\ names, with underscores inserted between words.
% For every parameter \cs{Umath\meta{\LuaTeX\ name}}, we define an expandable getter command \cs{um_\meta{\LaTeX3 name}:N} and a protected setter command \cs{um_set_\meta{\LaTeX3 name}:Nn}.
% The getter command takes one of the style primitives (\cs{displaystyle} etc.)\ and expands to the font parameter, which is a \meta{dimension}.
% The setter command takes a style primitive and a dimension expression, which is parsed with \cs{dim_eval:n}.
%
% Often, the mapping between font dimensions and font parameters is bijective, but there are cases which require special attention:
% \begin{itemize}
% \item Some parameters map to different dimensions in display and non-display styles.
% \item Likewise, one parameter maps to different dimensions in non-cramped and cramped styles.
% \item There are a few parameters for which \XeTeX\ doesn’t seem to provide \cs{fontdimen}s; in this case the getter and setter commands are left undefined.
% \end{itemize}
%
% \paragraph{Cramped style tokens}
% \LuaTeX\ has \cs{crampeddisplaystyle} etc.,\ but they are loaded as \cs{luatexcrampeddisplaystyle} etc.\ by the \pkg{luatextra} package.
% \XeTeX, however, doesn’t have these primitives, and their syntax cannot really be emulated.
% Nevertheless, we define these commands as quarks, so they can be used as arguments to the font parameter commands (but nowhere else).
% Making these commands available is necessary because we need to make a distinction between cramped and non-cramped styles for one font parameter.
%
% \begin{macro}{\um_new_cramped_style:N}
% \darg{command}
% Define \meta{command} as a new cramped style switch.
% For \LuaTeX, simply rename the correspronding primitive.
% For \XeTeX, define \meta{command} as a new quark.
%    \begin{macrocode}
\cs_new_protected_nopar:Nn \um_new_cramped_style:N
%<XE>  { \quark_new:N #1 }
%<LU>  { \cs_new_eq:Nc #1 { luatex \cs_to_str:N #1 } }
%    \end{macrocode}
% \end{macro}
%
% \begin{macro}{\crampeddisplaystyle}
% \begin{macro}{\crampedtextstyle}
% \begin{macro}{\crampedscriptstyle}
% \begin{macro}{\crampedscriptscriptstyle}
% The cramped style commands.
%    \begin{macrocode}
\um_new_cramped_style:N \crampeddisplaystyle
\um_new_cramped_style:N \crampedtextstyle
\um_new_cramped_style:N \crampedscriptstyle
\um_new_cramped_style:N \crampedscriptscriptstyle
%    \end{macrocode}
% \end{macro}
% \end{macro}
% \end{macro}
% \end{macro}
%
% \paragraph{Font dimension mapping}
% Font parameters may differ between the styles.
% \LuaTeX\ accounts for this by having the parameter primitives take a style token argument.
% To replicate this behavior in \XeTeX, we have to map style tokens to specific combinations of font dimension numbers and math fonts (\cs{textfont} etc.).
%
% \begin{macro}{\um_font_dimen:Nnnnn}
% \darg{style token}
% \darg{font dimen for display style}
% \darg{font dimen for cramped display style}
% \darg{font dimen for non-display styles}
% \darg{font dimen for cramped non-display styles}
% Map math style to \XeTeX\ math font dimension.
% \meta{style token} must be one of the style switches (\cs{displaystyle}, \cs{crampeddisplaystyle}, \dots).
% The other parameters are integer constants referring to font dimension numbers.
% The macro expands to a dimension which contains the appropriate font dimension.
%    \begin{macrocode}
%<*XE>
  \cs_new_nopar:Npn \um_font_dimen:Nnnnn #1 #2 #3 #4 #5 {
    \fontdimen
    \cs_if_eq:NNTF #1 \displaystyle {
      #2 \textfont
    } {
      \cs_if_eq:NNTF #1 \crampeddisplaystyle {
        #3 \textfont
      } {
        \cs_if_eq:NNTF #1 \textstyle {
          #4 \textfont
        } {
          \cs_if_eq:NNTF #1 \crampedtextstyle {
            #5 \textfont
          } {
            \cs_if_eq:NNTF #1 \scriptstyle {
              #4 \scriptfont
            } {
              \cs_if_eq:NNTF #1 \crampedscriptstyle {
                #5 \scriptfont
              } {
                \cs_if_eq:NNTF #1 \scriptscriptstyle {
                  #4 \scriptscriptfont
                } {
%    \end{macrocode}
% Should we check here if the style is invalid?
%    \begin{macrocode}
                  #5 \scriptscriptfont
                }
              }
            }
          }
        }
      }
    }
%    \end{macrocode}
% Which family to use?
%    \begin{macrocode}
    \c_two
  }
%</XE>
%    \end{macrocode}
% \end{macro}
%
% \paragraph{Font parameters}
% This paragraph contains macros for defining the font parameter interface, as well as the definition for all font parameters known to \LuaTeX.
%
% \begin{macro}{\um_font_param:nnnnn}
% \darg{name}
% \darg{font dimension for non-cramped display style}
% \darg{font dimension for cramped display style}
% \darg{font dimension for non-cramped non-display styles}
% \darg{font dimension for cramped non-display styles}
% This macro defines getter and setter functions for the font parameter \meta{name}.
% The \LuaTeX\ font parameter name is produced by removing all underscores and prefixing the result with |luatexUmath|.
% The \XeTeX\ font dimension numbers must be integer constants.
%    \begin{macrocode}
\cs_new_protected_nopar:Nn \um_font_param:nnnnn
%<*XE>
{
  \um_font_param_aux:ccnnnn { um_ #1 :N } { um_set_ #1 :N }
    { #2 } { #3 } { #4 } { #5 }
}
%</XE>
%<*LU>
{
  \tl_set:Nn \l_um_tmpa_tl { #1 }
  \tl_remove_all:Nn \l_um_tmpa_tl { _ }
  \um_font_param_aux:ccc { um_ #1 :N } { um_set_ #1 :N }
    { luatexUmath \l_um_tmpa_tl }
}
%</LU>
%    \end{macrocode}
% \end{macro}
%
% \begin{macro}{\um_font_param:nnn}
% \darg{name}
% \darg{font dimension for display style}
% \darg{font dimension for non-display styles}
% This macro defines getter and setter functions for the font parameter \meta{name}.
% The \LuaTeX\ font parameter name is produced by removing all underscores and prefixing the result with |luatexUmath|.
% The \XeTeX\ font dimension numbers must be integer constants.
%    \begin{macrocode}
\cs_new_protected_nopar:Npn \um_font_param:nnn #1 #2 #3 {
  \um_font_param:nnnnn { #1 } { #2 } { #2 } { #3 } { #3 }
}
%    \end{macrocode}
% \end{macro}
%
% \begin{macro}{\um_font_param:nn}
% \darg{name}
% \darg{font dimension}
% This macro defines getter and setter functions for the font parameter \meta{name}.
% The \LuaTeX\ font parameter name is produced by removing all underscores and prefixing the result with |luatexUmath|.
% The \XeTeX\ font dimension number must be an integer constant.
%    \begin{macrocode}
\cs_new_protected_nopar:Npn \um_font_param:nn #1 #2 {
  \um_font_param:nnnnn { #1 } { #2 } { #2 } { #2 } { #2 }
}
%    \end{macrocode}
% \end{macro}
%
% \begin{macro}{\um_font_param:n}
% \darg{name}
% This macro defines getter and setter functions for the font parameter \meta{name}, which is considered unavailable in \XeTeX\@.
% The \LuaTeX\ font parameter name is produced by removing all underscores and prefixing the result with |luatexUmath|.
%    \begin{macrocode}
\cs_new_protected_nopar:Nn \um_font_param:n
%<XE>  { }
%<LU>  { \um_font_param:nnnnn { #1 } { 0 } { 0 } { 0 } { 0 } }
%    \end{macrocode}
% \end{macro}
%
% \begin{macro}{\um_font_param_aux:NNnnnn}
% \begin{macro}{\um_font_param_aux:NNN}
% Auxiliary macros for generating font parameter accessor macros.
%    \begin{macrocode}
%<*XE>
\cs_new_protected_nopar:Nn \um_font_param_aux:NNnnnn
  {
    \cs_new_nopar:Npn #1 ##1 {
      \um_font_dimen:Nnnnn ##1 { #3 } { #4 } { #5 } { #6 }
    }
    \cs_new_protected_nopar:Npn #2 ##1 ##2 {
      #1 ##1 \dim_eval:n { ##2 }
    }
  }
\cs_generate_variant:Nn \um_font_param_aux:NNnnnn { cc }
%</XE>
%<*LU>
\cs_new_protected_nopar:Nn \um_font_param_aux:NNN
  {
    \cs_new_nopar:Npn #1 ##1 {
      #3 ##1
    }
    \cs_new_protected_nopar:Npn #2 ##1 ##2 {
      #3 ##1 \dim_eval:n { ##2 }
    }
  }
\cs_generate_variant:Nn \um_font_param_aux:NNN { ccc }
%</LU>
%    \end{macrocode}
% \end{macro}
% \end{macro}
%
% Now all font parameters that are listed in the \LuaTeX\ reference follow.
%    \begin{macrocode}
\um_font_param:nn { axis } { 15 }
\um_font_param:nn { operator_size } { 13 }
\um_font_param:n { fraction_del_size }
\um_font_param:nnn { fraction_denom_down } { 45 } { 44 }
\um_font_param:nnn { fraction_denom_vgap } { 50 } { 49 }
\um_font_param:nnn { fraction_num_up } { 43 } { 42 }
\um_font_param:nnn { fraction_num_vgap } { 47 } { 46 }
\um_font_param:nn { fraction_rule } { 48 }
\um_font_param:nn { limit_above_bgap } { 29 }
\um_font_param:n { limit_above_kern }
\um_font_param:nn { limit_above_vgap } { 28 }
\um_font_param:nn { limit_below_bgap } { 31 }
\um_font_param:n { limit_below_kern }
\um_font_param:nn { limit_below_vgap } { 30 }
\um_font_param:nn { over_delimiter_vgap } { 41 }
\um_font_param:nn { over_delimiter_bgap } { 38 }
\um_font_param:nn { under_delimiter_vgap } { 40 }
\um_font_param:nn { under_delimiter_bgap } { 39 }
\um_font_param:nn { overbar_kern } { 55 }
\um_font_param:nn { overbar_rule } { 54 }
\um_font_param:nn { overbar_vgap } { 53 }
\um_font_param:n { quad }
\um_font_param:nn { radical_kern } { 62 }
\um_font_param:nn { radical_rule } { 61 }
\um_font_param:nnn { radical_vgap } { 60 } { 59 }
\um_font_param:nn { radical_degree_before } { 63 }
\um_font_param:nn { radical_degree_after } { 64 }
\um_font_param:nn { radical_degree_raise } { 65 }
\um_font_param:nn { space_after_script } { 27 }
\um_font_param:nnn { stack_denom_down } { 35 } { 34 }
\um_font_param:nnn { stack_num_up } { 33 } { 32 }
\um_font_param:nnn { stack_vgap } { 37 } { 36 }
\um_font_param:nn { sub_shift_down } { 18 }
\um_font_param:nn { sub_shift_drop } { 20 }
\um_font_param:n { subsup_shift_down }
\um_font_param:nn { sub_top_max } { 19 }
\um_font_param:nn { subsup_vgap } { 25 }
\um_font_param:nn { sup_bottom_min } { 23 }
\um_font_param:nn { sup_shift_drop } { 24 }
\um_font_param:nnnnn { sup_shift_up } { 21 } { 22 } { 21 } { 22 }
\um_font_param:nn { supsub_bottom_max } { 26 }
\um_font_param:nn { underbar_kern } { 58 }
\um_font_param:nn { underbar_rule } { 57 }
\um_font_param:nn { underbar_vgap } { 56 }
\um_font_param:n { connector_overlap_min }
%    \end{macrocode}
%
% \section{Font features}
%
% \begin{macro}{\new@mathversion}
% Fix bug in the \LaTeX\ version.
% (Fixed upstream, too, but unsure when that will propagate.)
%    \begin{macrocode}
\def\new@mathversion#1{%
  \expandafter\in@\expandafter#1\expandafter{\version@list}%
  \ifin@
    \@font@info{Redeclaring math version
               `\expandafter\@gobblefour\string#1'}%
  \else
    \expandafter\newcount\csname c@\expandafter
                                \@gobble\string#1\endcsname
    \def\version@elt{\noexpand\version@elt\noexpand}%
    \edef\version@list{\version@list\version@elt#1}%
  \fi
  \toks@{}%
  \count@\z@
  \def\group@elt##1##2{%
       \advance\count@\@ne
       \addto@hook\toks@{\getanddefine@fonts##1##2}%
       }%
  \group@list
  \global\csname c@\expandafter\@gobble\string#1\endcsname\count@
  \def\alpha@elt##1##2##3{%
       \ifx##2\no@alphabet@error
         \toks@\expandafter{\the\toks@\install@mathalphabet##1%
             {\no@alphabet@error##1}}%
       \else
         \toks@\expandafter{\the\toks@\install@mathalphabet##1%
             {\select@group##1##2##3}}%
       \fi
          }%
  \alpha@list
  \xdef#1{\the\toks@}%
}
%    \end{macrocode}
% \end{macro}
%
% \subsection{Math version}
%    \begin{macrocode}
\keys_define:nn {unicode-math}
  {
    version .code:n =
      {
        \tl_set:Nn \l_um_mversion_tl {#1}
        \DeclareMathVersion{\l_um_mversion_tl}
      }
  }
%    \end{macrocode}
%
% \subsection{Script and scriptscript font options}
%    \begin{macrocode}
\keys_define:nn {unicode-math}
{
  script-features  .tl_set:N =  \l_um_script_features_tl ,
  sscript-features .tl_set:N = \l_um_sscript_features_tl ,
       script-font .tl_set:N =      \l_um_script_font_tl ,
      sscript-font .tl_set:N =     \l_um_sscript_font_tl ,
}
%    \end{macrocode}
%
% \subsection{Range processing}
% \seclabel{rangeproc}
%
%    \begin{macrocode}
\seq_new:N \l_um_mathalph_seq
\seq_new:N \l_um_char_range_seq
\seq_new:N \l_um_mclass_range_seq
\seq_new:N \l_um_cmd_range_seq
\keys_define:nn {unicode-math} {
  range .code:n = {
    \bool_set_false:N \l_um_init_bool
%    \end{macrocode}
% Set processing functions if we're not defining the full Unicode math repetoire.
% Math symbols are defined with \cmd\_um_sym:nnn; see \secref{mathsymbol}
% for the individual definitions
%    \begin{macrocode}
    \int_incr:N \g_um_fam_int
    \tl_set:Nx \um_symfont_tl {um_fam\int_use:N\g_um_fam_int}
    \cs_set_eq:NN \_um_sym:nnn \um_process_symbol_parse:nnn
    \cs_set_eq:NN \um_set_mathalphabet_char:Nnn \um_mathmap_parse:Nnn
    \cs_set_eq:NN \um_remap_symbol:nnn \um_remap_symbol_parse:nnn
    \cs_set_eq:NN \um_maybe_init_alphabet:n \use_none:n
    \cs_set_eq:NN \um_map_char_single:nn \um_map_char_parse:nn
    \cs_set_eq:NN \um_assign_delcode:nn \um_assign_delcode_parse:nn
    \cs_set_eq:NN \um_make_mathactive:nNN \um_make_mathactive_parse:nNN
%    \end{macrocode}
% Proceed by filling up the various `range' seqs according to the user options.
%    \begin{macrocode}
    \seq_clear:N \l_um_char_range_seq
    \seq_clear:N \l_um_mclass_range_seq
    \seq_clear:N \l_um_cmd_range_seq
    \seq_clear:N \l_um_mathalph_seq
    \clist_map_inline:nn {#1} {
      \um_if_mathalph_decl:nTF {##1} {
        \seq_put_right:Nx \l_um_mathalph_seq {
          { \exp_not:V \l_um_tmpa_tl }
          { \exp_not:V \l_um_tmpb_tl }
          { \exp_not:V \l_um_tmpc_tl }
        }
      }{
%    \end{macrocode}
% Four cases:
% math class matching the known list;
% single item that is a control sequence---command name;
% single item that isn't---edge case, must be 0--9;
% none of the above---char range.
%    \begin{macrocode}
        \seq_if_in:NnTF \g_um_mathclasses_seq {##1}
          { \seq_put_right:Nn \l_um_mclass_range_seq {##1} }
          {
            \bool_if:nTF { \tl_if_single_p:n {##1} && \token_if_cs_p:N ##1 }
              { \seq_put_right:Nn \l_um_cmd_range_seq {##1} }
              { \seq_put_right:Nn \l_um_char_range_seq {##1} }
          }
      }
    }
  }
}
\seq_new:N \g_um_mathclasses_seq
\seq_set_from_clist:Nn \g_um_mathclasses_seq
  {
    \mathord,\mathalpha,\mathop,\mathbin,\mathrel,
    \mathopen,\mathclose,\mathpunct,\mathaccent,
    \mathfence,\mathover,\mathunder,\mathbotaccent
  }
%    \end{macrocode}
%
%
% \begin{macro}{\um_if_mathalph_decl:nTF}
% Possible forms of input:\\
% |\mathscr|\\
% |\mathscr->\mathup|\\
% |\mathscr/{Latin}|\\
% |\mathscr/{Latin}->\mathup|\\
% Outputs:\\
% |tmpa|: math style (\eg, |\mathscr|)\\
% |tmpb|: alphabets (\eg, |Latin|)\\
% |tmpc|: remap style (\eg, |\mathup|). Defaults to |tmpa|.
%
% The remap style can also be |\mathcal->stixcal|, which I marginally prefer
% in the general case.
%    \begin{macrocode}
\prg_new_conditional:Nnn \um_if_mathalph_decl:n {TF} {
  \tl_set:Nx \l_um_tmpa_tl { \tl_trim_spaces:n {#1} }
  \tl_clear:N \l_um_tmpb_tl
  \tl_clear:N \l_um_tmpc_tl
  \tl_if_in:NnT \l_um_tmpa_tl {->} {
    \exp_after:wN \um_split_arrow:w \l_um_tmpa_tl \q_nil
  }
  \tl_if_in:NnT \l_um_tmpa_tl {/} {
    \exp_after:wN \um_split_slash:w \l_um_tmpa_tl \q_nil
  }
  \tl_if_empty:NT \l_um_tmpc_tl { \tl_set_eq:NN \l_um_tmpc_tl \l_um_tmpa_tl }
  \seq_if_in:NVTF \g_um_mathstyles_seq \l_um_tmpa_tl {
    \prg_return_true:
  }{
    \prg_return_false:
  }
}
\cs_set:Npn \um_split_arrow:w #1->#2 \q_nil {
  \tl_set:Nn \l_um_tmpa_tl {#1}
  \tl_if_single:nTF {#2}
    { \tl_set:Nn \l_um_tmpc_tl {#2} }
    { \exp_args:NNc \tl_set:Nn \l_um_tmpc_tl {math#2} }
}
\cs_set:Npn \um_split_slash:w #1/#2 \q_nil {
  \tl_set:Nn \l_um_tmpa_tl {#1}
  \tl_set:Nn \l_um_tmpb_tl {#2}
}
%    \end{macrocode}
% \end{macro}
%
% Pretty basic comma separated range processing.
% Donald Arseneau's \pkg{selectp} package has a cleverer technique.
%
% \begin{macro}{\um_if_char_spec:nNNT}
% \darg{Unicode character slot}
% \darg{control sequence (character macro)}
% \darg{control sequence (math class)}
% \darg{code to execute}
% This macro expands to |#4|
% if any of its arguments are contained in \cmd\l_um_char_range_seq.
% This list can contain either character ranges (for checking with |#1|) or control sequences.
% These latter can either be the command name of a specific character, \emph{or} the math
% type of one (\eg, \cmd\mathbin).
%
% Character ranges are passed to \cmd\um@parse@range, which accepts input in the form shown in \tabref{ranges}.
%
% \begin{table}[htbp]
% \centering
% \topcaption{Ranges accepted by \cmd\um@parse@range.}
% \label{tab:ranges}
% \begin{tabular}{>{\ttfamily}cc}
% \textrm{Input} & Range \\
% \hline
% x & $r=x$ \\
% x- & $r\geq x$ \\
% -y & $r\leq y$ \\
% x-y & $x \leq r \leq y$ \\
% \end{tabular}
% \end{table}
%
% We have three tests, performed sequentially in order of execution time.
% Any test finding a match jumps directly to the end.
%    \begin{macrocode}
\cs_new:Nn \um_if_char_spec:nNNT
  {

    % math class:
    \seq_if_in:NnT \l_um_mclass_range_seq {#3}
      { \use_none_delimit_by_q_nil:w }

    % command name:
    \seq_if_in:NnT \l_um_cmd_range_seq {#2}
      { \use_none_delimit_by_q_nil:w }

    % character slot:
    \seq_map_inline:Nn \l_um_char_range_seq
      {
        \um_int_if_slot_in_range:nnT {#1} {##1}
          { \seq_map_break:n { \use_none_delimit_by_q_nil:w } }
      }

    % this executes if no match was found:
    \use_none:nnn
    \q_nil
    \use:n
      {
        \clist_put_right:Nx \l_um_char_num_range_clist { \int_eval:n {#1} }
        #4
      }
  }
%    \end{macrocode}
% \end{macro}
%
% \begin{macro}{\um_int_if_slot_in_range:nnT}
% A `numrange' is like |-2,5-8,12,17-| (can be unsorted).
%
% Four cases, four argument types:
% \begin{Verbatim}
% input    #2     #3      #4
% "1  "   [ 1] - [qn] - [   ] qs
% "1- "   [ 1] - [  ] - [qn-] qs
% " -3"   [  ] - [ 3] - [qn-] qs
% "1-3"   [ 1] - [ 3] - [qn-] qs
% \end{Verbatim}
%
%    \begin{macrocode}
\cs_new:Nn \um_int_if_slot_in_range:nnT
  { \um_numrange_parse:nwT {#1} #2 - \q_nil - \q_stop {#3} }
%    \end{macrocode}
%
%    \begin{macrocode}
\cs_set:Npn \um_numrange_parse:nwT #1 #2 - #3 - #4 \q_stop #5
  {
    \tl_if_empty:nTF {#4} { \int_compare:nT {#1=#2} {#5} }
      {
    \tl_if_empty:nTF {#3} { \int_compare:nT {#1>=#2} {#5} }
      {
    \tl_if_empty:nTF {#2} { \int_compare:nT {#1<=#3} {#5} }
      {
    \int_compare:nT {#1>=#2} { \int_compare:nT {#1<=#3} {#5} }
      } } }
  }
%    \end{macrocode}
% \end{macro}
%
%
% \subsection{Resolving Greek symbol name control sequences}
%
% \begin{macro}{\um_resolve_greek:}
% This macro defines \cmd\Alpha\dots\cmd\omega\ as their corresponding
% Unicode (mathematical italic) character. Remember that the mapping
% to upright or italic happens with the mathcode definitions, whereas these macros
% just stand for the literal Unicode characters.
%    \begin{macrocode}
\AtBeginDocument{\um_resolve_greek:}
\cs_new:Npn \um_resolve_greek: {
  \clist_map_inline:nn {
    Alpha,Beta,Gamma,Delta,Epsilon,Zeta,Eta,Theta,Iota,Kappa,Lambda,
    alpha,beta,gamma,delta,        zeta,eta,theta,iota,kappa,lambda,
    Mu,Nu,Xi,Omicron,Pi,Rho,Sigma,Tau,Upsilon,Phi,Chi,Psi,Omega,
    mu,nu,xi,omicron,pi,rho,sigma,tau,upsilon,    chi,psi,omega,
    varTheta,
    varsigma,vartheta,varkappa,varrho,varpi
  }{
    \tl_set:cx {##1} { \exp_not:c { mit ##1 } }
  }
  \tl_set:Nn \epsilon {
    \bool_if:NTF \g_um_texgreek_bool \mitvarepsilon \mitepsilon
  }
  \tl_set:Nn \phi {
    \bool_if:NTF \g_um_texgreek_bool \mitvarphi \mitphi
  }
  \tl_set:Nn \varepsilon {
    \bool_if:NTF \g_um_texgreek_bool \mitepsilon \mitvarepsilon
  }
  \tl_set:Nn \varphi {
    \bool_if:NTF \g_um_texgreek_bool \mitphi \mitvarphi
  }
}
%    \end{macrocode}
% \end{macro}
%
%
% \section{Maths alphabets mapping definitions}
% \label{part:mathmap}
%
% Algorithm for setting alphabet fonts.
% By default, when |range| is empty, we are in \emph{implicit} mode.
% If |range| contains the name of the math alphabet, we are in \emph{explicit}
% mode and do things slightly differently.
%
% Implicit mode:
% \begin{itemize}
% \item Try and set all of the alphabet shapes.
% \item Check for the first glyph of each alphabet to detect if the font supports each
%       alphabet shape.
% \item For alphabets that do exist, overwrite whatever's already there.
% \item For alphabets that are not supported, \emph{do nothing}.
%       (This includes leaving the old alphabet definition in place.)
% \end{itemize}
%
% Explicit mode:
% \begin{itemize}
% \item Only set the alphabets specified.
% \item Check for the first glyph of the alphabet to detect if the font contains
%       the alphabet shape in the Unicode math plane.
% \item For Unicode math alphabets, overwrite whatever's already there.
% \item Otherwise, use the \ascii\ letters instead.
% \end{itemize}
%
% \subsection{Initialising math styles}
%
% \begin{macro}{\um_new_mathstyle:N}
% This function defines a new command like \cs{mathfrak}.
%    \begin{macrocode}
\cs_new:Nn \um_new_mathstyle:N {
  \um_prepare_mathstyle:f {\exp_after:wN \use_none:nnnnn \token_to_str:N #1}
  \seq_put_right:Nn \g_um_mathstyles_seq {#1}
}
%    \end{macrocode}
% \end{macro}
%
% \begin{macro}{\g_um_default_mathalph_seq}
% This sequence stores the alphabets in each math style.
%    \begin{macrocode}
\seq_new:N \g_um_default_mathalph_seq
%    \end{macrocode}
% \end{macro}
%
% \begin{macro}{\g_um_mathstyles_seq}
% This is every math style known to \pkg{unicode-math}.
%    \begin{macrocode}
\seq_new:N \g_um_mathstyles_seq
%    \end{macrocode}
% \end{macro}
%
%    \begin{macrocode}
\AtEndOfPackage{
\clist_map_inline:nn {
  {\mathup    } {latin,Latin,greek,Greek,num,misc} {\mathup    }  ,
  {\mathit    } {latin,Latin,greek,Greek,misc}     {\mathit    }  ,
  {\mathbb    } {latin,Latin,num,misc}             {\mathbb    }  ,
  {\mathbbit  } {misc}                             {\mathbbit  }  ,
  {\mathscr   } {latin,Latin}                      {\mathscr   }  ,
  {\mathcal   } {Latin}                            {\mathscr   }  ,
  {\mathbfcal } {Latin}                            {\mathbfscr }  ,
  {\mathfrak  } {latin,Latin}                      {\mathfrak  }  ,
  {\mathtt    } {latin,Latin,num}                  {\mathtt    }  ,
  {\mathsfup  } {latin,Latin,num}                  {\mathsfup  }  ,
  {\mathsfit  } {latin,Latin}                      {\mathsfit  }  ,
  {\mathbfup  } {latin,Latin,greek,Greek,num,misc} {\mathbfup  }  ,
  {\mathbfit  } {latin,Latin,greek,Greek,misc}     {\mathbfit  }  ,
  {\mathbfscr } {latin,Latin}                      {\mathbfscr }  ,
  {\mathbffrak} {latin,Latin}                      {\mathbffrak}  ,
  {\mathbfsfup} {latin,Latin,greek,Greek,num,misc} {\mathbfsfup}  ,
  {\mathbfsfit} {latin,Latin,greek,Greek,misc}     {\mathbfsfit}
}{
  \seq_put_right:Nn \g_um_default_mathalph_seq {#1}
  \exp_after:wN \um_new_mathstyle:N \use_i:nnn #1
}
%    \end{macrocode}
% These are `false' mathstyles that inherit other definitions:
%    \begin{macrocode}
\um_new_mathstyle:N \mathsf
\um_new_mathstyle:N \mathbf
\um_new_mathstyle:N \mathbfsf
%    \end{macrocode}
%    \begin{macrocode}
}
%    \end{macrocode}
%
%
% \subsection{Defining the math style macros}
%
% We call the different shapes that a math alphabet can be a `math style'.
% Note that different alphabets can exist within the same math style. E.g.,
% we call `bold' the math style |bf| and within it there are upper and lower
% case Greek and Roman alphabets and Arabic numerals.
%
% \begin{macro}{\um_prepare_mathstyle:n}
% \darg{math style name (e.g., \texttt{it} or \texttt{bb})}
% Define the high level math alphabet macros (\cs{mathit}, etc.) in terms of
% unicode-math definitions. Use \cs{bgroup}/\cs{egroup} so s'scripts scan the
% whole thing.
%
% The flag \cs{l_um_mathstyle_tl} is for other applications to query the
% current math style.
%    \begin{macrocode}
\cs_new:Nn \um_prepare_mathstyle:n {
  \um_init_alphabet:x {#1}
  \cs_set:cpn {_um_math#1_aux:n} ##1 {
    \use:c {um_switchto_math#1:} ##1 \egroup
  }
  \cs_set_protected:cpx {math#1} {
    \exp_not:n{
      \bgroup
      \mode_if_math:F
        {
          \egroup\expandafter
          \non@alpherr\expandafter{\csname math#1\endcsname\space}
        }
      \tl_set:Nn \l_um_mathstyle_tl {#1}
    }
    \exp_not:c {_um_math#1_aux:n}
  }
}
\tl_new:N \l_um_mathstyle_tl
\cs_generate_variant:Nn \um_prepare_mathstyle:n {f}
%    \end{macrocode}
% \end{macro}
%
% \begin{macro}{\um_init_alphabet:n}
% \darg{math alphabet name (e.g., \texttt{it} or \texttt{bb})}
% This macro initialises the macros used to set up a math alphabet.
% First used with the math alphabet macro is first defined, but then used
% later when redefining a particular maths alphabet.
%    \begin{macrocode}
\cs_set:Npn \um_init_alphabet:n #1 {
  \um_log:nx {alph-initialise} {#1}
  \cs_set_eq:cN {um_switchto_math#1:} \prg_do_nothing:
}
\cs_generate_variant:Nn \um_init_alphabet:n {x}
%    \end{macrocode}
% Variants (cannot use |\cs_generate_variant:Nn| because the base function is
% defined dynamically.)
%    \begin{macrocode}
\cs_new:Npn \um_maybe_init_alphabet:V {
  \exp_args:NV \um_maybe_init_alphabet:n
}
%    \end{macrocode}
% \end{macro}
%
% \subsection{Defining the math alphabets per style}
%
%
% Variables:
%    \begin{macrocode}
\seq_new:N \l_um_missing_alph_seq
%    \end{macrocode}
%
% \begin{macro}{\um_setup_alphabets:}
% This function is called within \cs{setmathfont} to configure the
% mapping between characters inside math styles.
%    \begin{macrocode}
\cs_new:Npn \um_setup_alphabets: {
%    \end{macrocode}
% If |range=| has been used to configure styles, those choices will be in
% |\l_um_mathalph_seq|. If not, set up the styles implicitly:
%    \begin{macrocode}
  \seq_if_empty:NTF \l_um_mathalph_seq {
    \um_log:n {setup-implicit}
    \seq_set_eq:NN \l_um_mathalph_seq \g_um_default_mathalph_seq
    \bool_set_true:N \l_um_implicit_alph_bool
    \um_maybe_init_alphabet:n  {sf}
    \um_maybe_init_alphabet:n  {bf}
    \um_maybe_init_alphabet:n  {bfsf}
  }
%    \end{macrocode}
% If |range=| has been used then we're in explicit mode:
%    \begin{macrocode}
  {
    \um_log:n {setup-explicit}
    \bool_set_false:N \l_um_implicit_alph_bool
    \cs_set_eq:NN \um_set_mathalphabet_char:Nnn \um_mathmap_noparse:Nnn
    \cs_set_eq:NN \um_map_char_single:nn \um_map_char_noparse:nn
  }
%    \end{macrocode}
% Now perform the mapping:
%    \begin{macrocode}
  \seq_map_inline:Nn \l_um_mathalph_seq {
    \tl_set:No \l_um_tmpa_tl { \use_i:nnn   ##1 }
    \tl_set:No \l_um_tmpb_tl { \use_ii:nnn  ##1 }
    \tl_set:No \l_um_remap_style_tl { \use_iii:nnn ##1 }
    \tl_set:Nx \l_um_remap_style_tl {
      \exp_after:wN \exp_after:wN \exp_after:wN \use_none:nnnnn
      \exp_after:wN \token_to_str:N \l_um_remap_style_tl
    }
    \tl_if_empty:NT \l_um_tmpb_tl {
      \cs_set_eq:NN \um_maybe_init_alphabet:n \um_init_alphabet:n
      \tl_set:Nn \l_um_tmpb_tl { latin,Latin,greek,Greek,num,misc }
    }
    \um_setup_math_alphabet:VVV
      \l_um_tmpa_tl \l_um_tmpb_tl \l_um_remap_style_tl
  }
  \seq_if_empty:NF \l_um_missing_alph_seq { \um_log:n { missing-alphabets } }
}
%    \end{macrocode}
% \end{macro}
%
% \begin{macro}{\um_setup_math_alphabet:Nnn}
% \darg{Math font style command (e.g., \cs{mathbb})}
% \darg{Math alphabets, comma separated of \{latin,Latin,greek,Greek,num\}}
% \darg{Name of the output math style (usually same as input \texttt{bb})}
%    \begin{macrocode}
\cs_new:Nn \um_setup_math_alphabet:Nnn {
  \tl_set:Nx \l_um_style_tl {
    \exp_after:wN \use_none:nnnnn \token_to_str:N #1
  }
%    \end{macrocode}
% First check that at least one of the alphabets for the font shape is defined\dots
%    \begin{macrocode}
  \clist_map_inline:nn {#2} {
    \tl_set:Nx \l_um_tmpa_tl { \tl_trim_spaces:n {##1} }
    \cs_if_exist:cT {um_config_ \l_um_style_tl _\l_um_tmpa_tl :n} {
      \str_if_eq:xxTF {\l_um_tmpa_tl}{misc} {
        \um_maybe_init_alphabet:V \l_um_style_tl
        \clist_map_break:
      }{
        \um_glyph_if_exist:cT { \um_to_usv:nn {#3}{\l_um_tmpa_tl} }{
          \um_maybe_init_alphabet:V \l_um_style_tl
          \clist_map_break:
        }
      }
    }
  }
%    \end{macrocode}
% \dots and then loop through them defining the individual ranges:
%    \begin{macrocode}
  \clist_map_inline:nn {#2} {
    \tl_set:Nx \l_um_tmpa_tl { \tl_trim_spaces:n {##1} }
    \cs_if_exist:cT {um_config_ \l_um_style_tl _ \l_um_tmpa_tl :n} {
      \str_if_eq:xxTF {\l_um_tmpa_tl}{misc} {
        \um_log:nx {setup-alph} {math \l_um_style_tl~(\l_um_tmpa_tl)}
        \use:c {um_config_ \l_um_style_tl _ \l_um_tmpa_tl :n} {#3}
      }{
        \um_glyph_if_exist:cTF { \um_to_usv:nn {#3}{\l_um_tmpa_tl} } {
          \um_log:nx {setup-alph} {math \l_um_style_tl~(\l_um_tmpa_tl)}
          \use:c {um_config_ \l_um_style_tl _ \l_um_tmpa_tl :n} {#3}
        }{
          \bool_if:NTF \l_um_implicit_alph_bool {
            \seq_put_right:Nx \l_um_missing_alph_seq {
              \@backslashchar math \l_um_style_tl \space
              (\tl_use:c{c_um_math_alphabet_name_ \l_um_tmpa_tl _tl})
            }
          }{
            \use:c {um_config_ \l_um_style_tl _ \l_um_tmpa_tl :n} {up}
          }
        }
      }
    }
  }
}
\cs_generate_variant:Nn \um_setup_math_alphabet:Nnn {VVV}
%    \end{macrocode}
% \end{macro}
%
%
%
% \subsection{Mapping `naked' math characters}
%
% Before we show the definitions of the alphabet mappings using the functions
% |\um_config_\l_um_style_tl_##1:n|, we first want to define some functions
% to be used inside them to actually perform the character mapping.
%
% \subsubsection{Functions}
%
% \begin{macro}{\um_map_char_single:nn}
% Wrapper for |\um_map_char_noparse:nn| or |\um_map_char_parse:nn|
% depending on the context.
% Cannot use |\cs_generate_variant:Nn| because the base function is
% defined dynamically.
%    \begin{macrocode}
\cs_new:Npn \um_map_char_single:cc { \exp_args:Ncc \um_map_char_single:nn }
%    \end{macrocode}
% \end{macro}
%
%
% \begin{macro}{\um_map_char_noparse:nn}
% \begin{macro}{\um_map_char_parse:nn}
%    \begin{macrocode}
\cs_new:Nn \um_map_char_noparse:nn {
  \um_set_mathcode:nnnn {#1}{\mathalpha}{\um_symfont_tl}{#2}
}
%    \end{macrocode}
%
%    \begin{macrocode}
\cs_new:Nn \um_map_char_parse:nn {
  \um_if_char_spec:nNNT {#1} {\@nil} {\mathalpha} {
    \um_map_char_noparse:nn {#1}{#2}
  }
}
%    \end{macrocode}
% \end{macro}
% \end{macro}
%
% \begin{macro}{\um_map_single:nnn}
% \darg{char name (`dotlessi')}
% \darg{from alphabet(s)}
% \darg{to alphabet}
%    \begin{macrocode}
\cs_new:Nn \um_map_char_single:nnn {
  \um_map_char_single:cc { \um_to_usv:nn {#1}{#3} }
                         { \um_to_usv:nn {#2}{#3} }
}
\cs_set:Npn \um_map_single:nnn #1#2#3 {
  \cs_if_exist:cT { \um_to_usv:nn {#3} {#1} }
  {
    \clist_map_inline:nn {#2} {
      \um_map_char_single:nnn {##1} {#3} {#1}
    }
  }
}
%    \end{macrocode}
% \end{macro}
%
%
% \begin{macro}{\um_map_chars_range:nnnn}
% \darg{Number of chars (26)}
% \darg{From style, one or more (it)}
% \darg{To style (up)}
% \darg{Alphabet name (Latin)}
% First the function with numbers:
%    \begin{macrocode}
\cs_set:Npn \um_map_chars_range:nnn #1#2#3 {
  \prg_stepwise_inline:nnnn {0}{1}{#1-1} {
    \um_map_char_single:nn {#2+##1}{#3+##1}
  }
}
\cs_generate_variant:Nn \um_map_chars_range:nnn {ncc}
%    \end{macrocode}
% And the wrapper with names:
%    \begin{macrocode}
\cs_new:Nn \um_map_chars_range:nnnn {
  \um_map_chars_range:ncc {#1} { \um_to_usv:nn {#2}{#4} }
                               { \um_to_usv:nn {#3}{#4} }
}
%    \end{macrocode}
% \end{macro}
%
% \subsubsection{Functions for alphabets}
%
%    \begin{macrocode}
\cs_new:Nn \um_map_chars_Latin:nn {
  \clist_map_inline:nn {#1} {
    \um_map_chars_range:nnnn {26} {##1} {#2} {Latin}
  }
}
%    \end{macrocode}
%
%    \begin{macrocode}
\cs_new:Nn \um_map_chars_latin:nn {
  \clist_map_inline:nn {#1} {
    \um_map_chars_range:nnnn {26} {##1} {#2} {latin}
  }
}
%    \end{macrocode}
%
%    \begin{macrocode}
\cs_new:Nn \um_map_chars_greek:nn {
  \clist_map_inline:nn {#1} {
    \um_map_chars_range:nnnn {25} {##1} {#2} {greek}
    \um_map_char_single:nnn {##1} {#2} {varepsilon}
    \um_map_char_single:nnn {##1} {#2} {vartheta}
    \um_map_char_single:nnn {##1} {#2} {varkappa}
    \um_map_char_single:nnn {##1} {#2} {varphi}
    \um_map_char_single:nnn {##1} {#2} {varrho}
    \um_map_char_single:nnn {##1} {#2} {varpi}
  }
}
%    \end{macrocode}
%
%    \begin{macrocode}
\cs_new:Nn \um_map_chars_Greek:nn {
  \clist_map_inline:nn {#1} {
    \um_map_chars_range:nnnn {25} {##1} {#2} {Greek}
    \um_map_char_single:nnn {##1} {#2} {varTheta}
  }
}
%    \end{macrocode}
%
%    \begin{macrocode}
\cs_new:Nn \um_map_chars_numbers:nn {
  \um_map_chars_range:nnnn {10} {#1} {#2} {num}
}
%    \end{macrocode}
%
%
% \subsection{Mapping chars inside a math style}
%
% \subsubsection{Functions for setting up the maths alphabets}
%
% \begin{macro}{\um_set_mathalphabet_char:Nnn}
% This is a wrapper for either |\um_mathmap_noparse:Nnn| or
% |\um_mathmap_parse:Nnn|, depending on the context.
% Cannot use |\cs_generate_variant:Nn| because the base function is
% defined dynamically.
%    \begin{macrocode}
\cs_new:Npn \um_set_mathalphabet_char:Ncc {
  \exp_args:NNcc \um_set_mathalphabet_char:Nnn
}
%    \end{macrocode}
% \end{macro}
%
% \begin{macro}{\um_mathmap_noparse:Nnn}
% \darg{Maths alphabet, \eg, \cmd\mathbb}
% \darg{Input slot(s), \eg, the slot for `A' (comma separated)}
% \darg{Output slot, \eg, the slot for `$\mathbb{A}$'}
% Adds \cs{um_set_mathcode:nnnn} declarations to the specified maths alphabet's definition.
%    \begin{macrocode}
\cs_new:Nn \um_mathmap_noparse:Nnn {
  \clist_map_inline:nn {#2} {
    \tl_put_right:cx {um_switchto_\cs_to_str:N #1:} {
      \um_set_mathcode:nnnn{##1}{\mathalpha}{\um_symfont_tl}{#3}
    }
  }
}
%    \end{macrocode}
% \end{macro}
%
% \begin{macro}{\um_mathmap_parse:Nnn}
% \darg{Maths alphabet, \eg, \cmd\mathbb}
% \darg{Input slot(s), \eg, the slot for `A' (comma separated)}
% \darg{Output slot, \eg, the slot for `$\mathbb{A}$'}
% When \cmd\um_if_char_spec:nNNT\ is executed, it populates the \cmd\l_um_char_num_range_clist\
% macro with slot numbers corresponding to the specified range. This range is used to
% conditionally add \cs{um_set_mathcode:nnnn} declaractions to the maths alphabet definition.
%    \begin{macrocode}
\cs_new:Nn \um_mathmap_parse:Nnn {
  \clist_if_in:NnT \l_um_char_num_range_clist {#3} {
    \um_mathmap_noparse:Nnn {#1}{#2}{#3}
  }
}
%    \end{macrocode}
% \end{macro}
%
% \begin{macro}{\um_set_mathalphabet_char:Nnnn}
% \darg{math style command}
% \darg{input math alphabet name}
% \darg{output math alphabet name}
% \darg{char name to map}
%    \begin{macrocode}
\cs_new:Npn \um_set_mathalphabet_char:Nnnn #1#2#3#4 {
  \um_set_mathalphabet_char:Ncc #1 { \um_to_usv:nn {#2} {#4} }
                                   { \um_to_usv:nn {#3} {#4} }
}
%    \end{macrocode}
% \end{macro}
%
% \begin{macro}{\um_set_mathalph_range:nNnn}
% \darg{Number of iterations}
% \darg{Maths alphabet}
% \darg{Starting input char (single)}
% \darg{Starting output char}
% Loops through character ranges setting \cmd\mathcode.
% First the version that uses numbers:
%    \begin{macrocode}
\cs_new:Npn \um_set_mathalph_range:nNnn #1#2#3#4 {
  \prg_stepwise_inline:nnnn {0}{1}{#1-1}
    { \um_set_mathalphabet_char:Nnn {#2} { ##1 + #3 } { ##1 + #4 } }
}
\cs_generate_variant:Nn \um_set_mathalph_range:nNnn {nNcc}
%    \end{macrocode}
% Then the wrapper version that uses names:
%    \begin{macrocode}
\cs_new:Npn \um_set_mathalph_range:nNnnn #1#2#3#4#5 {
  \um_set_mathalph_range:nNcc {#1} #2 { \um_to_usv:nn {#3} {#5} }
                                      { \um_to_usv:nn {#4} {#5} }
}
%    \end{macrocode}
% \end{macro}
%
% \subsubsection{Individual mapping functions for different alphabets}
%
%    \begin{macrocode}
\cs_new:Npn \um_set_mathalphabet_pos:Nnnn #1#2#3#4 {
  \cs_if_exist:cT { \um_to_usv:nn {#4}{#2} } {
    \clist_map_inline:nn {#3}
      { \um_set_mathalphabet_char:Nnnn #1 {##1} {#4} {#2} }
  }
}
%    \end{macrocode}
%
%    \begin{macrocode}
\cs_new:Nn \um_set_mathalphabet_numbers:Nnn {
  \clist_map_inline:nn {#2}
    { \um_set_mathalph_range:nNnnn {10} #1  {##1} {#3} {num} }
}
%    \end{macrocode}
%
%    \begin{macrocode}
\cs_new:Nn \um_set_mathalphabet_Latin:Nnn {
  \clist_map_inline:nn {#2}
    { \um_set_mathalph_range:nNnnn {26} #1 {##1} {#3} {Latin} }
}
%    \end{macrocode}
%
%    \begin{macrocode}
\cs_new:Nn \um_set_mathalphabet_latin:Nnn {
  \clist_map_inline:nn {#2} {
    \um_set_mathalph_range:nNnnn {26} #1 {##1} {#3} {latin}
    \um_set_mathalphabet_char:Nnnn    #1 {##1} {#3} {h}
  }
}
%    \end{macrocode}
%
%    \begin{macrocode}
\cs_new:Nn \um_set_mathalphabet_Greek:Nnn {
  \clist_map_inline:nn {#2} {
    \um_set_mathalph_range:nNnnn {25} #1 {##1} {#3} {Greek}
    \um_set_mathalphabet_char:Nnnn    #1 {##1} {#3} {varTheta}
  }
}
%    \end{macrocode}
%
%    \begin{macrocode}
\cs_new:Nn \um_set_mathalphabet_greek:Nnn {
  \clist_map_inline:nn {#2} {
    \um_set_mathalph_range:nNnnn {25} #1 {##1} {#3} {greek}
    \um_set_mathalphabet_char:Nnnn    #1 {##1} {#3} {varepsilon}
    \um_set_mathalphabet_char:Nnnn    #1 {##1} {#3} {vartheta}
    \um_set_mathalphabet_char:Nnnn    #1 {##1} {#3} {varkappa}
    \um_set_mathalphabet_char:Nnnn    #1 {##1} {#3} {varphi}
    \um_set_mathalphabet_char:Nnnn    #1 {##1} {#3} {varrho}
    \um_set_mathalphabet_char:Nnnn    #1 {##1} {#3} {varpi}
  }
}
%    \end{macrocode}
%
% \subsection{Alphabets}
%
% \subsubsection{Upright: \cmd\mathup}
%    \begin{macrocode}
\cs_new:Nn \um_config_up_num:n {
  \um_map_chars_numbers:nn {up}{#1}
  \um_set_mathalphabet_numbers:Nnn \mathup {up}{#1}
}
\cs_new:Nn \um_config_up_Latin:n
  {
    \bool_if:NTF \g_um_literal_bool { \um_map_chars_Latin:nn {up} {#1} }
  {
    \bool_if:NT \g_um_upLatin_bool { \um_map_chars_Latin:nn {up,it} {#1} }
  }
  \um_set_mathalphabet_Latin:Nnn \mathup {up,it}{#1}
}
\cs_new:Nn \um_config_up_latin:n {
  \bool_if:NTF \g_um_literal_bool { \um_map_chars_latin:nn {up} {#1} }
  {
    \bool_if:NT \g_um_uplatin_bool {
      \um_map_chars_latin:nn        {up,it} {#1}
      \um_map_single:nnn        {h} {up,it} {#1}
      \um_map_single:nnn {dotlessi} {up,it} {#1}
      \um_map_single:nnn {dotlessj} {up,it} {#1}
    }
  }
  \um_set_mathalphabet_latin:Nnn \mathup {up,it}{#1}
}
\cs_new:Nn \um_config_up_Greek:n {
  \bool_if:NTF \g_um_literal_bool { \um_map_chars_Greek:nn {up}{#1} }
  {
    \bool_if:NT \g_um_upGreek_bool { \um_map_chars_Greek:nn {up,it}{#1} }
  }
  \um_set_mathalphabet_Greek:Nnn \mathup {up,it}{#1}
}
\cs_new:Nn \um_config_up_greek:n {
  \bool_if:NTF \g_um_literal_bool { \um_map_chars_greek:nn {up} {#1} }
  {
    \bool_if:NT \g_um_upgreek_bool {
      \um_map_chars_greek:nn {up,it} {#1}
    }
  }
  \um_set_mathalphabet_greek:Nnn \mathup {up,it} {#1}
}
\cs_new:Nn \um_config_up_misc:n {
  \bool_if:NTF \g_um_literal_Nabla_bool {
    \um_map_single:nnn {Nabla}{up}{up}
  }{
    \bool_if:NT \g_um_upNabla_bool {
      \um_map_single:nnn {Nabla}{up,it}{up}
    }
  }
  \bool_if:NTF \g_um_literal_partial_bool {
    \um_map_single:nnn {partial}{up}{up}
  }{
    \bool_if:NT \g_um_uppartial_bool {
      \um_map_single:nnn {partial}{up,it}{up}
    }
  }
  \um_set_mathalphabet_pos:Nnnn \mathup  {partial} {up,it} {#1}
  \um_set_mathalphabet_pos:Nnnn \mathup    {Nabla} {up,it} {#1}
  \um_set_mathalphabet_pos:Nnnn \mathup {dotlessi} {up,it} {#1}
  \um_set_mathalphabet_pos:Nnnn \mathup {dotlessj} {up,it} {#1}
}
%    \end{macrocode}
%
% \subsubsection{Italic: \cmd\mathit}
%
%    \begin{macrocode}
\cs_new:Nn \um_config_it_Latin:n {
  \bool_if:NTF \g_um_literal_bool { \um_map_chars_Latin:nn {it} {#1} }
  {
    \bool_if:NF \g_um_upLatin_bool { \um_map_chars_Latin:nn {up,it} {#1} }
  }
  \um_set_mathalphabet_Latin:Nnn \mathit {up,it}{#1}
}
\cs_new:Nn \um_config_it_latin:n {
  \bool_if:NTF \g_um_literal_bool {
    \um_map_chars_latin:nn {it} {#1}
    \um_map_single:nnn {h}{it}{#1}
  }{
    \bool_if:NF \g_um_uplatin_bool {
      \um_map_chars_latin:nn {up,it} {#1}
      \um_map_single:nnn {h}{up,it}{#1}
      \um_map_single:nnn {dotlessi}{up,it}{#1}
      \um_map_single:nnn {dotlessj}{up,it}{#1}
    }
  }
  \um_set_mathalphabet_latin:Nnn \mathit            {up,it} {#1}
  \um_set_mathalphabet_pos:Nnnn  \mathit {dotlessi} {up,it} {#1}
  \um_set_mathalphabet_pos:Nnnn  \mathit {dotlessj} {up,it} {#1}
}
\cs_new:Nn \um_config_it_Greek:n {
  \bool_if:NTF \g_um_literal_bool { \um_map_chars_Greek:nn {it}{#1}
  }{
    \bool_if:NF \g_um_upGreek_bool { \um_map_chars_Greek:nn {up,it}{#1} }
  }
  \um_set_mathalphabet_Greek:Nnn \mathit {up,it}{#1}
}
\cs_new:Nn \um_config_it_greek:n {
  \bool_if:NTF \g_um_literal_bool { \um_map_chars_greek:nn {it} {#1} }
  {
    \bool_if:NF \g_um_upgreek_bool { \um_map_chars_greek:nn {it,up} {#1} }
  }
  \um_set_mathalphabet_greek:Nnn \mathit {up,it} {#1}
}
\cs_new:Nn \um_config_it_misc:n {
  \bool_if:NTF \g_um_literal_Nabla_bool {
    \um_map_single:nnn {Nabla}{it}{it}
  }{
    \bool_if:NF \g_um_upNabla_bool {
      \um_map_single:nnn {Nabla}{up,it}{it}
    }
  }
  \bool_if:NTF \g_um_literal_partial_bool {
    \um_map_single:nnn {partial}{it}{it}
  }{
    \bool_if:NF \g_um_uppartial_bool {
      \um_map_single:nnn {partial}{up,it}{it}
    }
  }
  \um_set_mathalphabet_pos:Nnnn \mathit {partial} {up,it}{#1}
  \um_set_mathalphabet_pos:Nnnn \mathit {Nabla}   {up,it}{#1}
}
%    \end{macrocode}
%
% \subsubsection{Blackboard or double-struck: \cmd\mathbb\ and \cmd\mathbbit}
%
%    \begin{macrocode}
\cs_new:Nn \um_config_bb_latin:n {
  \um_set_mathalphabet_latin:Nnn \mathbb {up,it}{#1}
}
\cs_new:Nn \um_config_bb_Latin:n {
  \um_set_mathalphabet_Latin:Nnn \mathbb {up,it}{#1}
  \um_set_mathalphabet_pos:Nnnn  \mathbb {C} {up,it} {#1}
  \um_set_mathalphabet_pos:Nnnn  \mathbb {H} {up,it} {#1}
  \um_set_mathalphabet_pos:Nnnn  \mathbb {N} {up,it} {#1}
  \um_set_mathalphabet_pos:Nnnn  \mathbb {P} {up,it} {#1}
  \um_set_mathalphabet_pos:Nnnn  \mathbb {Q} {up,it} {#1}
  \um_set_mathalphabet_pos:Nnnn  \mathbb {R} {up,it} {#1}
  \um_set_mathalphabet_pos:Nnnn  \mathbb {Z} {up,it} {#1}
}
\cs_new:Nn \um_config_bb_num:n {
  \um_set_mathalphabet_numbers:Nnn \mathbb {up}{#1}
}
\cs_new:Nn \um_config_bb_misc:n {
  \um_set_mathalphabet_pos:Nnnn \mathbb        {Pi} {up,it} {#1}
  \um_set_mathalphabet_pos:Nnnn \mathbb        {pi} {up,it} {#1}
  \um_set_mathalphabet_pos:Nnnn \mathbb     {Gamma} {up,it} {#1}
  \um_set_mathalphabet_pos:Nnnn \mathbb     {gamma} {up,it} {#1}
  \um_set_mathalphabet_pos:Nnnn \mathbb {summation} {up} {#1}
}
\cs_new:Nn \um_config_bbit_misc:n {
  \um_set_mathalphabet_pos:Nnnn \mathbbit {D} {up,it} {#1}
  \um_set_mathalphabet_pos:Nnnn \mathbbit {d} {up,it} {#1}
  \um_set_mathalphabet_pos:Nnnn \mathbbit {e} {up,it} {#1}
  \um_set_mathalphabet_pos:Nnnn \mathbbit {i} {up,it} {#1}
  \um_set_mathalphabet_pos:Nnnn \mathbbit {j} {up,it} {#1}
}
%    \end{macrocode}
%
% \subsubsection{Script and caligraphic: \cmd\mathscr\ and \cmd\mathcal}
%
%    \begin{macrocode}
\cs_new:Nn \um_config_scr_Latin:n {
  \um_set_mathalphabet_Latin:Nnn \mathscr    {up,it}{#1}
  \um_set_mathalphabet_pos:Nnnn  \mathscr {B}{up,it}{#1}
  \um_set_mathalphabet_pos:Nnnn  \mathscr {E}{up,it}{#1}
  \um_set_mathalphabet_pos:Nnnn  \mathscr {F}{up,it}{#1}
  \um_set_mathalphabet_pos:Nnnn  \mathscr {H}{up,it}{#1}
  \um_set_mathalphabet_pos:Nnnn  \mathscr {I}{up,it}{#1}
  \um_set_mathalphabet_pos:Nnnn  \mathscr {L}{up,it}{#1}
  \um_set_mathalphabet_pos:Nnnn  \mathscr {M}{up,it}{#1}
  \um_set_mathalphabet_pos:Nnnn  \mathscr {R}{up,it}{#1}
}
\cs_new:Nn \um_config_scr_latin:n {
  \um_set_mathalphabet_latin:Nnn \mathscr    {up,it}{#1}
  \um_set_mathalphabet_pos:Nnnn  \mathscr {e}{up,it}{#1}
  \um_set_mathalphabet_pos:Nnnn  \mathscr {g}{up,it}{#1}
  \um_set_mathalphabet_pos:Nnnn  \mathscr {o}{up,it}{#1}
}
%    \end{macrocode}
% These are by default synonyms for the above, but with the STIX
% fonts we want to use the alternate alphabet.
%    \begin{macrocode}
\cs_new:Nn \um_config_cal_Latin:n {
  \um_set_mathalphabet_Latin:Nnn  \mathcal  {up,it}{#1}
  \um_set_mathalphabet_pos:Nnnn   \mathcal {B}{up,it}{#1}
  \um_set_mathalphabet_pos:Nnnn   \mathcal {E}{up,it}{#1}
  \um_set_mathalphabet_pos:Nnnn   \mathcal {F}{up,it}{#1}
  \um_set_mathalphabet_pos:Nnnn   \mathcal {H}{up,it}{#1}
  \um_set_mathalphabet_pos:Nnnn   \mathcal {I}{up,it}{#1}
  \um_set_mathalphabet_pos:Nnnn   \mathcal {L}{up,it}{#1}
  \um_set_mathalphabet_pos:Nnnn   \mathcal {M}{up,it}{#1}
  \um_set_mathalphabet_pos:Nnnn   \mathcal {R}{up,it}{#1}
}
%    \end{macrocode}
%
% \subsubsection{Fractur or fraktur or blackletter: \cmd\mathfrak}
%
%    \begin{macrocode}
\cs_new:Nn \um_config_frak_Latin:n {
  \um_set_mathalphabet_Latin:Nnn \mathfrak    {up,it}{#1}
  \um_set_mathalphabet_pos:Nnnn  \mathfrak {C}{up,it}{#1}
  \um_set_mathalphabet_pos:Nnnn  \mathfrak {H}{up,it}{#1}
  \um_set_mathalphabet_pos:Nnnn  \mathfrak {I}{up,it}{#1}
  \um_set_mathalphabet_pos:Nnnn  \mathfrak {R}{up,it}{#1}
  \um_set_mathalphabet_pos:Nnnn  \mathfrak {Z}{up,it}{#1}
}
\cs_new:Nn \um_config_frak_latin:n {
  \um_set_mathalphabet_latin:Nnn \mathfrak {up,it}{#1}
}
%    \end{macrocode}
%
% \subsubsection{Sans serif upright: \cmd\mathsfup}
%    \begin{macrocode}
\cs_new:Nn \um_config_sfup_num:n {
  \um_set_mathalphabet_numbers:Nnn \mathsf   {up}{#1}
  \um_set_mathalphabet_numbers:Nnn \mathsfup {up}{#1}
}
\cs_new:Nn \um_config_sfup_Latin:n {
  \bool_if:NTF \g_um_sfliteral_bool {
    \um_map_chars_Latin:nn {sfup} {#1}
    \um_set_mathalphabet_Latin:Nnn \mathsf {up}{#1}
  }{
    \bool_if:NT \g_um_upsans_bool {
      \um_map_chars_Latin:nn {sfup,sfit} {#1}
      \um_set_mathalphabet_Latin:Nnn \mathsf {up,it}{#1}
    }
  }
  \um_set_mathalphabet_Latin:Nnn \mathsfup {up,it}{#1}
}
\cs_new:Nn \um_config_sfup_latin:n {
  \bool_if:NTF \g_um_sfliteral_bool {
    \um_map_chars_latin:nn {sfup} {#1}
    \um_set_mathalphabet_latin:Nnn \mathsf {up}{#1}
  }{
    \bool_if:NT \g_um_upsans_bool {
      \um_map_chars_latin:nn {sfup,sfit} {#1}
      \um_set_mathalphabet_latin:Nnn \mathsf {up,it}{#1}
    }
  }
  \um_set_mathalphabet_latin:Nnn \mathsfup {up,it}{#1}
}
%    \end{macrocode}
%
% \subsubsection{Sans serif italic: \cmd\mathsfit}
%
%    \begin{macrocode}
\cs_new:Nn \um_config_sfit_Latin:n {
  \bool_if:NTF \g_um_sfliteral_bool {
    \um_map_chars_Latin:nn {sfit} {#1}
    \um_set_mathalphabet_Latin:Nnn \mathsf {it}{#1}
  }{
    \bool_if:NF \g_um_upsans_bool {
      \um_map_chars_Latin:nn {sfup,sfit} {#1}
      \um_set_mathalphabet_Latin:Nnn \mathsf {up,it}{#1}
    }
  }
  \um_set_mathalphabet_Latin:Nnn \mathsfit {up,it}{#1}
}
\cs_new:Nn \um_config_sfit_latin:n {
  \bool_if:NTF \g_um_sfliteral_bool {
    \um_map_chars_latin:nn {sfit} {#1}
    \um_set_mathalphabet_latin:Nnn \mathsf {it}{#1}
  }{
    \bool_if:NF \g_um_upsans_bool {
      \um_map_chars_latin:nn {sfup,sfit} {#1}
      \um_set_mathalphabet_latin:Nnn \mathsf {up,it}{#1}
    }
  }
  \um_set_mathalphabet_latin:Nnn \mathsfit {up,it}{#1}
}
%    \end{macrocode}
%
% \subsubsection{Typewriter or monospaced: \cmd\mathtt}
%    \begin{macrocode}
\cs_new:Nn \um_config_tt_num:n {
  \um_set_mathalphabet_numbers:Nnn \mathtt {up}{#1}
}
\cs_new:Nn \um_config_tt_Latin:n {
  \um_set_mathalphabet_Latin:Nnn \mathtt {up,it}{#1}
}
\cs_new:Nn \um_config_tt_latin:n {
  \um_set_mathalphabet_latin:Nnn \mathtt {up,it}{#1}
}
%    \end{macrocode}
%
%
% \subsubsection{Bold Italic: \cmd\mathbfit}
%    \begin{macrocode}
\cs_new:Nn \um_config_bfit_Latin:n {
  \bool_if:NF \g_um_bfupLatin_bool {
    \um_map_chars_Latin:nn {bfup,bfit} {#1}
  }
  \um_set_mathalphabet_Latin:Nnn \mathbfit {up,it}{#1}
  \bool_if:NTF \g_um_bfliteral_bool {
    \um_map_chars_Latin:nn {bfit} {#1}
    \um_set_mathalphabet_Latin:Nnn \mathbf {it}{#1}
  }{
    \bool_if:NF \g_um_bfupLatin_bool {
      \um_map_chars_Latin:nn {bfup,bfit} {#1}
      \um_set_mathalphabet_Latin:Nnn \mathbf {up,it}{#1}
    }
  }
}
\cs_new:Nn \um_config_bfit_latin:n {
  \bool_if:NF \g_um_bfuplatin_bool {
    \um_map_chars_latin:nn {bfup,bfit} {#1}
  }
  \um_set_mathalphabet_latin:Nnn \mathbfit {up,it}{#1}
  \bool_if:NTF \g_um_bfliteral_bool {
    \um_map_chars_latin:nn {bfit} {#1}
    \um_set_mathalphabet_latin:Nnn \mathbf {it}{#1}
  }{
    \bool_if:NF \g_um_bfuplatin_bool {
      \um_map_chars_latin:nn {bfup,bfit} {#1}
      \um_set_mathalphabet_latin:Nnn \mathbf {up,it}{#1}
    }
  }
}
\cs_new:Nn \um_config_bfit_Greek:n {
  \um_set_mathalphabet_Greek:Nnn \mathbfit {up,it}{#1}
  \bool_if:NTF \g_um_bfliteral_bool {
    \um_map_chars_Greek:nn {bfit}{#1}
    \um_set_mathalphabet_Greek:Nnn \mathbf {it}{#1}
  }{
    \bool_if:NF \g_um_bfupGreek_bool {
      \um_map_chars_Greek:nn {bfup,bfit}{#1}
      \um_set_mathalphabet_Greek:Nnn \mathbf {up,it}{#1}
    }
  }
}
\cs_new:Nn \um_config_bfit_greek:n {
  \um_set_mathalphabet_greek:Nnn \mathbfit {up,it} {#1}
  \bool_if:NTF \g_um_bfliteral_bool {
    \um_map_chars_greek:nn {bfit} {#1}
    \um_set_mathalphabet_greek:Nnn \mathbf {it} {#1}
  }{
    \bool_if:NF \g_um_bfupgreek_bool {
      \um_map_chars_greek:nn {bfit,bfup} {#1}
      \um_set_mathalphabet_greek:Nnn \mathbf {up,it} {#1}
    }
  }
}
\cs_new:Nn \um_config_bfit_misc:n {
  \bool_if:NTF \g_um_literal_Nabla_bool {
    \um_map_single:nnn {Nabla}{bfit}{#1}
  }{
    \bool_if:NF \g_um_upNabla_bool {
      \um_map_single:nnn {Nabla}{bfup,bfit}{#1}
    }
  }
  \bool_if:NTF \g_um_literal_partial_bool {
    \um_map_single:nnn {partial}{bfit}{#1}
  }{
    \bool_if:NF \g_um_uppartial_bool {
      \um_map_single:nnn {partial}{bfup,bfit}{#1}
    }
  }
  \um_set_mathalphabet_pos:Nnnn  \mathbfit {partial} {up,it}{#1}
  \um_set_mathalphabet_pos:Nnnn  \mathbfit {Nabla}   {up,it}{#1}
  \bool_if:NTF \g_um_literal_partial_bool {
    \um_set_mathalphabet_pos:Nnnn  \mathbf {partial} {it}{#1}
  }{
    \bool_if:NF \g_um_uppartial_bool {
      \um_set_mathalphabet_pos:Nnnn  \mathbf {partial} {up,it}{#1}
    }
  }
  \bool_if:NTF \g_um_literal_Nabla_bool {
    \um_set_mathalphabet_pos:Nnnn  \mathbf {Nabla}   {it}{#1}
  }{
    \bool_if:NF \g_um_upNabla_bool {
      \um_set_mathalphabet_pos:Nnnn  \mathbf {Nabla}   {up,it}{#1}
    }
  }
}
%    \end{macrocode}
%
%
% \subsubsection{Bold Upright: \cmd\mathbfup}
%    \begin{macrocode}
\cs_new:Nn \um_config_bfup_num:n {
  \um_set_mathalphabet_numbers:Nnn \mathbf   {up}{#1}
  \um_set_mathalphabet_numbers:Nnn \mathbfup {up}{#1}
}
\cs_new:Nn \um_config_bfup_Latin:n {
  \bool_if:NT \g_um_bfupLatin_bool {
    \um_map_chars_Latin:nn {bfup,bfit} {#1}
  }
  \um_set_mathalphabet_Latin:Nnn \mathbfup {up,it}{#1}
  \bool_if:NTF \g_um_bfliteral_bool {
    \um_map_chars_Latin:nn {bfup} {#1}
    \um_set_mathalphabet_Latin:Nnn \mathbf {up}{#1}
  }{
    \bool_if:NT \g_um_bfupLatin_bool {
      \um_map_chars_Latin:nn {bfup,bfit} {#1}
      \um_set_mathalphabet_Latin:Nnn \mathbf {up,it}{#1}
    }
  }
}
\cs_new:Nn \um_config_bfup_latin:n {
  \bool_if:NT \g_um_bfuplatin_bool {
    \um_map_chars_latin:nn {bfup,bfit} {#1}
  }
  \um_set_mathalphabet_latin:Nnn \mathbfup {up,it}{#1}
  \bool_if:NTF \g_um_bfliteral_bool {
    \um_map_chars_latin:nn {bfup} {#1}
    \um_set_mathalphabet_latin:Nnn \mathbf {up}{#1}
  }{
    \bool_if:NT \g_um_bfuplatin_bool {
      \um_map_chars_latin:nn {bfup,bfit} {#1}
      \um_set_mathalphabet_latin:Nnn \mathbf {up,it}{#1}
    }
  }
}
\cs_new:Nn \um_config_bfup_Greek:n {
  \um_set_mathalphabet_Greek:Nnn \mathbfup {up,it}{#1}
  \bool_if:NTF \g_um_bfliteral_bool {
    \um_map_chars_Greek:nn {bfup}{#1}
    \um_set_mathalphabet_Greek:Nnn \mathbf {up}{#1}
  }{
    \bool_if:NT \g_um_bfupGreek_bool {
      \um_map_chars_Greek:nn {bfup,bfit}{#1}
      \um_set_mathalphabet_Greek:Nnn \mathbf {up,it}{#1}
    }
  }
}
\cs_new:Nn \um_config_bfup_greek:n {
  \um_set_mathalphabet_greek:Nnn \mathbfup {up,it} {#1}
  \bool_if:NTF \g_um_bfliteral_bool {
    \um_map_chars_greek:nn {bfup} {#1}
    \um_set_mathalphabet_greek:Nnn \mathbf {up} {#1}
  }{
    \bool_if:NT \g_um_bfupgreek_bool {
      \um_map_chars_greek:nn {bfup,bfit} {#1}
      \um_set_mathalphabet_greek:Nnn \mathbf {up,it} {#1}
    }
  }
}
\cs_new:Nn \um_config_bfup_misc:n {
  \bool_if:NTF \g_um_literal_Nabla_bool {
    \um_map_single:nnn {Nabla}{bfup}{#1}
  }{
    \bool_if:NT \g_um_upNabla_bool {
      \um_map_single:nnn {Nabla}{bfup,bfit}{#1}
    }
  }
  \bool_if:NTF \g_um_literal_partial_bool {
    \um_map_single:nnn {partial}{bfup}{#1}
  }{
    \bool_if:NT \g_um_uppartial_bool {
      \um_map_single:nnn {partial}{bfup,bfit}{#1}
    }
  }
  \um_set_mathalphabet_pos:Nnnn  \mathbfup {partial} {up,it}{#1}
  \um_set_mathalphabet_pos:Nnnn  \mathbfup {Nabla}   {up,it}{#1}
  \um_set_mathalphabet_pos:Nnnn  \mathbfup {digamma} {up}{#1}
  \um_set_mathalphabet_pos:Nnnn  \mathbfup {Digamma} {up}{#1}
  \um_set_mathalphabet_pos:Nnnn  \mathbf   {digamma} {up}{#1}
  \um_set_mathalphabet_pos:Nnnn  \mathbf   {Digamma} {up}{#1}
  \bool_if:NTF \g_um_literal_partial_bool {
    \um_set_mathalphabet_pos:Nnnn  \mathbf {partial} {up}{#1}
  }{
    \bool_if:NT \g_um_uppartial_bool {
      \um_set_mathalphabet_pos:Nnnn  \mathbf {partial} {up,it}{#1}
    }
  }
  \bool_if:NTF \g_um_literal_Nabla_bool {
    \um_set_mathalphabet_pos:Nnnn  \mathbf {Nabla}   {up}{#1}
  }{
    \bool_if:NT \g_um_upNabla_bool {
      \um_set_mathalphabet_pos:Nnnn  \mathbf {Nabla}   {up,it}{#1}
    }
  }
}
%    \end{macrocode}
%
% \subsubsection{Bold fractur or fraktur or blackletter: \cmd\mathbffrak}
%    \begin{macrocode}
\cs_new:Nn \um_config_bffrak_Latin:n {
  \um_set_mathalphabet_Latin:Nnn \mathbffrak {up,it}{#1}
}
\cs_new:Nn \um_config_bffrak_latin:n {
  \um_set_mathalphabet_latin:Nnn \mathbffrak {up,it}{#1}
}
%    \end{macrocode}
%
% \subsubsection{Bold script or calligraphic: \cmd\mathbfscr}
%    \begin{macrocode}
\cs_new:Nn \um_config_bfscr_Latin:n {
  \um_set_mathalphabet_Latin:Nnn \mathbfscr {up,it}{#1}
}
\cs_new:Nn \um_config_bfscr_latin:n {
  \um_set_mathalphabet_latin:Nnn \mathbfscr {up,it}{#1}
}
\cs_new:Nn \um_config_bfcal_Latin:n {
  \um_set_mathalphabet_Latin:Nnn   \mathbfcal  {up,it}{#1}
}
%    \end{macrocode}
%
% \subsubsection{Bold upright sans serif: \cmd\mathbfsfup}
%    \begin{macrocode}
\cs_new:Nn \um_config_bfsfup_num:n {
  \um_set_mathalphabet_numbers:Nnn \mathbfsf   {up}{#1}
  \um_set_mathalphabet_numbers:Nnn \mathbfsfup {up}{#1}
}
\cs_new:Nn \um_config_bfsfup_Latin:n {
  \bool_if:NTF \g_um_sfliteral_bool {
    \um_map_chars_Latin:nn {bfsfup} {#1}
    \um_set_mathalphabet_Latin:Nnn \mathbfsf {up}{#1}
  }{
    \bool_if:NT \g_um_upsans_bool {
      \um_map_chars_Latin:nn {bfsfup,bfsfit} {#1}
      \um_set_mathalphabet_Latin:Nnn \mathbfsf {up,it}{#1}
    }
  }
  \um_set_mathalphabet_Latin:Nnn \mathbfsfup {up,it}{#1}
}
\cs_new:Nn \um_config_bfsfup_latin:n {
  \bool_if:NTF \g_um_sfliteral_bool {
    \um_map_chars_latin:nn {bfsfup} {#1}
    \um_set_mathalphabet_latin:Nnn \mathbfsf {up}{#1}
  }{
    \bool_if:NT \g_um_upsans_bool {
      \um_map_chars_latin:nn {bfsfup,bfsfit} {#1}
      \um_set_mathalphabet_latin:Nnn \mathbfsf {up,it}{#1}
    }
  }
  \um_set_mathalphabet_latin:Nnn \mathbfsfup {up,it}{#1}
}
\cs_new:Nn \um_config_bfsfup_Greek:n {
  \bool_if:NTF \g_um_sfliteral_bool {
    \um_map_chars_Greek:nn {bfsfup}{#1}
    \um_set_mathalphabet_Greek:Nnn \mathbfsf {up}{#1}
  }{
    \bool_if:NT \g_um_upsans_bool {
      \um_map_chars_Greek:nn {bfsfup,bfsfit}{#1}
      \um_set_mathalphabet_Greek:Nnn \mathbfsf {up,it}{#1}
    }
  }
  \um_set_mathalphabet_Greek:Nnn \mathbfsfup {up,it}{#1}
}
\cs_new:Nn \um_config_bfsfup_greek:n {
  \bool_if:NTF \g_um_sfliteral_bool {
    \um_map_chars_greek:nn {bfsfup} {#1}
    \um_set_mathalphabet_greek:Nnn \mathbfsf {up} {#1}
  }{
    \bool_if:NT \g_um_upsans_bool {
      \um_map_chars_greek:nn {bfsfup,bfsfit} {#1}
      \um_set_mathalphabet_greek:Nnn \mathbfsf {up,it} {#1}
    }
  }
  \um_set_mathalphabet_greek:Nnn \mathbfsfup {up,it} {#1}
}
\cs_new:Nn \um_config_bfsfup_misc:n {
  \bool_if:NTF \g_um_literal_Nabla_bool {
    \um_map_single:nnn {Nabla}{bfsfup}{#1}
  }{
    \bool_if:NT \g_um_upNabla_bool {
      \um_map_single:nnn {Nabla}{bfsfup,bfsfit}{#1}
    }
  }
  \bool_if:NTF \g_um_literal_partial_bool {
    \um_map_single:nnn {partial}{bfsfup}{#1}
  }{
    \bool_if:NT \g_um_uppartial_bool {
      \um_map_single:nnn {partial}{bfsfup,bfsfit}{#1}
    }
  }
  \um_set_mathalphabet_pos:Nnnn  \mathbfsfup {partial} {up,it}{#1}
  \um_set_mathalphabet_pos:Nnnn  \mathbfsfup {Nabla}   {up,it}{#1}
  \bool_if:NTF \g_um_literal_partial_bool {
    \um_set_mathalphabet_pos:Nnnn  \mathbfsf {partial} {up}{#1}
  }{
    \bool_if:NT \g_um_uppartial_bool {
      \um_set_mathalphabet_pos:Nnnn  \mathbfsf {partial} {up,it}{#1}
    }
  }
  \bool_if:NTF \g_um_literal_Nabla_bool {
    \um_set_mathalphabet_pos:Nnnn  \mathbfsf {Nabla}   {up}{#1}
  }{
    \bool_if:NT \g_um_upNabla_bool {
      \um_set_mathalphabet_pos:Nnnn  \mathbfsf {Nabla}   {up,it}{#1}
    }
  }
}
%    \end{macrocode}
%
%
% \subsubsection{Bold italic sans serif: \cmd\mathbfsfit}
%    \begin{macrocode}
\cs_new:Nn \um_config_bfsfit_Latin:n {
  \bool_if:NTF \g_um_sfliteral_bool {
    \um_map_chars_Latin:nn {bfsfit} {#1}
    \um_set_mathalphabet_Latin:Nnn \mathbfsf {it}{#1}
  }{
    \bool_if:NF \g_um_upsans_bool {
      \um_map_chars_Latin:nn {bfsfup,bfsfit} {#1}
      \um_set_mathalphabet_Latin:Nnn \mathbfsf {up,it}{#1}
    }
  }
  \um_set_mathalphabet_Latin:Nnn \mathbfsfit {up,it}{#1}
}
\cs_new:Nn \um_config_bfsfit_latin:n {
  \bool_if:NTF \g_um_sfliteral_bool {
    \um_map_chars_latin:nn {bfsfit} {#1}
    \um_set_mathalphabet_latin:Nnn \mathbfsf {it}{#1}
  }{
    \bool_if:NF \g_um_upsans_bool {
      \um_map_chars_latin:nn {bfsfup,bfsfit} {#1}
      \um_set_mathalphabet_latin:Nnn \mathbfsf {up,it}{#1}
    }
  }
  \um_set_mathalphabet_latin:Nnn \mathbfsfit {up,it}{#1}
}
\cs_new:Nn \um_config_bfsfit_Greek:n {
  \bool_if:NTF \g_um_sfliteral_bool {
    \um_map_chars_Greek:nn {bfsfit}{#1}
    \um_set_mathalphabet_Greek:Nnn \mathbfsf {it}{#1}
  }{
    \bool_if:NF \g_um_upsans_bool {
      \um_map_chars_Greek:nn {bfsfup,bfsfit}{#1}
      \um_set_mathalphabet_Greek:Nnn \mathbfsf {up,it}{#1}
    }
  }
  \um_set_mathalphabet_Greek:Nnn \mathbfsfit {up,it}{#1}
}
\cs_new:Nn \um_config_bfsfit_greek:n {
  \bool_if:NTF \g_um_sfliteral_bool {
    \um_map_chars_greek:nn {bfsfit} {#1}
    \um_set_mathalphabet_greek:Nnn \mathbfsf {it} {#1}
  }{
    \bool_if:NF \g_um_upsans_bool {
      \um_map_chars_greek:nn {bfsfup,bfsfit} {#1}
      \um_set_mathalphabet_greek:Nnn \mathbfsf {up,it} {#1}
    }
  }
  \um_set_mathalphabet_greek:Nnn \mathbfsfit {up,it} {#1}
}
\cs_new:Nn \um_config_bfsfit_misc:n {
  \bool_if:NTF \g_um_literal_Nabla_bool {
    \um_map_single:nnn {Nabla}{bfsfit}{#1}
  }{
    \bool_if:NF \g_um_upNabla_bool {
      \um_map_single:nnn {Nabla}{bfsfup,bfsfit}{#1}
    }
  }
  \bool_if:NTF \g_um_literal_partial_bool {
    \um_map_single:nnn {partial}{bfsfit}{#1}
  }{
    \bool_if:NF \g_um_uppartial_bool {
      \um_map_single:nnn {partial}{bfsfup,bfsfit}{#1}
    }
  }
  \um_set_mathalphabet_pos:Nnnn  \mathbfsfit {partial} {up,it}{#1}
  \um_set_mathalphabet_pos:Nnnn  \mathbfsfit {Nabla}   {up,it}{#1}
  \bool_if:NTF \g_um_literal_partial_bool {
    \um_set_mathalphabet_pos:Nnnn  \mathbfsf {partial} {it}{#1}
  }{
    \bool_if:NF \g_um_uppartial_bool {
      \um_set_mathalphabet_pos:Nnnn  \mathbfsf {partial} {up,it}{#1}
    }
  }
  \bool_if:NTF \g_um_literal_Nabla_bool {
    \um_set_mathalphabet_pos:Nnnn  \mathbfsf {Nabla}   {it}{#1}
  }{
    \bool_if:NF \g_um_upNabla_bool {
      \um_set_mathalphabet_pos:Nnnn  \mathbfsf {Nabla}   {up,it}{#1}
    }
  }
}
%    \end{macrocode}
%
% \section{A token list to contain the data of the math table}
%
% Instead of \cmd\input-ing the unicode math table every time we
% want to re-read its data, we save it within a macro. This has two
% advantages: 1.~it should be slightly faster, at the expense of memory;
% 2.~we don't need to worry about catcodes later, since they're frozen
% at this point.
%
% In time, the case statement inside |set_mathsymbol| will be moved in here
% to avoid re-running it every time.
%    \begin{macrocode}
\cs_new:Npn \um_symbol_setup: {
  \cs_set:Npn \UnicodeMathSymbol ##1##2##3##4 {
    \exp_not:n {\_um_sym:nnn{##1}{##2}{##3}}
  }
}
%    \end{macrocode}
%
%    \begin{macrocode}
\CatchFileEdef \g_um_mathtable_tl {unicode-math-table.tex} {\um_symbol_setup:}
%    \end{macrocode}
%
%
% \begin{macro}{\um_input_math_symbol_table:}
% This function simply expands to the token list containing all the data.
%    \begin{macrocode}
\cs_new:Nn \um_input_math_symbol_table: {\g_um_mathtable_tl}
%    \end{macrocode}
% \end{macro}
%
%
% \section{Definitions of the active math characters}
%
% Here we define every Unicode math codepoint an equivalent macro name.
% The two are equivalent, in a |\let\xyz=^^^^1234| kind of way.
%
% \begin{macro}{\um_cs_set_eq_active_char:Nw}
% \begin{macro}{\um_active_char_set:wc}
% We need to do some trickery to transform the |\_um_sym:nnn| argument
% |"ABCDEF| into the \XeTeX\ `caret input' form |^^^^^abcdef|. It is \emph{very important}
% that the argument has five characters. Otherwise we need to change the number of |^| chars.
%
% To do this, turn |^| into a regular `other' character and define the macro
% to perform the lowercasing and |\let|. \cmd\scantokens\ changes the carets
% back into their original meaning after the group has ended and |^|'s catcode returns to normal.
%    \begin{macrocode}
\group_begin:
  \char_set_catcode_other:N \^
  \cs_gset:Npn \um_cs_set_eq_active_char:Nw #1 = "#2 \q_nil {
    \tex_lowercase:D {
      \tl_rescan:nn {
        \ExplSyntaxOn
        \char_set_catcode_other:N \{
        \char_set_catcode_other:N \}
        \char_set_catcode_other:N \&
        \char_set_catcode_other:N \%
        \char_set_catcode_other:N \$
      }{
        \cs_gset_eq:NN #1 ^^^^^#2
      }
    }
  }
%    \end{macrocode}
% Making |^| the right catcode isn't strictly necessary right now but it helps
% to future proof us with, e.g., breqn.
% Because we're inside a |\tl_rescan:nn|, use plain old \TeX\ syntax to avoid
% any catcode problems.
%    \begin{macrocode}
  \cs_new:Npn \um_active_char_set:wc "#1 \q_nil #2 {
    \tex_lowercase:D {
      \tl_rescan:nn { \ExplSyntaxOn }
        { \cs_gset_protected_nopar:Npx ^^^^^#1 { \exp_not:c {#2} } }
    }
  }
\group_end:
%    \end{macrocode}
% \end{macro}
% \end{macro}
%
% Now give \cmd\_um_sym:nnn\ a definition in terms of \cmd\um_cs_set_eq_active_char:Nw\
% and we're good to go.
%
% Ensure catcodes are appropriate;
% make sure |#| is an `other' so that we don't get confused with \cs{mathoctothorpe}.
%    \begin{macrocode}
\AtBeginDocument{\um_define_math_chars:}
\cs_new:Nn \um_define_math_chars: {
  \group_begin:
    \char_set_catcode_math_superscript:N \^
    \cs_set:Npn \_um_sym:nnn ##1##2##3 {
      \bool_if:nF { \cs_if_eq_p:NN ##3 \mathaccent ||
                    \cs_if_eq_p:NN ##3 \mathopen   ||
                    \cs_if_eq_p:NN ##3 \mathclose  ||
                    \cs_if_eq_p:NN ##3 \mathover   ||
                    \cs_if_eq_p:NN ##3 \mathunder  ||
                    \cs_if_eq_p:NN ##3 \mathbotaccent } {
        \um_cs_set_eq_active_char:Nw ##2 = ##1 \q_nil \ignorespaces
      }
    }
    \char_set_catcode_other:N \#
    \um_input_math_symbol_table:
  \group_end:
}
%    \end{macrocode}
% Fix \cs{backslash}, which is defined as the escape char character
% above:
%    \begin{macrocode}
\group_begin:
  \lccode`\*=`\\
  \char_set_catcode_escape:N \|
  \char_set_catcode_other:N \\
  |lowercase{
    |AtBeginDocument{
      |let|backslash=*
    }
  }
|group_end:
%    \end{macrocode}
% Fix \cs{backslash}: ^^A |
%    \begin{macrocode}
%    \end{macrocode}
%
% \section{Fall-back font}
%
% Want to load Latin Modern Math if nothing else.
%    \begin{macrocode}
\AtBeginDocument { \um_load_lm_if_necessary: }
\cs_new:Nn \um_load_lm_if_necessary:
  {
    \cs_if_exist:NF \l_um_fontname_tl
      {
        % XXX: update this when lmmath-bold.otf is released
        \setmathfont[BoldFont={lmmath-regular.otf}]{lmmath-regular.otf}
      }
  }
%    \end{macrocode}
%
% \section{Epilogue}
%
% Lots of little things to tidy up.
%
% \subsection{Primes}
%
% We need a new `prime' algorithm. Unicode math has four pre-drawn prime glyphs.
% \begin{quote}\obeylines
% \unichar{2032} {prime} (\cs{prime}): $x\prime$
% \unichar{2033} {double prime} (\cs{dprime}): $x\dprime$
% \unichar{2034} {triple prime} (\cs{trprime}): $x\trprime$
% \unichar{2057} {quadruple prime} (\cs{qprime}): $x\qprime$
% \end{quote}
% As you can see, they're all drawn at the correct height without being superscripted.
% However, in a correctly behaving OpenType font,
% we also see different behaviour after the \texttt{ssty} feature is applied:
% \begin{quote}
% \font\1="Cambria Math:script=math,+ssty=0"\1
% \char"1D465\char"2032\quad
% \char"1D465\char"2033\quad
% \char"1D465\char"2034\quad
% \char"1D465\char"2057
% \end{quote}
% The glyphs are now `full size' so that when placed inside a superscript,
% their shape will match the originally sized ones. Many thanks to Ross Mills
% of Tiro Typeworks for originally pointing out this behaviour.
%
% In regular \LaTeX, primes can be entered with the straight quote character
% |'|, and multiple straight quotes chain together to produce multiple
% primes. Better results can be achieved in \pkg{unicode-math} by chaining
% multiple single primes into a pre-drawn multi-prime glyph; consider
% $x\prime{}\prime{}\prime$ vs.\ $x\trprime$.
%
% For Unicode maths, we wish to conserve this behaviour and augment it with
% the possibility of adding any combination of Unicode prime or any of the
% $n$-prime characters. E.g., the user might copy-paste a double prime from
% another source and then later type another single prime after it; the output
% should be the triple prime.
%
% Our algorithm is:
% \begin{itemize}[nolistsep]
% \item Prime encountered; pcount=1.
% \item Scan ahead; if prime: pcount:=pcount+1; repeat.
% \item If not prime, stop scanning.
% \item If pcount=1, \cs{prime}, end.
% \item If pcount=2, check \cs{dprime}; if it exists, use it, end; if not, goto last step.
% \item Ditto pcount=3 \& \cs{trprime}.
% \item Ditto pcount=4 \& \cs{qprime}.
% \item If pcount>4 or the glyph doesn't exist, insert pcount \cs{prime}s with \cs{primekern} between each.
% \end{itemize}
%
% This is a wrapper to insert a superscript; if there is a subsequent
% trailing superscript, then it is included within the insertion.
%    \begin{macrocode}
\cs_new:Nn \um_arg_i_before_egroup:n {#1\egroup}
\cs_new:Nn \um_superscript:n {
  ^\bgroup #1
  \peek_meaning_remove:NTF ^ \um_arg_i_before_egroup:n \egroup
}
%    \end{macrocode}
%
%    \begin{macrocode}
\muskip_new:N \g_um_primekern_muskip
\muskip_gset:Nn \g_um_primekern_muskip { -\thinmuskip/2 }% arbitrary
\int_new:N \l_um_primecount_int
%    \end{macrocode}
%
%    \begin{macrocode}
\cs_new:Nn \um_nprimes:Nn {
  \um_superscript:n {
    #1
    \prg_replicate:nn {#2-1} { \mskip \g_um_primekern_muskip #1 }
  }
}
\cs_new:Nn \um_nprimes_select:nn {
  \prg_case_int:nnn {#2}{
    {1} { \um_superscript:n {#1} }
    {2} {
      \um_glyph_if_exist:nTF {"2033}
        { \um_superscript:n {\um_prime_double_mchar} }
        { \um_nprimes:Nn #1 {#2} }
    }
    {3} {
      \um_glyph_if_exist:nTF {"2034}
        { \um_superscript:n {\um_prime_triple_mchar} }
        { \um_nprimes:Nn #1 {#2} }
    }
    {4} {
      \um_glyph_if_exist:nTF {"2057}
        { \um_superscript:n {\um_prime_quad_mchar} }
        { \um_nprimes:Nn #1 {#2} }
    }
  }{
    \um_nprimes:Nn #1 {#2}
  }
}
\cs_new:Nn \um_nbackprimes_select:nn {
  \prg_case_int:nnn {#2}{
    {1} { \um_superscript:n {#1} }
    {2} {
      \um_glyph_if_exist:nTF {"2036}
        { \um_superscript:n {\um_backprime_double_mchar} }
        { \um_nprimes:Nn #1 {#2} }
    }
    {3} {
      \um_glyph_if_exist:nTF {"2037}
        { \um_superscript:n {\um_backprime_triple_mchar} }
        { \um_nprimes:Nn #1 {#2} }
    }
  }{
    \um_nprimes:Nn #1 {#2}
  }
}
%    \end{macrocode}
%
% Scanning is annoying because I'm too lazy to do it for the general case.
%
%    \begin{macrocode}
\cs_new:Npn \um_scan_prime: {
  \cs_set_eq:NN \um_superscript:n \use:n
  \int_zero:N \l_um_primecount_int
  \um_scanprime_collect:N \um_prime_single_mchar
}
\cs_new:Npn \um_scan_dprime: {
  \cs_set_eq:NN \um_superscript:n \use:n
  \int_set:Nn \l_um_primecount_int {1}
  \um_scanprime_collect:N \um_prime_single_mchar
}
\cs_new:Npn \um_scan_trprime: {
  \cs_set_eq:NN \um_superscript:n \use:n
  \int_set:Nn \l_um_primecount_int {2}
  \um_scanprime_collect:N \um_prime_single_mchar
}
\cs_new:Npn \um_scan_qprime: {
  \cs_set_eq:NN \um_superscript:n \use:n
  \int_set:Nn \l_um_primecount_int {3}
  \um_scanprime_collect:N \um_prime_single_mchar
}
\cs_new:Npn \um_scan_sup_prime: {
  \int_zero:N \l_um_primecount_int
  \um_scanprime_collect:N \um_prime_single_mchar
}
\cs_new:Npn \um_scan_sup_dprime: {
  \int_set:Nn \l_um_primecount_int {1}
  \um_scanprime_collect:N \um_prime_single_mchar
}
\cs_new:Npn \um_scan_sup_trprime: {
  \int_set:Nn \l_um_primecount_int {2}
  \um_scanprime_collect:N \um_prime_single_mchar
}
\cs_new:Npn \um_scan_sup_qprime: {
  \int_set:Nn \l_um_primecount_int {3}
  \um_scanprime_collect:N \um_prime_single_mchar
}
\cs_new:Nn \um_scanprime_collect:N {
  \int_incr:N \l_um_primecount_int
  \peek_meaning_remove:NTF ' {
    \um_scanprime_collect:N #1
  }{
    \peek_meaning_remove:NTF \um_scan_prime: {
      \um_scanprime_collect:N #1
    }{
      \peek_meaning_remove:NTF ^^^^2032 {
        \um_scanprime_collect:N #1
      }{
        \peek_meaning_remove:NTF \um_scan_dprime: {
          \int_incr:N \l_um_primecount_int
          \um_scanprime_collect:N #1
        }{
          \peek_meaning_remove:NTF ^^^^2033 {
            \int_incr:N \l_um_primecount_int
            \um_scanprime_collect:N #1
          }{
            \peek_meaning_remove:NTF \um_scan_trprime: {
              \int_add:Nn \l_um_primecount_int {2}
              \um_scanprime_collect:N #1
            }{
              \peek_meaning_remove:NTF ^^^^2034 {
                \int_add:Nn \l_um_primecount_int {2}
                \um_scanprime_collect:N #1
              }{
                \peek_meaning_remove:NTF \um_scan_qprime: {
                  \int_add:Nn \l_um_primecount_int {3}
                  \um_scanprime_collect:N #1
                }{
                  \peek_meaning_remove:NTF ^^^^2057 {
                    \int_add:Nn \l_um_primecount_int {3}
                    \um_scanprime_collect:N #1
                  }{
                    \um_nprimes_select:nn {#1} {\l_um_primecount_int}
                  }
                }
              }
            }
          }
        }
      }
    }
  }
}
\cs_new:Npn \um_scan_backprime: {
  \cs_set_eq:NN \um_superscript:n \use:n
  \int_zero:N \l_um_primecount_int
  \um_scanbackprime_collect:N \um_backprime_single_mchar
}
\cs_new:Npn \um_scan_backdprime: {
  \cs_set_eq:NN \um_superscript:n \use:n
  \int_set:Nn \l_um_primecount_int {1}
  \um_scanbackprime_collect:N \um_backprime_single_mchar
}
\cs_new:Npn \um_scan_backtrprime: {
  \cs_set_eq:NN \um_superscript:n \use:n
  \int_set:Nn \l_um_primecount_int {2}
  \um_scanbackprime_collect:N \um_backprime_single_mchar
}
\cs_new:Npn \um_scan_sup_backprime: {
  \int_zero:N \l_um_primecount_int
  \um_scanbackprime_collect:N \um_backprime_single_mchar
}
\cs_new:Npn \um_scan_sup_backdprime: {
  \int_set:Nn \l_um_primecount_int {1}
  \um_scanbackprime_collect:N \um_backprime_single_mchar
}
\cs_new:Npn \um_scan_sup_backtrprime: {
  \int_set:Nn \l_um_primecount_int {2}
  \um_scanbackprime_collect:N \um_backprime_single_mchar
}
\cs_new:Nn \um_scanbackprime_collect:N {
  \int_incr:N \l_um_primecount_int
  \peek_meaning_remove:NTF ` {
    \um_scanbackprime_collect:N #1
  }{
    \peek_meaning_remove:NTF \um_scan_backprime: {
      \um_scanbackprime_collect:N #1
    }{
      \peek_meaning_remove:NTF ^^^^2035 {
        \um_scanbackprime_collect:N #1
      }{
        \peek_meaning_remove:NTF \um_scan_backdprime: {
          \int_incr:N \l_um_primecount_int
          \um_scanbackprime_collect:N #1
        }{
          \peek_meaning_remove:NTF ^^^^2036 {
            \int_incr:N \l_um_primecount_int
            \um_scanbackprime_collect:N #1
          }{
            \peek_meaning_remove:NTF \um_scan_backtrprime: {
              \int_add:Nn \l_um_primecount_int {2}
              \um_scanbackprime_collect:N #1
            }{
              \peek_meaning_remove:NTF ^^^^2037 {
                \int_add:Nn \l_um_primecount_int {2}
                \um_scanbackprime_collect:N #1
              }{
                \um_nbackprimes_select:nn {#1} {\l_um_primecount_int}
              }
            }
          }
        }
      }
    }
  }
}
%    \end{macrocode}
%
%    \begin{macrocode}
\AtBeginDocument{\um_define_prime_commands: \um_define_prime_chars:}
\cs_new:Nn \um_define_prime_commands: {
  \cs_set_eq:NN \prime       \um_prime_single_mchar
  \cs_set_eq:NN \dprime      \um_prime_double_mchar
  \cs_set_eq:NN \trprime     \um_prime_triple_mchar
  \cs_set_eq:NN \qprime      \um_prime_quad_mchar
  \cs_set_eq:NN \backprime   \um_backprime_single_mchar
  \cs_set_eq:NN \backdprime  \um_backprime_double_mchar
  \cs_set_eq:NN \backtrprime \um_backprime_triple_mchar
}
\group_begin:
  \char_set_catcode_active:N \'
  \char_set_catcode_active:N \`
  \char_set_catcode_active:n {"2032}
  \char_set_catcode_active:n {"2033}
  \char_set_catcode_active:n {"2034}
  \char_set_catcode_active:n {"2057}
  \char_set_catcode_active:n {"2035}
  \char_set_catcode_active:n {"2036}
  \char_set_catcode_active:n {"2037}
  \cs_gset:Nn \um_define_prime_chars: {
    \cs_set_eq:NN '        \um_scan_sup_prime:
    \cs_set_eq:NN ^^^^2032 \um_scan_sup_prime:
    \cs_set_eq:NN ^^^^2033 \um_scan_sup_dprime:
    \cs_set_eq:NN ^^^^2034 \um_scan_sup_trprime:
    \cs_set_eq:NN ^^^^2057 \um_scan_sup_qprime:
    \cs_set_eq:NN `        \um_scan_sup_backprime:
    \cs_set_eq:NN ^^^^2035 \um_scan_sup_backprime:
    \cs_set_eq:NN ^^^^2036 \um_scan_sup_backdprime:
    \cs_set_eq:NN ^^^^2037 \um_scan_sup_backtrprime:
  }
\group_end:
%    \end{macrocode}
%
% \subsection{Unicode radicals}
%
%    \begin{macrocode}
\AtBeginDocument{\um_redefine_radical:}
\cs_new:Nn \um_redefine_radical:
%<*XE>
 {
  \@ifpackageloaded { amsmath } { } {
%    \end{macrocode}
% \begin{macro}{\r@@t}
% \darg{A mathstyle (for \cmd\mathpalette)}
% \darg{Leading superscript for the sqrt sign}
% A re-implementation of \LaTeX's hard-coded n-root sign using the appropriate \cmd\fontdimen s.
%    \begin{macrocode}
    \cs_set_nopar:Npn \r@@t ##1 ##2 {
      \hbox_set:Nn \l_tmpa_box {
        \c_math_toggle_token
        \m@th
        ##1
        \sqrtsign { ##2 }
        \c_math_toggle_token
      }
      \um_mathstyle_scale:Nnn ##1 { \kern } {
        \fontdimen 63 \l_um_font
      }
      \box_move_up:nn {
        (\box_ht:N \l_tmpa_box - \box_dp:N \l_tmpa_box)
        * \number \fontdimen 65 \l_um_font / 100
      } {
        \box_use:N \rootbox
      }
      \um_mathstyle_scale:Nnn ##1 { \kern } {
        \fontdimen 64 \l_um_font
      }
      \box_use_clear:N \l_tmpa_box
    }
%    \end{macrocode}
% \end{macro}
%    \begin{macrocode}
  }
 }
%</XE>
%<*LU>
 {
  \@ifpackageloaded { amsmath } { } {
%    \end{macrocode}
% \begin{macro}{\root}
% Redefine this macro for \LuaTeX, which provides us a nice primitive to use.
%    \begin{macrocode}
    \cs_set:Npn \root ##1 \of ##2 {
      \luatexUroot \l_um_radical_sqrt_tl { ##1 } { ##2 }
    }
%    \end{macrocode}
% \end{macro}
%    \begin{macrocode}
  }
 }
%</LU>
%    \end{macrocode}
%
%
% \begin{macro}{\um_fontdimen_to_percent:nn}
% \begin{macro}{\um_fontdimen_to_scale:nn}
% \darg{Font dimen number}
% \darg{Font `variable'}
% \cmd\fontdimen s |10|, |11|, and |65| aren't actually dimensions, they're percentage values given in units of |sp|.
% \cs{um_fontdimen_to_percent:nn} takes a font dimension number and outputs the decimal value of the associated parameter.
% \cs{um_fontdimen_to_scale:nn} returns a dimension correspond to the current
% font size relative proportion based on that percentage.
%    \begin{macrocode}
\cs_new:Nn \um_fontdimen_to_percent:nn {
  \strip@pt\dimexpr\fontdimen#1#2*65536/100\relax
}
\cs_new:Nn \um_fontdimen_to_scale:nn
  {
    \um_fontdimen_to_percent:nn {#1} {#2} \dimexpr \f@size pt\relax
  }
%    \end{macrocode}
% \end{macro}
% \end{macro}
%
% \begin{macro}{\um_mathstyle_scale:Nnn}
% \darg{A math style (\cs{scriptstyle}, say)}
% \darg{Macro that takes a non-delimited length argument (like \cmd\kern)}
% \darg{Length control sequence to be scaled according to the math style}
% This macro is used to scale the lengths reported by \cmd\fontdimen\ according to the scale factor for script- and scriptscript-size objects.
%    \begin{macrocode}
\cs_new:Nn \um_mathstyle_scale:Nnn {
  \ifx#1\scriptstyle
    #2\um_fontdimen_to_percent:nn{10}\l_um_font#3
  \else
    \ifx#1\scriptscriptstyle
      #2\um_fontdimen_to_percent:nn{11}\l_um_font#3
    \else
      #2#3
    \fi
  \fi
}
%    \end{macrocode}
% \end{macro}
%
% \subsection{Unicode sub- and super-scripts}
%
% The idea here is to enter a scanning state after a superscript or subscript
% is encountered.
% If subsequent superscripts or subscripts (resp.) are found,
% they are lumped together.
% Each sub/super has a corresponding regular size
% glyph which is used by \XeTeX\ to typeset the results; this means that the
% actual subscript/superscript glyphs are never seen in the output
% document~--- they are only used as input characters.
%
% Open question: should the superscript-like `modifiers' (\unichar{1D2C}
% {modifier capital letter a} and on) be included here?
%    \begin{macrocode}
\prop_new:N \g_um_supers_prop
\prop_new:N \g_um_subs_prop
\group_begin:
%    \end{macrocode}
% \paragraph{Superscripts}
% Populate a property list with superscript characters; their meaning as their
% key, for reasons that will become apparent soon, and their replacement as
% each key's value.
% Then make the superscript active and bind it to the scanning function.
%
% \cs{scantokens} makes this process much simpler since we can activate the
% char and assign its meaning in one step.
%    \begin{macrocode}
\cs_new:Nn \um_setup_active_superscript:nn {
  \prop_gput:Nxn \g_um_supers_prop   {\meaning #1} {#2}
  \char_set_catcode_active:N #1
  \char_gmake_mathactive:N #1
  \scantokens{
    \cs_gset:Npn #1 {
      \tl_set:Nn \l_um_ss_chain_tl {#2}
      \cs_set_eq:NN \um_sub_or_super:n \sp
      \tl_set:Nn \l_um_tmpa_tl {supers}
      \um_scan_sscript:
    }
  }
}
%    \end{macrocode}
% Bam:
%    \begin{macrocode}
\um_setup_active_superscript:nn {^^^^2070} {0}
\um_setup_active_superscript:nn {^^^^00b9} {1}
\um_setup_active_superscript:nn {^^^^00b2} {2}
\um_setup_active_superscript:nn {^^^^00b3} {3}
\um_setup_active_superscript:nn {^^^^2074} {4}
\um_setup_active_superscript:nn {^^^^2075} {5}
\um_setup_active_superscript:nn {^^^^2076} {6}
\um_setup_active_superscript:nn {^^^^2077} {7}
\um_setup_active_superscript:nn {^^^^2078} {8}
\um_setup_active_superscript:nn {^^^^2079} {9}
\um_setup_active_superscript:nn {^^^^207a} {+}
\um_setup_active_superscript:nn {^^^^207b} {-}
\um_setup_active_superscript:nn {^^^^207c} {=}
\um_setup_active_superscript:nn {^^^^207d} {(}
\um_setup_active_superscript:nn {^^^^207e} {)}
\um_setup_active_superscript:nn {^^^^2071} {i}
\um_setup_active_superscript:nn {^^^^207f} {n}
%    \end{macrocode}
% \paragraph{Subscripts} Ditto above.
%    \begin{macrocode}
\cs_new:Nn \um_setup_active_subscript:nn {
  \prop_gput:Nxn \g_um_subs_prop   {\meaning #1} {#2}
  \char_set_catcode_active:N #1
  \char_gmake_mathactive:N #1
  \scantokens{
    \cs_gset:Npn #1 {
      \tl_set:Nn \l_um_ss_chain_tl {#2}
      \cs_set_eq:NN \um_sub_or_super:n \sb
      \tl_set:Nn \l_um_tmpa_tl {subs}
      \um_scan_sscript:
    }
  }
}
%    \end{macrocode}
% A few more subscripts than superscripts:
%    \begin{macrocode}
\um_setup_active_subscript:nn {^^^^2080} {0}
\um_setup_active_subscript:nn {^^^^2081} {1}
\um_setup_active_subscript:nn {^^^^2082} {2}
\um_setup_active_subscript:nn {^^^^2083} {3}
\um_setup_active_subscript:nn {^^^^2084} {4}
\um_setup_active_subscript:nn {^^^^2085} {5}
\um_setup_active_subscript:nn {^^^^2086} {6}
\um_setup_active_subscript:nn {^^^^2087} {7}
\um_setup_active_subscript:nn {^^^^2088} {8}
\um_setup_active_subscript:nn {^^^^2089} {9}
\um_setup_active_subscript:nn {^^^^208a} {+}
\um_setup_active_subscript:nn {^^^^208b} {-}
\um_setup_active_subscript:nn {^^^^208c} {=}
\um_setup_active_subscript:nn {^^^^208d} {(}
\um_setup_active_subscript:nn {^^^^208e} {)}
\um_setup_active_subscript:nn {^^^^2090} {a}
\um_setup_active_subscript:nn {^^^^2091} {e}
\um_setup_active_subscript:nn {^^^^1d62} {i}
\um_setup_active_subscript:nn {^^^^2092} {o}
\um_setup_active_subscript:nn {^^^^1d63} {r}
\um_setup_active_subscript:nn {^^^^1d64} {u}
\um_setup_active_subscript:nn {^^^^1d65} {v}
\um_setup_active_subscript:nn {^^^^2093} {x}
\um_setup_active_subscript:nn {^^^^1d66} {\beta}
\um_setup_active_subscript:nn {^^^^1d67} {\gamma}
\um_setup_active_subscript:nn {^^^^1d68} {\rho}
\um_setup_active_subscript:nn {^^^^1d69} {\phi}
\um_setup_active_subscript:nn {^^^^1d6a} {\chi}
%    \end{macrocode}
%
%    \begin{macrocode}
\group_end:
%    \end{macrocode}
% The scanning command, evident in its purpose:
%    \begin{macrocode}
\cs_new:Npn \um_scan_sscript: {
  \um_scan_sscript:TF {
    \um_scan_sscript:
  }{
    \um_sub_or_super:n {\l_um_ss_chain_tl}
  }
}
%    \end{macrocode}
% The main theme here is stolen from the source to the various \cs{peek_} functions.
% Consider this function as simply boilerplate:
% TODO: move all this to expl3, and don't use internal expl3 macros.
%    \begin{macrocode}
\cs_new:Npn \um_scan_sscript:TF #1#2 {
  \tl_set:Nx \peek_true_aux:w { \exp_not:n{ #1 } }
  \tl_set_eq:NN \peek_true:w \peek_true_remove:w
  \tl_set:Nx \peek_false:w { \exp_not:n { \group_align_safe_end: #2 } }
  \group_align_safe_begin:
    \peek_after:Nw \um_peek_execute_branches_ss:
}
%    \end{macrocode}
% We do not skip spaces when scanning ahead, and we explicitly wish to
% bail out on encountering a space or a brace.
%    \begin{macrocode}
\cs_new:Npn \um_peek_execute_branches_ss: {
  \bool_if:nTF {
    \token_if_eq_catcode_p:NN \l_peek_token \c_group_begin_token ||
    \token_if_eq_catcode_p:NN \l_peek_token \c_group_end_token ||
    \token_if_eq_meaning_p:NN \l_peek_token \c_space_token
  }
  { \peek_false:w  }
  { \um_peek_execute_branches_ss_aux: }
}
%    \end{macrocode}
% This is the actual comparison code.
% Because the peeking has already tokenised the next token,
% it's too late to extract its charcode directly. Instead,
% we look at its meaning, which remains a `character' even
% though it is itself math-active. If the character is ever
% made fully active, this will break our assumptions!
%
% If the char's meaning exists as a property list key, we
% build up a chain of sub-/superscripts and iterate. (If not, exit and
% typeset what we've already collected.)
%    \begin{macrocode}
\cs_new:Npn \um_peek_execute_branches_ss_aux: {
  \prop_if_in:cxTF
    {g_um_\l_um_tmpa_tl _prop} {\meaning\l_peek_token}
    {
      \prop_get:cxN
        {g_um_\l_um_tmpa_tl _prop} {\meaning\l_peek_token} \l_um_tmpb_tl
      \tl_put_right:NV \l_um_ss_chain_tl \l_um_tmpb_tl
      \peek_true:w
    }
    { \peek_false:w }
}
%    \end{macrocode}
%
% \subsubsection{Active fractions}
% Active fractions can be setup independently of any maths font definition;
% all it requires is a mapping from the Unicode input chars to the relevant
% \LaTeX\ fraction declaration.
%
%    \begin{macrocode}
\cs_new:Npn \um_define_active_frac:Nw #1 #2/#3 {
  \char_set_catcode_active:N #1
  \char_gmake_mathactive:N #1
  \tl_rescan:nn {
    \catcode`\_=11\relax
    \catcode`\:=11\relax
  }{
    \cs_gset:Npx #1 {
      \bool_if:NTF \l_um_smallfrac_bool {\exp_not:N\tfrac} {\exp_not:N\frac}
          {#2} {#3}
    }
  }
}
%    \end{macrocode}
% These are redefined for each math font selection in case the |active-frac|
% feature changes.
%    \begin{macrocode}
\cs_new:Npn \um_setup_active_frac: {
  \group_begin:
  \um_define_active_frac:Nw  ^^^^2189  0/3
  \um_define_active_frac:Nw  ^^^^2152  1/{10}
  \um_define_active_frac:Nw  ^^^^2151  1/9
  \um_define_active_frac:Nw  ^^^^215b  1/8
  \um_define_active_frac:Nw  ^^^^2150  1/7
  \um_define_active_frac:Nw  ^^^^2159  1/6
  \um_define_active_frac:Nw  ^^^^2155  1/5
  \um_define_active_frac:Nw  ^^^^00bc  1/4
  \um_define_active_frac:Nw  ^^^^2153  1/3
  \um_define_active_frac:Nw  ^^^^215c  3/8
  \um_define_active_frac:Nw  ^^^^2156  2/5
  \um_define_active_frac:Nw  ^^^^00bd  1/2
  \um_define_active_frac:Nw  ^^^^2157  3/5
  \um_define_active_frac:Nw  ^^^^215d  5/8
  \um_define_active_frac:Nw  ^^^^2154  2/3
  \um_define_active_frac:Nw  ^^^^00be  3/4
  \um_define_active_frac:Nw  ^^^^2158  4/5
  \um_define_active_frac:Nw  ^^^^215a  5/6
  \um_define_active_frac:Nw  ^^^^215e  7/8
  \group_end:
}
\um_setup_active_frac:
%    \end{macrocode}
%
% \subsection{Synonyms and all the rest}
%
% These are symbols with multiple names. Eventually to be taken care of
% automatically by the maths characters database.
%    \begin{macrocode}
\def\to{\rightarrow}
\def\le{\leq}
\def\ge{\geq}
\def\neq{\ne}
\def\triangle{\mathord{\bigtriangleup}}
\def\bigcirc{\mdlgwhtcircle}
\def\circ{\vysmwhtcircle}
\def\bullet{\smblkcircle}
\def\mathyen{\yen}
\def\mathsterling{\sterling}
\def\diamond{\smwhtdiamond}
\def\emptyset{\varnothing}
\def\hbar{\hslash}
\def\land{\wedge}
\def\lor{\vee}
\def\owns{\ni}
\def\gets{\leftarrow}
\def\mathring{\ocirc}
\def\lnot{\neg}
\def\longdivision{\longdivisionsign}
%    \end{macrocode}
% These are somewhat odd: (and their usual Unicode uprightness does not match their amssymb glyphs)
%    \begin{macrocode}
\def\backepsilon{\upbackepsilon}
\def\eth{\matheth}
%    \end{macrocode}
%
% Due to the magic of OpenType math, big operators are automatically
% enlarged when necessary. Since there isn't a separate unicode glyph for
% `small integral', I'm not sure if there is a better way to do this:
%    \begin{macrocode}
\def\smallint{{\textstyle\int}\limits}
%    \end{macrocode}
%
% \begin{macro}{\colon}
% Define \cs{colon} as a mathpunct `|:|'.
% This is wrong: it should be \unichar{003A} {colon} instead!
% We hope no-one will notice.
%    \begin{macrocode}
\@ifpackageloaded{amsmath}{
  % define their own colon, perhaps I should just steal it. (It does look much better.)
}{
  \cs_set_protected:Npn \colon {
    \bool_if:NTF \g_um_literal_colon_bool {:} { \mathpunct{:} }
  }
}
%    \end{macrocode}
% \end{macro}
%
% \begin{macro}{\mathrm}
%    \begin{macrocode}
\def\mathrm{\mathup}
\let\mathfence\mathord
%    \end{macrocode}
% \end{macro}
%
% \begin{macro}{\digamma}
% \begin{macro}{\Digamma}
% I might end up just changing these in the table.
%    \begin{macrocode}
\def\digamma{\updigamma}
\def\Digamma{\upDigamma}
%    \end{macrocode}
% \end{macro}
% \end{macro}
%
%
% \subsection{Compatibility}
%
% We need to change \LaTeX's idea of the font used to typeset
% things like \cmd\sin\ and \cmd\cos:
%    \begin{macrocode}
\def\operator@font{\um_switchto_mathup:}
%    \end{macrocode}
%
% \begin{macro}{\um_tmpa:w}
% A scratch macro.
%    \begin{macrocode}
\chk_if_free_cs:N \um_tmpa:w
%    \end{macrocode}
% \end{macro}
%
% \begin{macro}{\um_check_and_fix:NNnnnn}
% \darg{command}
% \darg{factory command}
% \darg{parameter text}
% \darg{expected replacement text}
% \darg{new replacement text for \LuaTeX}
% \darg{new replacement text for \XeTeX}
% Tries to patch \meta{command}.
% If \meta{command} is undefined, do nothing.
% Otherwise it must be a macro with the given \meta{parameter text} and \meta{expected replacement text}, created by the given \meta{factory command} or equivalent.
% In this case it will be overwritten using the \meta{parameter text} and the \meta{new replacement text for \LuaTeX} or the \meta{new replacement text for \XeTeX}, depending on the engine.
% Otherwise issue a warning and don’t overwrite.
%    \begin{macrocode}
\cs_new_protected_nopar:Npn \um_check_and_fix:NNnnnn #1 #2 #3 #4 #5 #6 {
  \cs_if_exist:NT #1 {
    \token_if_macro:NTF #1 {
      \group_begin:
      #2 \um_tmpa:w #3 { #4 }
      \cs_if_eq:NNTF #1 \um_tmpa:w {
        \msg_info:nnx { unicode-math } { patch-macro }
          { \token_to_str:N #1 }
        \group_end:
        #2 #1 #3
%<XE>          { #6 }
%<LU>          { #5 }
      } {
        \msg_warning:nnxxx { unicode-math } { wrong-meaning }
          { \token_to_str:N #1 } { \token_to_meaning:N #1 }
          { \token_to_meaning:N \um_tmpa:w }
        \group_end:
      }
    } {
      \msg_warning:nnx { unicode-math } { macro-expected }
        { \token_to_str:N #1 }
    }
  }
}
%    \end{macrocode}
% \end{macro}
%
% \begin{macro}{\um_check_and_fix:NNnnn}
% \darg{command}
% \darg{factory command}
% \darg{parameter text}
% \darg{expected replacement text}
% \darg{new replacement text}
% Tries to patch \meta{command}.
% If \meta{command} is undefined, do nothing.
% Otherwise it must be a macro with the given \meta{parameter text} and \meta{expected replacement text}, created by the given \meta{factory command} or equivalent.
% In this case it will be overwritten using the \meta{parameter text} and the \meta{new replacement text}.
% Otherwise issue a warning and don’t overwrite.
%    \begin{macrocode}
\cs_new_protected_nopar:Npn \um_check_and_fix:NNnnn #1 #2 #3 #4 #5 {
  \um_check_and_fix:NNnnnn #1 #2 { #3 } { #4 } { #5 } { #5 }
}
%    \end{macrocode}
% \end{macro}
%
% \begin{macro}{\um_check_and_fix_luatex:NNnnn}
% \begin{macro}{\um_check_and_fix_luatex:cNnnn}
% \darg{command}
% \darg{factory command}
% \darg{parameter text}
% \darg{expected replacement text}
% \darg{new replacement text}
% Tries to patch \meta{command}.
% If \XeTeX\ is the current engine or \meta{command} is undefined, do nothing.
% Otherwise it must be a macro with the given \meta{parameter text} and \meta{expected replacement text}, created by the given \meta{factory command} or equivalent.
% In this case it will be overwritten using the \meta{parameter text} and the \meta{new replacement text}.
% Otherwise issue a warning and don’t overwrite.
%    \begin{macrocode}
\cs_new_protected_nopar:Npn \um_check_and_fix_luatex:NNnnn #1 #2 #3 #4 #5 {
  \luatex_if_engine:T {
    \um_check_and_fix:NNnnn #1 #2 { #3 } { #4 } { #5 }
  }
}
\cs_generate_variant:Nn \um_check_and_fix_luatex:NNnnn { c }
%    \end{macrocode}
% \end{macro}
% \end{macro}
%
% \paragraph{\pkg{url}}
% Simply need to get \pkg{url} in a state such that
% when it switches to math mode and enters \ascii\ characters, the maths
% setup (i.e., \pkg{unicode-math}) doesn't remap the symbols into Plane 1.
% Which is, of course, what \cs{mathup} is doing.
%
% This is the same as writing, e.g., |\def\UrlFont{\ttfamily\um_switchto_mathup:}|
% but activates automatically so old documents that might change the \cs{url}
% font still work correctly.
%    \begin{macrocode}
\AtEndOfPackageFile * {url} {
  \tl_put_left:Nn \Url@FormatString { \um_switchto_mathup: }
  \tl_put_right:Nn \UrlSpecials {
    \do\`{\mathchar`\`}
    \do\'{\mathchar`\'}
    \do\${\mathchar`\$}
    \do\&{\mathchar`\&}
  }
}
%    \end{macrocode}
%
% \paragraph{\pkg{amsmath}}
% Since the mathcode of |`\-| is greater than eight bits, this piece of |\AtBeginDocument| code from \pkg{amsmath} dies if we try and set the maths font in the preamble:
%    \begin{macrocode}
\AtEndOfPackageFile * {amsmath} {
%<*XE>
    \tl_remove_once:Nn \@begindocumenthook {
      \mathchardef\std@minus\mathcode`\-\relax
      \mathchardef\std@equal\mathcode`\=\relax
    }
    \def\std@minus{\Umathcharnum\Umathcodenum`\-\relax}
    \def\std@equal{\Umathcharnum\Umathcodenum`\=\relax}
%</XE>
%    \end{macrocode}
%
%    \begin{macrocode}
  \cs_set:Npn \@cdots {\mathinner{\cdots}}
  \cs_set_eq:NN \dotsb@ \cdots
%    \end{macrocode}
% This isn't as clever as the \pkg{amsmath} definition but I think it works:
%    \begin{macrocode}
%<*XE>
    \def \resetMathstrut@ {%
      \setbox\z@\hbox{$($}%)
      \ht\Mathstrutbox@\ht\z@ \dp\Mathstrutbox@\dp\z@
    }
%    \end{macrocode}
% The |subarray| environment uses inappropriate font dimensions.
%    \begin{macrocode}
    \um_check_and_fix:NNnnn \subarray \cs_set:Npn { #1 } {
      \vcenter
      \bgroup
      \Let@
      \restore@math@cr
      \default@tag
      \baselineskip \fontdimen 10~ \scriptfont \tw@
      \advance \baselineskip \fontdimen 12~ \scriptfont \tw@
      \lineskip \thr@@ \fontdimen 8~ \scriptfont \thr@@
      \lineskiplimit \lineskip
      \ialign
      \bgroup
      \ifx c #1 \hfil \fi
      $ \m@th \scriptstyle ## $
      \hfil
      \crcr
    } {
      \vcenter
      \c_group_begin_token
      \Let@
      \restore@math@cr
      \default@tag
      \skip_set:Nn \baselineskip {
%    \end{macrocode}
% Here we use stack top shift + stack bottom shift, which sounds reasonable.
%    \begin{macrocode}
        \um_stack_num_up:N \scriptstyle
        + \um_stack_denom_down:N \scriptstyle
      }
%    \end{macrocode}
% Here we use the minimum stack gap.
%    \begin{macrocode}
      \lineskip \um_stack_vgap:N \scriptstyle
      \lineskiplimit \lineskip
      \ialign
      \c_group_begin_token
      \token_if_eq_meaning:NNT c #1 { \hfil }
      \c_math_toggle_token
      \m@th
      \scriptstyle
      \c_parameter_token \c_parameter_token
      \c_math_toggle_token
      \hfil
      \crcr
    }
%</XE>
%    \end{macrocode}
% The roots need a complete rework.
%    \begin{macrocode}
  \um_check_and_fix_luatex:NNnnn \plainroot@ \cs_set_nopar:Npn { #1 \of #2 } {
    \setbox \rootbox \hbox {
      $ \m@th \scriptscriptstyle { #1 } $
    }
    \mathchoice
      { \r@@t \displaystyle      { #2 } }
      { \r@@t \textstyle         { #2 } }~
      { \r@@t \scriptstyle       { #2 } }
      { \r@@t \scriptscriptstyle { #2 } }
    \egroup
  } {
    \bool_if:nTF {
      \int_compare_p:nNn { \uproot@ } = { \c_zero }
      && \int_compare_p:nNn { \leftroot@ } = { \c_zero }
    } {
      \luatexUroot \l_um_radical_sqrt_tl { #1 } { #2 }
    } {
      \hbox_set:Nn \rootbox {
        \c_math_toggle_token
        \m@th
        \scriptscriptstyle { #1 }
        \c_math_toggle_token
      }
      \mathchoice
        { \r@@t \displaystyle      { #2 } }
        { \r@@t \textstyle         { #2 } }
        { \r@@t \scriptstyle       { #2 } }
        { \r@@t \scriptscriptstyle { #2 } }
    }
    \c_group_end_token
  }
  \um_check_and_fix:NNnnnn \r@@t \cs_set_nopar:Npn { #1 #2 } {
    \setboxz@h { $ \m@th #1 \sqrtsign { #2 } $ }
    \dimen@ \ht\z@
    \advance \dimen@ -\dp\z@
    \setbox\@ne \hbox { $ \m@th #1 \mskip \uproot@ mu $ }
    \advance \dimen@ by 1.667 \wd\@ne
    \mkern -\leftroot@ mu
    \mkern 5mu
    \raise .6\dimen@ \copy\rootbox
    \mkern -10mu
    \mkern \leftroot@ mu
    \boxz@
  } {
    \hbox_set:Nn \l_tmpa_box {
      \c_math_toggle_token
      \m@th
      #1
      \mskip \uproot@ mu
      \c_math_toggle_token
    }
    \luatexUroot \l_um_radical_sqrt_tl {
      \box_move_up:nn { \box_wd:N \l_tmpa_box } {
        \hbox:n {
          \c_math_toggle_token
          \m@th
          \mkern -\leftroot@ mu
          \box_use:N \rootbox
          \mkern \leftroot@ mu
          \c_math_toggle_token
        }
      }
    } {
      #2
    }
  } {
    \hbox_set:Nn \l_tmpa_box {
      \c_math_toggle_token
      \m@th
      #1
      \sqrtsign { #2 }
      \c_math_toggle_token
    }
    \hbox_set:Nn \l_tmpb_box {
      \c_math_toggle_token
      \m@th
      #1
      \mskip \uproot@ mu
      \c_math_toggle_token
    }
    \mkern -\leftroot@ mu
    \um_mathstyle_scale:Nnn #1 { \kern } {
      \fontdimen 63 \l_um_font
    }
    \box_move_up:nn {
      \box_wd:N \l_tmpb_box
      + (\box_ht:N \l_tmpa_box - \box_dp:N \l_tmpa_box)
      * \number \fontdimen 65 \l_um_font / 100
    } {
      \box_use:N \rootbox
    }
    \um_mathstyle_scale:Nnn #1 { \kern } {
      \fontdimen 64 \l_um_font
    }
    \mkern \leftroot@ mu
    \box_use_clear:N \l_tmpa_box
  }
}
%    \end{macrocode}
%
% \paragraph{\pkg{amsopn}}
% This code is to improve the output of analphabetic symbols in text of operator names (\cs{sin}, \cs{cos}, etc.). Just comment out the offending lines for now:
%    \begin{macrocode}
\AtEndOfPackageFile * {amsopn} {
  \cs_set:Npn \newmcodes@ {
    \mathcode`\'39\scan_stop:
    \mathcode`\*42\scan_stop:
    \mathcode`\."613A\scan_stop:
%%  \ifnum\mathcode`\-=45 \else
%%    \mathchardef\std@minus\mathcode`\-\relax
%%  \fi
    \mathcode`\-45\scan_stop:
    \mathcode`\/47\scan_stop:
    \mathcode`\:"603A\scan_stop:
  }
}
%    \end{macrocode}
% \paragraph{Symbols}
%    \begin{macrocode}
\cs_set:Npn \| {\Vert}
%    \end{macrocode}
% \cs{mathinner} items:
%    \begin{macrocode}
\cs_set:Npn \mathellipsis {\mathinner{\unicodeellipsis}}
\cs_set:Npn \cdots {\mathinner{\unicodecdots}}
%    \end{macrocode}
% \paragraph{Accents}
%    \begin{macrocode}
\cs_new_protected_nopar:Nn \um_setup_accents: {
  \cs_gset_protected_nopar:Npx \widehat {
    \um_accent:nnn {} { \um_symfont_tl } { "0302 }
  }
  \cs_gset_protected_nopar:Npx \widetilde {
    \um_accent:nnn {} { \um_symfont_tl } { "0303 }
  }
  \cs_gset_protected_nopar:Npx \overleftarrow {
    \um_accent:nnn {} { \um_symfont_tl } { "20D6 }
  }
  \cs_gset_protected_nopar:Npx \overrightarrow {
    \um_accent:nnn {} { \um_symfont_tl } { "20D7 }
  }
  \cs_gset_protected_nopar:Npx \overleftrightarrow {
    \um_accent:nnn {} { \um_symfont_tl } { "20E1 }
  }
  \cs_gset_protected_nopar:Npx \wideutilde {
    \um_accent:nnn {bottom} { \um_symfont_tl } { "0330 }
  }
  \cs_gset_protected_nopar:Npx \underrightharpoondown {
    \um_accent:nnn {bottom} { \um_symfont_tl } { "20EC }
  }
  \cs_gset_protected_nopar:Npx \underleftharpoondown {
    \um_accent:nnn {bottom} { \um_symfont_tl } { "20ED }
  }
  \cs_gset_protected_nopar:Npx \underleftarrow {
    \um_accent:nnn {bottom} { \um_symfont_tl } { "20EE }
  }
  \cs_gset_protected_nopar:Npx \underrightarrow {
    \um_accent:nnn {bottom} { \um_symfont_tl } { "20EF }
  }
  \cs_gset_protected_nopar:Npx \underleftrightarrow {
    \um_accent:nnn {bottom} { \um_symfont_tl } { "034D }
  }
}
%    \end{macrocode}
%
%    \begin{macrocode}
\cs_set_eq:NN \um_text_slash: \slash
\cs_set_protected:Npn \slash {
  \mode_if_math:TF {\mathslash} {\um_text_slash:}
}
%    \end{macrocode}
%
% \paragraph{\cs{not}}
% The situation of \cs{not} symbol is currently messy, in Unicode it is defined
% as a combining mark so naturally it should be treated as a math accent,
% however neither Lua\TeX\ nor \XeTeX\ correctly place it as it needs special
% treatment compared to other accents, furthermore a math accent changes the
% spacing of its nucleus, so \cs{not=} will be spaced as an ordinary not
% relational symbol, which is undesired.
%
% Here modify \cs{not} to a macro that tries to use predefined negated symbols,
% which would give better results in most cases, until there is more robust
% solution in the engines.
%
% This code is based on an answer to a TeX -- Stack Exchange question by Enrico
% Gregorio\footnote{\url{http://tex.stackexchange.com/a/47260/729}}.
%
%    \begin{macrocode}
\tl_new:N \l_not_token_name_tl

\cs_new:Npn \not_newnot:N #1 {
   \tl_set:Nx \l_not_token_name_tl { \token_to_str:N #1 }
   \tl_if_empty:xF { \tl_tail:V \l_not_token_name_tl } {
     \tl_set:Nx \l_not_token_name_tl { \tl_tail:V \l_not_token_name_tl }
   }
   \cs_if_exist:cTF { n \l_not_token_name_tl } {
     \use:c { n \l_not_token_name_tl }
   } {
     \cs_if_exist:cTF { not \l_not_token_name_tl } {
       \use:c { not \l_not_token_name_tl }
     } {
       \not_oldnot: #1 %\l_not_token_name_tl
     }
   }
}

\cs_new_protected_nopar:Nn \um_setup_negations: {
  \cs_set_eq:NN \not_oldnot: \not
  \cs_set_eq:NN \not \not_newnot:N

  \cs_gset:cpn { not= }    { \neq }
  \cs_gset:cpn { not< }    { \nless }
  \cs_gset:cpn { not> }    { \ngtr }
  \cs_gset:Npn  \ngets     { \nleftarrow }
  \cs_gset:Npn  \nsimeq    { \nsime }
  \cs_gset:Npn  \nequal    { \ne }
  \cs_gset:Npn  \nle       { \nleq }
  \cs_gset:Npn  \nge       { \ngeq }
  \cs_gset:Npn  \ngreater  { \ngtr }
  \cs_gset:Npn  \nforksnot { \forks }
}
%    \end{macrocode}
%
% \paragraph{\pkg{mathtools}}
% \pkg{mathtools}’s |\cramped| command and others that make use of its internal version use an incorrect font dimension.
%
%    \begin{macrocode}
\AtEndOfPackageFile * { mathtools } {
%<*XE>
    \chk_if_free_cs:N \g_um_empty_fam
    \newfam \g_um_empty_fam
    \um_check_and_fix:NNnnn
        \MT_cramped_internal:Nn \cs_set_nopar:Npn { #1 #2 }
    {
      \sbox \z@ {
        $
        \m@th
        #1
        \nulldelimiterspace = \z@
        \radical \z@ { #2 }
        $
      }
      \ifx #1 \displaystyle
        \dimen@ = \fontdimen 8 \textfont 3
        \advance \dimen@ .25 \fontdimen 5 \textfont 2
      \else
        \dimen@ = 1.25 \fontdimen 8
        \ifx #1 \textstyle
          \textfont
        \else
          \ifx #1 \scriptstyle
            \scriptfont
          \else
            \scriptscriptfont
          \fi
        \fi
        3
      \fi
      \advance \dimen@ -\ht\z@
      \ht\z@ = -\dimen@
      \box\z@
    }
%    \end{macrocode}
% The \XeTeX\ version is pretty similar to the legacy version, only using the correct font dimensions.
% Note we used `\verb|\XeTeXradical|' with a newly-allocated empty family to make sure that the radical rule width is not set.
%    \begin{macrocode}
    {
      \hbox_set:Nn \l_tmpa_box {
        \color@setgroup
        \c_math_toggle_token
        \m@th
        #1
        \dim_zero:N \nulldelimiterspace
        \XeTeXradical \g_um_empty_fam \c_zero { #2 }
        \c_math_toggle_token
        \color@endgroup
      }
      \box_set_ht:Nn \l_tmpa_box {
        \box_ht:N \l_tmpa_box
%    \end{macrocode}
% Here we use the radical vertical gap.
%    \begin{macrocode}
        - \um_radical_vgap:N #1
      }
      \box_use_clear:N \l_tmpa_box
    }
%</XE>
%    \end{macrocode}
% \begin{macro}{\overbracket}
% \begin{macro}{\underbracket}
%   \pkg{mathtools}’s |\overbracket| and |\underbracket| take optional
%   arguments and are defined in terms of rules, so we keep them, and rename
%   ours to |\Uoverbracket| and |\Uunderbracket|.
%    \begin{macrocode}
\AtEndOfPackageFile * { mathtools } {
    \let\MToverbracket =\overbracket
    \let\MTunderbracket=\underbracket

    \AtBeginDocument {
        \msg_warning:nn { unicode-math } { mathtools-overbracket }

	\def\downbracketfill#1#2{%
%    \end{macrocode}
%   Original definition used the height of |\braceld| which is not available
%   with Unicode fonts, so we are hard coding the $5/18$ex suggested by
%   \pkg{mathtools}’s documentation.
%    \begin{macrocode}
            \edef\l_MT_bracketheight_fdim{.27ex}%
            \downbracketend{#1}{#2}
            \leaders \vrule \@height #1 \@depth \z@ \hfill
            \downbracketend{#1}{#2}%
        }
	\def\upbracketfill#1#2{%
            \edef\l_MT_bracketheight_fdim{.27ex}%
            \upbracketend{#1}{#2}
            \leaders \vrule \@height \z@ \@depth #1 \hfill
            \upbracketend{#1}{#2}%
        }
	\let\Uoverbracket =\overbracket
	\let\Uunderbracket=\underbracket
        \let\overbracket  =\MToverbracket
        \let\underbracket =\MTunderbracket
    }
}
%    \end{macrocode}
% \end{macro}
% \end{macro}
%
% \begin{macro}{\dblcolon}
% \begin{macro}{\coloneqq}
% \begin{macro}{\Coloneqq}
% \begin{macro}{\eqqcolon}
%   \pkg{mathtools} defines several commands as combinations of colons and
%   other characters, but with meanings incompatible to \pkg{unicode-math}.
%   Thus we issue a warning.  Because \pkg{mathtools} uses
%   \cmd{\providecommand} \cmd{\AtBeginDocument}, we can just define the
%   offending commands here.
%    \begin{macrocode}
  \msg_warning:nn { unicode-math } { mathtools-colon }
  \NewDocumentCommand \dblcolon { } { \Colon }
  \NewDocumentCommand \coloneqq { } { \coloneq }
  \NewDocumentCommand \Coloneqq { } { \Coloneq }
  \NewDocumentCommand \eqqcolon { } { \eqcolon }
}
%    \end{macrocode}
% \end{macro}
% \end{macro}
% \end{macro}
% \end{macro}
%
% \paragraph{\pkg{colonequals}}
%
% \begin{macro}{\ratio}
% \begin{macro}{\coloncolon}
% \begin{macro}{\minuscolon}
% \begin{macro}{\colonequals}
% \begin{macro}{\equalscolon}
% \begin{macro}{\coloncolonequals}
%   Similarly to \pkg{mathtools}, the \pkg{colonequals} defines several colon
%   combinations.  Fortunately there are no name clashes, so we can just
%   overwrite their definitions.
%    \begin{macrocode}
\AtEndOfPackageFile * { colonequals } {
  \msg_warning:nn { unicode-math } { colonequals }
  \RenewDocumentCommand \ratio { } { \mathratio }
  \RenewDocumentCommand \coloncolon { } { \Colon }
  \RenewDocumentCommand \minuscolon { } { \dashcolon }
  \RenewDocumentCommand \colonequals { } { \coloneq }
  \RenewDocumentCommand \equalscolon { } { \eqcolon }
  \RenewDocumentCommand \coloncolonequals { } { \Coloneq }
}
%    \end{macrocode}
% \end{macro}
% \end{macro}
% \end{macro}
% \end{macro}
% \end{macro}
% \end{macro}
%
%    \begin{macrocode}
\ExplSyntaxOff
%</package&(XE|LU)>
%    \end{macrocode}
%
% \section{Error messages}
% \seclabel{codemsg}
%
% These are defined at the beginning of the package, but we leave their
% definition until now in the source to keep them out of the way.
%
%    \begin{macrocode}
%<*msg>
%    \end{macrocode}
%
% Wrapper functions:
%    \begin{macrocode}
\cs_new:Npn \um_warning:n { \msg_warning:nn {unicode-math} }
\cs_new:Npn \um_log:n   { \msg_log:nn   {unicode-math} }
\cs_new:Npn \um_log:nx  { \msg_log:nnx  {unicode-math} }
%    \end{macrocode}
%
%    \begin{macrocode}
\msg_new:nnn {unicode-math} {no-tfrac}
{
  Small~ fraction~ command~ \protect\tfrac\ not~ defined.\\
  Load~ amsmath~ or~ define~ it~ manually~ before~ loading~ unicode-math.
}
\msg_new:nnn {unicode-math} {default-math-font}
{
  Defining~ the~ default~ maths~ font~ as~ '\l_um_fontname_tl'.
}
\msg_new:nnn {unicode-math} {setup-implicit}
{
  Setup~ alphabets:~ implicit~ mode.
}
\msg_new:nnn {unicode-math} {setup-explicit}
{
  Setup~ alphabets:~ explicit~ mode.
}
\msg_new:nnn {unicode-math} {alph-initialise}
{
  Initialising~ \@backslashchar math#1.
}
\msg_new:nnn {unicode-math} {setup-alph}
{
  Setup~ alphabet:~ #1.
}
\msg_new:nnn { unicode-math } { missing-alphabets }
  {
    Missing~math~alphabets~in~font~ "\fontname\l_um_font" \\ \\
    \seq_map_function:NN \l_um_missing_alph_seq \um_print_indent:n
  }
\cs_new:Nn \um_print_indent:n { \space\space\space\space #1 \\ }
\msg_new:nnn {unicode-math} {macro-expected}
{
  I've~ expected~ that~ #1~ is~ a~ macro,~ but~ it~ isn't.
}
\msg_new:nnn {unicode-math} {wrong-meaning}
{
  I've~ expected~ #1~ to~ have~ the~ meaning~ #3,~ but~ it~ has~ the~ meaning~ #2.
}
\msg_new:nnn {unicode-math} {patch-macro}
{
  I'm~ going~ to~ patch~ macro~ #1.
}
\msg_new:nnn { unicode-math } { mathtools-overbracket } {
  Using~ \token_to_str:N \overbracket\ and~
         \token_to_str:N \underbracke\ from~
	 `mathtools'~ package.\\
  \\
  Use~ \token_to_str:N \Uoverbracket\ and~
       \token_to_str:N \Uunderbracke\ for~
       original~ `unicode-math'~ definition.
}
\msg_new:nnn { unicode-math } { mathtools-colon } {
  I'm~ going~ to~ overwrite~ the~ following~ commands~ from~
  the~ `mathtools'~ package: \\ \\
  \ \ \ \ \token_to_str:N \dblcolon,~
  \token_to_str:N \coloneqq,~
  \token_to_str:N \Coloneqq,~
  \token_to_str:N \eqqcolon. \\ \\
  Note~ that~ since~ I~ won't~ overwrite~ the~ other~ colon-like~
  commands,~ using~ them~ will~ lead~ to~ inconsistencies.
}
\msg_new:nnn { unicode-math } { colonequals } {
  I'm~ going~ to~ overwrite~ the~ following~ commands~ from~
  the~ `colonequals'~ package: \\ \\
  \ \ \ \ \token_to_str:N \ratio,~
          \token_to_str:N \coloncolon,~
          \token_to_str:N \minuscolon, \\
  \ \ \ \ \token_to_str:N \colonequals,~
          \token_to_str:N \equalscolon,~
          \token_to_str:N \coloncolonequals. \\ \\
  Note~ that~ since~ I~ won't~ overwrite~ the~ other~ colon-like~
  commands,~ using~ them~ will~ lead~ to~ inconsistencies.~
  Furthermore,~ changing~ \token_to_str:N \colonsep \c_space_tl
  or~ \token_to_str:N \doublecolonsep \c_space_tl won't~ have~
  any~ effect~ on~ the~ re-defined~ commands.
}
%    \end{macrocode}
%
%    \begin{macrocode}
%</msg>
%    \end{macrocode}
%
% The end.
%
%
% \section{\STIX\ table data extraction}\label{part:awk}
%\iffalse
%<*awk>
%\fi
%
% The source for the \TeX\ names for the very large number of mathematical
% glyphs are provided via Barbara Beeton's table file for the \STIX\ project
% (|ams.org/STIX|). A version is located at
% |http://www.ams.org/STIX/bnb/stix-tbl.asc|
% but check |http://www.ams.org/STIX/| for more up-to-date info.
%
% This table is converted into a form suitable for reading by \XeTeX.
% A single file is produced containing all (more than 3298) symbols.
% Future optimisations might include generating various (possibly overlapping) subsets
% so not all definitions must be read just to redefine a small range of symbols.
% Performance for now seems to be acceptable without such measures.
%
% This file is currently developed outside this DTX file. It will be
% incorporated when the final version is ready. (I know this is not how
% things are supposed to work!)
%
%    \begin{macrocode}
< See stix-extract.sh for now. >
%    \end{macrocode}
%\iffalse
%</awk>
%\fi
%
% \appendix
%
% \section{Documenting maths support in the NFSS}
%
% In the following, \meta{NFSS decl.} stands for something like |{T1}{lmr}{m}{n}|.
%
% \begin{description}
% \item[Maths symbol fonts] Fonts for symbols: $\propto$, $\leq$, $\rightarrow$
%
% \cmd\DeclareSymbolFont\marg{name}\meta{NFSS decl.}\\
% Declares a named maths font such as |operators| from which symbols are defined with \cmd\DeclareMathSymbol.
%
% \item[Maths alphabet fonts] Fonts for {\font\1=cmmi10 at 10pt\1 ABC}\,–\,{\font\1=cmmi10 at 10pt\1 xyz}, {\font\1=eufm10 at 10pt\1 ABC}\,–\,{\font\1=cmsy10 at 10pt\1 XYZ}, etc.
%
% \cmd\DeclareMathAlphabet\marg{cmd}\meta{NFSS decl.}
%
% For commands such as \cmd\mathbf, accessed
% through maths mode that are unaffected by the current text font, and which are used for
% alphabetic symbols in the \ascii\ range.
%
% \cmd\DeclareSymbolFontAlphabet\marg{cmd}\marg{name}
%
% Alternative (and optimisation) for \cmd\DeclareMathAlphabet\ if a single font is being used
% for both alphabetic characters (as above) and symbols.
%
% \item[Maths `versions'] Different maths weights can be defined with the following, switched
% in text with the \cmd\mathversion\marg{maths version} command.
%
% \cmd\SetSymbolFont\marg{name}\marg{maths version}\meta{NFSS decl.}\\
% \cmd\SetMathAlphabet\marg{cmd}\marg{maths version}\meta{NFSS decl.}
%
% \item[Maths symbols] Symbol definitions in maths for both characters (=) and macros (\cmd\eqdef):
% \cmd\DeclareMathSymbol\marg{symbol}\marg{type}\marg{named font}\marg{slot}
% This is the macro that actually defines which font each symbol comes from and how they behave.
% \end{description}
% Delimiters and radicals use wrappers around \TeX's \cmd\delimiter/\cmd\radical\ primitives,
% which are re-designed in \XeTeX. The syntax used in \LaTeX's NFSS is therefore not so relevant here.
% \begin{description}
% \item[Delimiters] A special class of maths symbol which enlarge themselves in certain contexts.
%
% \cmd\DeclareMathDelimiter\marg{symbol}\marg{type}\marg{sym.\ font}\marg{slot}\marg{sym.\ font}\marg{slot}
%
% \item[Radicals] Similar to delimiters (\cmd\DeclareMathRadical\ takes the same syntax) but
% behave `weirdly'.
% \end{description}
% In those cases, glyph slots in \emph{two} symbol fonts are required; one for the small (`regular') case,
% the other for situations when the glyph is larger. This is not the case in \XeTeX.
%
% Accents are not included yet.
%
% \paragraph{Summary}
%
% For symbols, something like:
% \begin{verbatim}
% \def\DeclareMathSymbol#1#2#3#4{
%   \global\mathchardef#1"\mathchar@type#2
%     \expandafter\hexnumber@\csname sym#2\endcsname
%     {\hexnumber@{\count\z@}\hexnumber@{\count\tw@}}}
% \end{verbatim}
% For characters, something like:
% \begin{verbatim}
% \def\DeclareMathSymbol#1#2#3#4{
%   \global\mathcode`#1"\mathchar@type#2
%     \expandafter\hexnumber@\csname sym#2\endcsname
%     {\hexnumber@{\count\z@}\hexnumber@{\count\tw@}}}
% \end{verbatim}
%
% \section{Legacy \TeX\ font dimensions}
%
% \centerline{%
% \begin{tabular}[t]{@{}lp{4cm}@{}}
% \toprule
% \multicolumn{2}{@{}c@{}}{Text fonts} \\
% \midrule
% $\phi_1$ & slant per pt                \\
% $\phi_2$ & interword space             \\
% $\phi_3$ & interword stretch           \\
% $\phi_4$ & interword shrink            \\
% $\phi_5$ & x-height                    \\
% $\phi_6$ & quad width                  \\
% $\phi_7$ & extra space                 \\
% $\phi_8$ & cap height (\XeTeX\ only)   \\
% \bottomrule
% \end{tabular}
% \quad
% \begin{tabular}[t]{@{}lp{4cm}@{}}
% \toprule
% \multicolumn{2}{@{}c@{}}{Maths font, \cs{fam}2} \\
% \midrule
% $\sigma_5$    & x height                    \\
% $\sigma_6$    & quad                        \\
% $\sigma_8$    & num1                        \\
% $\sigma_9$    & num2                        \\
% $\sigma_{10}$ & num3                        \\
% $\sigma_{11}$ & denom1                      \\
% $\sigma_{12}$ & denom2                      \\
% $\sigma_{13}$ & sup1                        \\
% $\sigma_{14}$ & sup2                        \\
% $\sigma_{15}$ & sup3                        \\
% $\sigma_{16}$ & sub1                        \\
% $\sigma_{17}$ & sub2                        \\
% $\sigma_{18}$ & sup drop                    \\
% $\sigma_{19}$ & sub drop                    \\
% $\sigma_{20}$ & delim1                      \\
% $\sigma_{21}$ & delim2                      \\
% $\sigma_{22}$ & axis height                 \\
% \bottomrule
% \end{tabular}
% \quad
% \begin{tabular}[t]{@{}lp{4cm}@{}}
% \toprule
% \multicolumn{2}{@{}c@{}}{Maths font, \cs{fam}3} \\
% \midrule
% $\xi_8$    & default rule thickness      \\
% $\xi_9$    & big op spacing1             \\
% $\xi_{10}$ & big op spacing2             \\
% $\xi_{11}$ & big op spacing3             \\
% $\xi_{12}$ & big op spacing4             \\
% $\xi_{13}$ & big op spacing5             \\
% \bottomrule
% \end{tabular}
% }
%
%
% \section{\Hologo{XeTeX} math font dimensions}
%
% These are the extended \cmd\fontdimen s available for suitable fonts
% in \XeTeX. Note that Lua\TeX\ takes an alternative route, and this package
% will eventually provide a wrapper interface to the two (I hope).
%
% \newcounter{mfdimen}
% \setcounter{mfdimen}{9}
% \newcommand\mathfontdimen[2]{^^A
%   \stepcounter{mfdimen}^^A
%   \themfdimen & {\scshape\small #1} & #2\vspace{0.5ex} \tabularnewline}
%
% \begin{longtable}{
%   @{}c>{\raggedright\parfillskip=0pt}p{4cm}>{\raggedright}p{7cm}@{}}
% \toprule \cmd\fontdimen & Dimension name & Description\tabularnewline\midrule \endhead
% \bottomrule\endfoot
% \mathfontdimen{Script\-Percent\-Scale\-Down}
% {Percentage of scaling down for script level 1. Suggested value: 80\%.}
% \mathfontdimen{Script\-Script\-Percent\-Scale\-Down}
% {Percentage of scaling down for script level 2 (Script\-Script). Suggested value: 60\%.}
% \mathfontdimen{Delimited\-Sub\-Formula\-Min\-Height}
% {Minimum height required for a delimited expression to be treated as a subformula. Suggested value: normal line height\,×\,1.5.}
% \mathfontdimen{Display\-Operator\-Min\-Height}
% {Minimum height of n-ary operators (such as integral and summation) for formulas in display mode.}
% \mathfontdimen{Math\-Leading}
% {White space to be left between math formulas to ensure proper line spacing. For example, for applications that treat line gap as a part of line ascender, formulas with ink  going above (os2.sTypoAscender + os2.sTypoLineGap – MathLeading) or with ink going below os2.sTypoDescender will result in increasing line height.}
% \mathfontdimen{Axis\-Height}
% {Axis height of the font. }
% \mathfontdimen{Accent\-Base\-Height}
% {Maximum (ink) height of accent base that does not require raising the accents. Suggested: x-height of the font (os2.sxHeight) plus any possible overshots. }
% \mathfontdimen{Flattened\-Accent\-Base\-Height}
% {Maximum (ink) height of accent base that does not require flattening the accents. Suggested: cap height of the font (os2.sCapHeight).}
% \mathfontdimen{Subscript\-Shift\-Down}
% {The standard shift down applied to subscript elements. Positive for moving in the downward direction. Suggested: os2.ySubscriptYOffset.}
% \mathfontdimen{Subscript\-Top\-Max}
% {Maximum allowed height of the (ink) top of subscripts that does not require moving subscripts further down. Suggested: /5 x-height.}
% \mathfontdimen{Subscript\-Baseline\-Drop\-Min}
% {Minimum allowed drop of the baseline of subscripts relative to the (ink) bottom of the base. Checked for bases that are treated as a box or extended shape. Positive for subscript baseline dropped below the base bottom.}
% \mathfontdimen{Superscript\-Shift\-Up}
% {Standard shift up applied to superscript elements. Suggested: os2.ySuperscriptYOffset.}
% \mathfontdimen{Superscript\-Shift\-Up\-Cramped}
% {Standard shift of superscripts relative to the base, in cramped style.}
% \mathfontdimen{Superscript\-Bottom\-Min}
% {Minimum allowed height of the (ink) bottom of superscripts that does not require moving subscripts further up. Suggested: ¼ x-height.}
% \mathfontdimen{Superscript\-Baseline\-Drop\-Max}
% {Maximum allowed drop of the baseline of superscripts relative to the (ink) top of the base. Checked for bases that are treated as a box or extended shape. Positive for superscript baseline below the base top.}
% \mathfontdimen{Sub\-Superscript\-Gap\-Min}
% {Minimum gap between the superscript and subscript ink. Suggested: 4×default rule thickness.}
% \mathfontdimen{Superscript\-Bottom\-Max\-With\-Subscript}
% {The maximum level to which the (ink) bottom of superscript can be pushed to increase the gap between superscript and subscript, before subscript starts being moved down.
% Suggested: /5 x-height.}
% \mathfontdimen{Space\-After\-Script}
% {Extra white space to be added after each subscript and superscript. Suggested: 0.5pt for a 12 pt font.}
% \mathfontdimen{Upper\-Limit\-Gap\-Min}
% {Minimum gap between the (ink) bottom of the upper limit, and the (ink) top of the base operator. }
% \mathfontdimen{Upper\-Limit\-Baseline\-Rise\-Min}
% {Minimum distance between baseline of upper limit and (ink) top of the base operator.}
% \mathfontdimen{Lower\-Limit\-Gap\-Min}
% {Minimum gap between (ink) top of the lower limit, and (ink) bottom of the base operator.}
% \mathfontdimen{Lower\-Limit\-Baseline\-Drop\-Min}
% {Minimum distance between baseline of the lower limit and (ink) bottom of the base operator.}
% \mathfontdimen{Stack\-Top\-Shift\-Up}
% {Standard shift up applied to the top element of a stack.}
% \mathfontdimen{Stack\-Top\-Display\-Style\-Shift\-Up}
% {Standard shift up applied to the top element of a stack in display style.}
% \mathfontdimen{Stack\-Bottom\-Shift\-Down}
% {Standard shift down applied to the bottom element of a stack. Positive for moving in the downward direction.}
% \mathfontdimen{Stack\-Bottom\-Display\-Style\-Shift\-Down}
% {Standard shift down applied to the bottom element of a stack in display style. Positive for moving in the downward direction.}
% \mathfontdimen{Stack\-Gap\-Min}
% {Minimum gap between (ink) bottom of the top element of a stack, and the (ink) top of the bottom element. Suggested: 3×default rule thickness.}
% \mathfontdimen{Stack\-Display\-Style\-Gap\-Min}
% {Minimum gap between (ink) bottom of the top element of a stack, and the (ink) top of the bottom element in display style. Suggested: 7×default rule thickness.}
% \mathfontdimen{Stretch\-Stack\-Top\-Shift\-Up}
% {Standard shift up applied to the top element of the stretch stack.}
% \mathfontdimen{Stretch\-Stack\-Bottom\-Shift\-Down}
% {Standard shift down applied to the bottom element of the stretch stack. Positive for moving in the downward direction.}
% \mathfontdimen{Stretch\-Stack\-Gap\-Above\-Min}
% {Minimum gap between the ink of the stretched element, and the (ink) bottom of the element above. Suggested: Upper\-Limit\-Gap\-Min}
% \mathfontdimen{Stretch\-Stack\-Gap\-Below\-Min}
% {Minimum gap between the ink of the stretched element, and the (ink) top of the element below. Suggested: Lower\-Limit\-Gap\-Min.}
% \mathfontdimen{Fraction\-Numerator\-Shift\-Up}
% {Standard shift up applied to the numerator. }
% \mathfontdimen{Fraction\-Numerator\-Display\-Style\-Shift\-Up}
% {Standard shift up applied to the numerator in display style. Suggested: Stack\-Top\-Display\-Style\-Shift\-Up.}
% \mathfontdimen{Fraction\-Denominator\-Shift\-Down}
% {Standard shift down applied to the denominator. Positive for moving in the downward direction.}
% \mathfontdimen{Fraction\-Denominator\-Display\-Style\-Shift\-Down}
% {Standard shift down applied to the denominator in display style. Positive for moving in the downward direction. Suggested: Stack\-Bottom\-Display\-Style\-Shift\-Down.}
% \mathfontdimen{Fraction\-Numerator\-Gap\-Min}
% {Minimum tolerated gap between the (ink) bottom of the numerator and the ink of the fraction bar. Suggested: default rule thickness}
% \mathfontdimen{Fraction\-Num\-Display\-Style\-Gap\-Min}
% {Minimum tolerated gap between the (ink) bottom of the numerator and the ink of the fraction bar in display style. Suggested: 3×default rule thickness.}
% \mathfontdimen{Fraction\-Rule\-Thickness}
% {Thickness of the fraction bar. Suggested: default rule thickness.}
% \mathfontdimen{Fraction\-Denominator\-Gap\-Min}
% {Minimum tolerated gap between the (ink) top of the denominator and the ink of the fraction bar. Suggested: default rule thickness}
% \mathfontdimen{Fraction\-Denom\-Display\-Style\-Gap\-Min}
% {Minimum tolerated gap between the (ink) top of the denominator and the ink of the fraction bar in display style. Suggested: 3×default rule thickness.}
% \mathfontdimen{Skewed\-Fraction\-Horizontal\-Gap}
% {Horizontal distance between the top and bottom elements of a skewed fraction.}
% \mathfontdimen{Skewed\-Fraction\-Vertical\-Gap}
% {Vertical distance between the ink of the top and bottom elements of a skewed fraction.}
% \mathfontdimen{Overbar\-Vertical\-Gap}
% {Distance between the overbar and the (ink) top of he base. Suggested: 3×default rule thickness.}
% \mathfontdimen{Overbar\-Rule\-Thickness}
% {Thickness of overbar. Suggested: default rule thickness.}
% \mathfontdimen{Overbar\-Extra\-Ascender}
% {Extra white space reserved above the overbar. Suggested: default rule thickness.}
% \mathfontdimen{Underbar\-Vertical\-Gap}
% {Distance between underbar and (ink) bottom of the base. Suggested: 3×default rule thickness.}
% \mathfontdimen{Underbar\-Rule\-Thickness}
% {Thickness of underbar. Suggested: default rule thickness.}
% \mathfontdimen{Underbar\-Extra\-Descender}
% {Extra white space reserved below the underbar. Always positive. Suggested: default rule thickness.}
% \mathfontdimen{Radical\-Vertical\-Gap}
% {Space between the (ink) top of the expression and the bar over it. Suggested: 1¼ default rule thickness.}
% \mathfontdimen{Radical\-Display\-Style\-Vertical\-Gap}
% {Space between the (ink) top of the expression and the bar over it. Suggested: default rule thickness + ¼ x-height. }
% \mathfontdimen{Radical\-Rule\-Thickness}
% {Thickness of the radical rule. This is the thickness of the rule in designed or constructed radical signs. Suggested: default rule thickness.}
% \mathfontdimen{Radical\-Extra\-Ascender}
% {Extra white space reserved above the radical. Suggested: Radical\-Rule\-Thickness.}
% \mathfontdimen{Radical\-Kern\-Before\-Degree}
% {Extra horizontal kern before the degree of a radical, if such is present. Suggested: 5/18 of em.}
% \mathfontdimen{Radical\-Kern\-After\-Degree}
% {Negative kern after the degree of a radical, if such is present. Suggested: −10/18 of em.}
% \mathfontdimen{Radical\-Degree\-Bottom\-Raise\-Percent}
% {Height of the bottom of the radical degree, if such is present, in proportion to the ascender of the radical sign. Suggested: 60\%.}
% \end{longtable}
%
% \Finale
%
% \iffalse
%
%<*dtx-style>
%    \begin{macrocode}
\ProvidesPackage{dtx-style}

\GetFileInfo{\jobname.dtx}
\let\umfiledate\filedate
\let\umfileversion\fileversion

\CheckSum{0}
\EnableCrossrefs
\CodelineIndex

\errorcontextlines=999

\def\@dotsep{1000}
\setcounter{tocdepth}{2}
\setlength\columnseprule{0.4pt}
\renewcommand\tableofcontents{\relax
  \begin{multicols}{2}[\section*{\contentsname}]\relax
    \@starttoc{toc}\relax
  \end{multicols}}

\setcounter{IndexColumns}{2}
\renewenvironment{theglossary}
  {\small\list{}{}
     \item\relax
     \glossary@prologue\GlossaryParms
     \let\item\@idxitem \ignorespaces
     \def\pfill{\hspace*{\fill}}}
  {\endlist}

\usepackage[svgnames]{xcolor}
\usepackage{array,booktabs,calc,enumitem,fancyvrb,graphicx,ifthen,longtable,refstyle,subfig,topcapt,url,varioref,underscore}
\setcounter{LTchunksize}{100}
\usepackage[slash-delimiter=frac,nabla=literal]{unicode-math}
\usepackage{metalogo,hologo}

%\usepackage[rm,small]{titlesec}

\setmainfont[Ligatures=TeX]{TeX Gyre Pagella}
\setsansfont[Scale=MatchLowercase,Ligatures=TeX]{Candara}
\setmonofont[Scale=MatchLowercase]{Consolas}
\setmathfont{texgyrepagella-math.otf}
\newfontface\umfont{XITS Math}

\usepackage{hypdoc}
\hypersetup{linktocpage}

\usepackage{minitoc}

\linespread{1.069}      % A bit more space between lines
\frenchspacing         % Remove ugly extra space after punctuation

\definecolor{niceblue}{rgb}{0.2,0.4,0.8}

\def\theCodelineNo{\textcolor{niceblue}{\sffamily\tiny\arabic{CodelineNo}}}

\newcommand*\name[1]{{#1}}
\newcommand*\pkg[1]{\textsf{#1}}
\newcommand*\feat[1]{\texttt{#1}}
\newcommand*\opt[1]{\texttt{#1}}

\newcommand*\note[1]{\unskip\footnote{#1}}

\let\latin\textit
\def\eg{\latin{e.g.}}
\def\Eg{\latin{E.g.}}
\def\ie{\latin{i.e.}}
\def\etc{\@ifnextchar.{\latin{etc}}{\latin{etc.}\@}}

\def\STIX{\textsc{stix}}
\def\MacOSX{Mac~OS~X}
\def\ascii{\textsc{ascii}}
\def\OMEGA{Omega}

\newcounter{argument}
\g@addto@macro\endmacro{\setcounter{argument}{0}}
\newcommand*\darg[1]{%
  \stepcounter{argument}%
  {\ttfamily\char`\#\theargument~:~}#1\par\noindent\ignorespaces
}
\newcommand*\doarg[1]{%
  \stepcounter{argument}%
  {\ttfamily\makebox[0pt][r]{[}\char`\#\theargument]:~}#1\par\noindent\ignorespaces
}

\newcommand\codeline[1]{\par{\centering#1\par\noindent}\ignorespaces}

\newcommand\unichar[1]{\textsc{u}+\texttt{\small#1}}

\setlength\parindent{2em}

\def \MakePrivateLetters {%
  \catcode`\@=11\relax
  \catcode`\_=11\relax
  \catcode`\:=11\relax
}

\def\partname{Part}

%    \end{macrocode}
%</dtx-style>
%\fi
%
% \typeout{*************************************************************}
% \typeout{*}
% \typeout{* To finish the installation you have to move the following}
% \typeout{* file into a directory searched by XeTeX:}
% \typeout{*}
% \typeout{* \space\space\space unicode-math.sty}
% \typeout{*}
% \typeout{*************************************************************}
%
\endinput


